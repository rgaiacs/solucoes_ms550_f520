%        File: lista1.tex
%     Created: qua mar 07 04:00  2012 HodB
% Last Change: qua mar 07 04:00  2012 HodB
%
\documentclass[a4paper,12pt,answers]{exam}
\usepackage[top=3cm, bottom=3cm, left=2cm, right=2cm]{geometry}
\usepackage[utf8]{inputenc}
\usepackage[brazil]{babel}
\usepackage{amsmath}
\usepackage{hyperref}

% Customiza\c{c}\~{a}o da classe exam
\firstpageheader{MS550, F520}{Solu\c{c}\~{a}o da Lista 1}{1º semestre de 2012}
\firstpageheadrule
\footer{Dispon\'{i}vel em \\% Filename: repository.tex
% 
% This code is part of 'Solutions for MS550, M\'{e}todos de Matem\'{a}tica Aplicada I, and F520, M\'{e}todos Matem\'{a}ticos da F\'{i}sica I'
% 
% Description: This file keeps the url of the repository.
% 
% Created: 07.03.12 04:00:00 PM
% Last Change: 30.05.12 04:40:25 PM
% 
% Authors:
% - Raniere Silva (2012): initial version
% 
% Copyright (c) 2012 Raniere Silva <r.gaia.cs@gmail.com>
% 
% This work is licensed under the Creative Commons Attribution-ShareAlike 3.0 Unported License. To view a copy of this license, visit http://creativecommons.org/licenses/by-sa/3.0/ or send a letter to Creative Commons, 444 Castro Street, Suite 900, Mountain View, California, 94041, USA.
%
% This work is distributed in the hope that it will be useful, but WITHOUT ANY WARRANTY; without even the implied warranty of MERCHANTABILITY or FITNESS FOR A PARTICULAR PURPOSE.
%
\url{https://github.com/r-gaia-cs/solucoes_listas_metodos}
}{}{Reportar erros para \\\href{mailto:r.gaia.cs@gmail.com}{r.gaia.cs@gmail.com}
}
\footrule 
\pagestyle{foot}
\renewcommand{\solutiontitle}{\noindent\textbf{Solu\c{c}\~{a}o:}\enspace}

% Novos comandos
\newcommand{\devp}[2]{\frac{\partial #1}{\partial #2}}
\begin{document}
\thispagestyle{headandfoot}
\begin{questions}
  \question Seja $\vec{V} = z\vec{i} - 2x\vec{j} + y \vec{k}$. Mostre que as componentes de $\vec{V}$ em coordenadas cil\'{i}ndricas circulares s\~{a}o dadas por
  \begin{align*}
	V_\rho &= z \cos \theta - 2 \rho \cos \theta \sin \theta, \\
	V_\theta &= -z \sin \theta - 2 \rho \cos^2 \theta, \\
	V_z &= \rho \sin \theta
  \end{align*}
  \begin{solution}
	
  \end{solution}

  \question Seja o campo vetorial
  \[
  \vec{V} = V_\rho(\rho, \theta) \vec{e}_\rho + V_\theta(\rho, \theta) \vec{e}_\theta.
  \]
  Mostre que $\nabla \times \vec{V} = \mbox{rot } \vec{V}$ possui componente apenas na dire\c{c}\~{a}o $z$.
  \begin{solution}
	
  \end{solution}

  \question Seja o campo vetorial $\vec{V} = \rho \vec{e}_z$. Mostre que
  \begin{align*}
	\nabla \times \vec{V} = -\vec{e}_\theta, \\
	\nabla \times (\vec{V} \times (\nabla \times \vec{V})) = 0.
  \end{align*}
  \begin{solution}
	
  \end{solution}

  \question Calcule $\devp{\vec{e}_{q_i}}{q_j}$ com $i,j = 1, 2, 3$ quando $q_i$ s\~{a}o as coordenadas cil\'{i}ndricas e mostre que as únicas derivadas n\~{a}o nulas s\~{a}o
  \begin{align*}
	\devp{\vec{e}_\rho}{\theta} &= \vec{e}_\theta, \\
	\devp{\vec{e}_\theta}{\theta} &= - \vec{e}\rho.
  \end{align*}
  \begin{solution}
	
  \end{solution}

  \question Sejam as coordenadas esferoidais achatadas $(\xi, \eta, \phi)$ dadas por
  \begin{align*}
	x &= a \cosh \xi \cos \eta \cos \phi, \\
	y &= a \cosh \xi \cos \eta \sin \phi, \\
	z &= a \sinh \eta \sin \eta,
  \end{align*}
  onde $\xi \geq 0$, $-\pi/2 \leq \eta \leq \pi/2$, $0 \leq \phi \leq 2\pi$.
  
  Mostre que os fatores de escala s\~{a}o dados por
  \begin{align*}
	h_\xi = h\eta &= a \sqrt{\sinh^2 \xi + \sin^2 \eta}, \\
	h_\phi &= a \cosh \xi \cos \eta.
  \end{align*}
  \begin{solution}
	
  \end{solution}

  \question Sejam $(u, v, z)$ as coordenadas cil\'{i}ndricas parab\'{o}licas, definidas como
  \[
  x = \frac{1}{2} (u^2 - v^2), \ y = uv, \ z = z,
  \]
  com $-\infty < u < +\infty$, $v \geq 0$, $-\infty < z < +\infty$, e $\vec{r}$ o vetor posi\c{c}\~{a}o, $\vec{r} = x \vec{i} + y \vec{j} + z \vec{k}$. Mostre que em coordenadas cil\'{i}ndricas parab\'{o}licas temos
  \[
  \vec{r} = \frac{1}{2} \sqrt{u^2 + v^2} (u \vec{e}_u + v \vec{e}_v) + z \vec{e}_z.
  \]
  Usando essas coordenadas, mostre que $\mbox{div } \vec{r} = 3$.
  \begin{solution}
	
  \end{solution}

  \question Sejam $(u, v, z)$ as coordenadas cil\'{i}ndricas parab\'{o}licas e o campo vetorial
  \[
  \vec{V} = \frac{1}{4} \sqrt{u^2 + v^2} (-v \vec{e}_u + u \vec{e}_v).
  \]
  Mostre que $\mbox{rot } \vec{V} = \vec{e}_z$.
  \begin{solution}
	
  \end{solution}

  \question Sejam as coordenadas $(u, v, z)$ definidas como
  \[
  x = \frac{1}{2} (u^2 + v^2), \ y = uv, \ z = z.
  \]
  Mostre que esse sistema de coordenadas n\~{a}o \'{e} ortogonal.
  \begin{solution}
	
  \end{solution}

  \question Calcule $\devp{\vec{e}_{q_i}}{q_j}$ com $i, j = 1, 2, 3$ quando $q_i$ s\~{a}o as coordenads esf\'{e}ricas e mostre que as únicas derivadas n\~{a}o nulas s\~{a}o
  \begin{align*}
	\devp{\vec{e}_r}{\phi} &= \sin \theta \vec{e}_\phi, \\
	\devp{\vec{e}_\phi}{\phi} &= -\cos \theta \vec{e}_\theta - \sin \theta \vec{e}_r, \\
	\devp{\vec{e}_\theta}{\phi} &= \cos \theta \vec{e}_\phi, \\
	\devp{\vec{e}_\theta}{\theta} &= -\vec{e}_r.
  \end{align*}
  \begin{solution}
	
  \end{solution}

  \question Seja o campo vetorial
  \[
  \vec{V} = \frac{y z}{r(x^2 + y^2} \vec{i} - \frac{x z}{r (x^2 + y^2)} \vec{j},
  \]
  onde $r = \sqrt{x^2 + y^2 + z^2}$.
  \begin{parts}
	\part Usando coordenadas cartesianas, mostre que
	\[
	\nabla \times \vec{V} = \frac{\vec{r}}{r^3}.
	\]
	\begin{solution}
	  
	\end{solution}

	\part Mostre que em termos de coordenadas esf\'{e}ricas
	\[
	\vec{V} = - \frac{\cot \theta}{r} \vec{e}_\phi.
	\]
	\begin{solution}
	  
	\end{solution}

	\part Obtenha o resultado acima para $\nabla \times \vec{V}$ usando coordenadas esf\'{e}ricas.
	\begin{solution}
	  
	\end{solution}
  \end{parts}

  \question Seja $f(r)$ uma fun\c{c}\~{a}o de $r = \sqrt{x^2 + y^2 + z^2}$. Mostre que
  \begin{align*}
	\nabla f(r) &= f'(r) \vec{e}_r, \\
	\nabla \cdot (\vec{e}_r f(r)) &= \frac{2}{r} f(r) + f'(r), \\
	\nabla \times (\vec{e}_r f(r)) &= 0
  \end{align*}
  usando:
  \begin{parts}
	\part coordenadas cartesianas,
	\begin{solution}
	  
	\end{solution}

	\part coordenads esf\'{e}ricas.
	\begin{solution}
	  
	\end{solution}
  \end{parts}

  \question Sejam
  \begin{align*}
	\nabla &= \vec{e}_r \devp{}{r} + \vec{e}_\theta \frac{1}{r} \devp{}{\theta} + \vec{e}_\phi \frac{1}{r \sin \theta} \devp{}{\phi}, \\
	\vec{V} &= V_r \vec{e}_r + V_\theta \vec{e}_\theta + V_\phi \vec{e}_\phi,
  \end{align*}
  e $\vec{V} \cdot \nabla$ o operador dado por
  \[
  \vec{V} \cdot \nabla = V_r \devp{}{r} + \frac{V_\theta}{r} \devp{}{\theta} + \frac{V_\phi}{r \sin \theta} \devp{}{\phi}.
  \]
  Mostre que $(\vec{V} \cdot \nabla) r = \vec{V}$.
  \begin{solution}
	
  \end{solution}
\end{questions}
\end{document}


