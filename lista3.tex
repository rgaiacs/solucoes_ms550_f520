% Filename: lista3.tex
% This code is part of solu\c{c}\~{o}es de m\'{e}todos.
% 
% Description: Lista 03.
% 
% Created: 25.03.12 10:25:19 AM
% Last Change: 25.03.12 10:43:43 AM
% 
% Author: Raniere Gaia C. da Silva, r.gaia.cs@gmail.com
% Organization: UNICAMP
% 
% Copyright (c) 2012, Raniere Gaia C. da Silva. All rights reserved.
% 
% This file is license under the terms of the 
%
\documentclass[a4paper,12pt, leqno, answers]{exam}
\usepackage[top=3cm, bottom=3cm, left=2cm, right=2cm]{geometry}
\usepackage[utf8]{inputenc}
\usepackage[brazil]{babel}
\usepackage{amsmath}
\usepackage{hyperref}

% Customiza\c{c}\~{a}o da classe exam
\firstpageheader{MS550, F520}{Solu\c{c}\~{a}o da Lista 3}{1º semestre de 2012}
\firstpageheadrule
\footer{Dispon\'{i}vel em \\% Filename: repository.tex
% 
% This code is part of 'Solutions for MS550, M\'{e}todos de Matem\'{a}tica Aplicada I, and F520, M\'{e}todos Matem\'{a}ticos da F\'{i}sica I'
% 
% Description: This file keeps the url of the repository.
% 
% Created: 07.03.12 04:00:00 PM
% Last Change: 30.05.12 04:40:25 PM
% 
% Authors:
% - Raniere Silva (2012): initial version
% 
% Copyright (c) 2012 Raniere Silva <r.gaia.cs@gmail.com>
% 
% This work is licensed under the Creative Commons Attribution-ShareAlike 3.0 Unported License. To view a copy of this license, visit http://creativecommons.org/licenses/by-sa/3.0/ or send a letter to Creative Commons, 444 Castro Street, Suite 900, Mountain View, California, 94041, USA.
%
% This work is distributed in the hope that it will be useful, but WITHOUT ANY WARRANTY; without even the implied warranty of MERCHANTABILITY or FITNESS FOR A PARTICULAR PURPOSE.
%
\url{https://github.com/r-gaia-cs/solucoes_listas_metodos}
}{}{Reportar erros para \\\href{mailto:r.gaia.cs@gmail.com}{r.gaia.cs@gmail.com}
}
\footrule 
\pagestyle{foot}
\renewcommand{\solutiontitle}{\noindent\textbf{Solu\c{c}\~{a}o:}\enspace}
\SolutionEmphasis{\itshape}
\unframedsolutions
\pointname{}

% Customiza\c{c}\~{a}o do pacote amsmath
\allowdisplaybreaks[4]

%Novos ambientes
\newenvironment{fwsolution}{\begin{EnvFullwidth}\begin{TheSolution}}{\end{TheSolution}\end{EnvFullwidth}}

% Novos comandos
\newcommand{\devp}[2]{\frac{\partial #1}{\partial #2}}
\newcommand{\grad}{\mbox{grad }}
\newcommand{\diver}{\mbox{div }}
\newcommand{\rot}{\mbox{rot }}

\begin{document}
\thispagestyle{headandfoot}
\begin{questions}
    \question Mostre que a solu\c{c}\~{a}o de
    \[
    y' + z^{-2} y = 0
    \]
    possui uma singularidade essencial em $z = 0$.
    \begin{solution}
        
    \end{solution}

    \question Mostre que nenhuma solu\c{c}\~{a}o n\~{a}o trivial da equa\c{c}\~{a}o
    \[
    z^2 y'' + z y' + y = 0
    \]
    que \'{e} real no semieixo real positivo do plano complexo pode ser real no semieixo real negativo.
    \begin{solution}
        
    \end{solution}

    \question {}
    \begin{parts}
        \part Mostre que, se
        \begin{equation}
            y'' + p(z) y' + q(z) y = 0
            \label{eq:q02:orig}
        \end{equation}
        possui um ponto singular regular em $z = 0$ e $q(0) \neq 0$, ent\~{a}o
        \begin{equation}
            y'' + \left[ p - \left( q' / q \right) \right] y' + \left[ p' - \left( p q' / q \right) + q \right] y = 0
            \label{eq:q02:mod}
        \end{equation}
        possui um ponto singular regular em $z = 0$.
        \begin{solution}
            
        \end{solution}

        \part Mostre que, se $\left\{ y_1, y_2 \right\}$ \'{e} um conjunto fundamental de solu\c{c}\~{o}es de \eqref{eq:q02:orig} ent\~{a}o $\left\{ y_1', y_2' \right\}$ \'{e} um conjunto fundamental de solu\c{c}\~{o}es de \eqref{eq:q02:mod}.
        \begin{solution}
            
        \end{solution}
    \end{parts}

    \question Seja a equa\c{c}\~{a}o
    \[
    y'' + p(z) y' q(z) y = 0
    \]
    e tal que $z = 0$ \'{e} um ponto singular regular. Mostre que uma mudan\c{c}a de vari\'{a}vel dependente $y \rightarrow w$ da forma $y(z) = z^\rho \phi (z) w(z)$, onde $\phi (0) \neq 0$ e $\phi (z)$ regular em torno de $z = 0$, leva o ponto singular regular $z = 0$ com equa\c{c}\~{a}o indicial $I(r) = 0$ nesse ponto com equa\c{c}\~{a}o indicial $I(r + \rho) = 0$.
    \begin{solution}
        
    \end{solution}

    \question Mostre que a única equa\c{c}\~{a}o diferencial linear de segunda ordem que possui apenas dois pontos singulares regulares e localizados em $z = 0$ e $z = \infty$ \'{e} a equa\c{c}\~{a}o diferencial de Euler
    \[
    z^2 y'' + \alpha z y' + \beta y = 0.
    \]
    \begin{solution}
        
    \end{solution}

    \question Mostre que a equa\c{c}\~{a}o de Bessel
    \[
    z^2 y'' + z y' + \left( z^2 - v^2 \right) y = 0
    \]
    possui um ponto singular irregular em $z = \infty$.
    \begin{solution}
        
    \end{solution}

    \question Mostre que, se $f(0) = 0$ mas $f'(0) \neq 0$, a mudan\c{c}a de vari\'{a}vel $z = f(t)$ leva uma equa\c{c}\~{a}o diferencial
    \[
    y''  p(z) y' + q(z) y = 0
    \]
    com ponto singular regular localizado em $z = 0$ em uma equa\c{c}\~{a}o diferencial com um ponto singular regular localizado em $t = 0$ satisfazendo a mesma equa\c{c}\~{a}o indicial.
    \begin{solution}
        
    \end{solution}
\end{questions}
\end{document}
