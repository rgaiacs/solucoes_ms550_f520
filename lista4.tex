% Filename: lista4.tex
% This code is part of 
% 
% Description: lista 4 de m\'{e}todos
% 
% Created: 18.04.12 05:32:59 PM
% Last Change: 14.05.12 02:34:38 PM
% 
% Author: , 
% Organization: 
% 
% Copyright (c) , . All rights reserved.
% 
% This file is license under the terms of 
%
\documentclass[a4paper,12pt, leqno, answers]{exam}
\usepackage[top=3cm, bottom=3cm, left=2cm, right=2cm]{geometry}
\usepackage[utf8]{inputenc}
\usepackage[brazil]{babel}
\usepackage{amsmath}
\usepackage{amsfonts}
\usepackage{hyperref}

% Customiza\c{c}\~{a}o da classe exam
\firstpageheader{MS550, F520}{Solu\c{c}\~{a}o da Lista 4}{1º semestre de 2012}
\firstpageheadrule
\footer{Dispon\'{i}vel em \\% Filename: repository.tex
% 
% This code is part of 'Solutions for MS550, M\'{e}todos de Matem\'{a}tica Aplicada I, and F520, M\'{e}todos Matem\'{a}ticos da F\'{i}sica I'
% 
% Description: This file keeps the url of the repository.
% 
% Created: 07.03.12 04:00:00 PM
% Last Change: 30.05.12 04:40:25 PM
% 
% Authors:
% - Raniere Silva (2012): initial version
% 
% Copyright (c) 2012 Raniere Silva <r.gaia.cs@gmail.com>
% 
% This work is licensed under the Creative Commons Attribution-ShareAlike 3.0 Unported License. To view a copy of this license, visit http://creativecommons.org/licenses/by-sa/3.0/ or send a letter to Creative Commons, 444 Castro Street, Suite 900, Mountain View, California, 94041, USA.
%
% This work is distributed in the hope that it will be useful, but WITHOUT ANY WARRANTY; without even the implied warranty of MERCHANTABILITY or FITNESS FOR A PARTICULAR PURPOSE.
%
\url{https://github.com/r-gaia-cs/solucoes_listas_metodos}
}{}{Reportar erros para \\\href{mailto:r.gaia.cs@gmail.com}{r.gaia.cs@gmail.com}
}
\footrule 
\pagestyle{foot}
\renewcommand{\solutiontitle}{\noindent\textbf{Solu\c{c}\~{a}o:}\enspace}
\SolutionEmphasis{\itshape}
\unframedsolutions
\pointname{}

% Customiza\c{c}\~{a}o do pacote amsmath
\allowdisplaybreaks[4]

%Novos ambientes
\newenvironment{fwsolution}{\begin{EnvFullwidth}\begin{TheSolution}}{\end{TheSolution}\end{EnvFullwidth}}

% Novos comandos
\newcommand{\devp}[2]{\frac{\partial #1}{\partial #2}}
\newcommand{\grad}{\mbox{grad }}
\newcommand{\diver}{\mbox{div }}
\newcommand{\rot}{\mbox{rot }}

\begin{document}
\thispagestyle{headandfoot}
\begin{questions}
    \question Mostre que $\Gamma(z) = \int_0^1 \left[ \ln(1/t) \right]^{z - 1} dt$.
    \begin{solution}
        Fazendo $t = e^{-y}$ temos
        \begin{align*}
            \exp (y) &= 1 / t \\
            y &= \ln \left( 1 / t \right) \\
            dy &= \frac{\left( - 1 / t^2 \right)}{1 / t} dt = \frac{-1}{t} dt.
        \end{align*}

        Avaliando os extremos:
        \begin{align*}
            y \to 0, t \to 1, \\
            y \to \infty, t \to 0.
        \end{align*}
        Ent\~{a}o
        \begin{align*}
            \Gamma(z) &= \int_1^0 t \left[ \ln \left( 1 / t \right) \right]^{z - 1} \left( -1 / t \right) dt \\
            &= \int_0^1 \left[ \ln \left( 1 / t \right) \right]^{z - 1} dt.
        \end{align*}
    \end{solution}

    \question Use a fun\c{c}\~{a}o gama para mostrar os seguintes resultados:
    \begin{parts}
        \part $\int_0^\infty \exp \left( - x^4 \right) dx = \Gamma(5/5)$.
        \begin{solution}
            Fazendo $x^4 = t$ ent\~{a}o $x = t^{1 / 4}$. E portanto $dt = 4 x^3 dx = 4 t^{3 / 4} dx$ implica que $dx = \left( 1 / 4 \right) t^{-3 / 4} dt$. Logo
            \begin{align*}
                \int_0^\infty \exp (-x^4) dx &= \int_0^\infty \exp (-t ) \frac{1}{4} t^{-3 / 4} dt \\
                &= \frac{1}{4} \int_0^\infty \exp (-t) t^{-3 / 4} dt \\
                &= \frac{1}{4} \Gamma \left( 1 / 4 \right).
            \end{align*}

            Como $\Gamma \left( z + 1 \right) = z \Gamma \left( z \right)$, $\left( 1 / 4 \right) \Gamma \left( 1 / 4 \right) = \Gamma \left( 5 / 4 \right)$ e portanto
            \begin{align*}
                \int_0^\infty exp (-x^4) dx = \Gamma \left( 5 / 4 \right).
            \end{align*}
        \end{solution}

        \part $\int_0^\infty x^{2s + 1} \exp \left( -a x^2 \right) dx = \Gamma(s + 1_ / \left( 2 a^{s + 1} \right)$.
        \begin{solution}
            Fazendo $t = a x^2$, temos $dt = 2 a x dx$. E avaliando os extremos:
            \begin{align*}
                x \to 0, t \to 0, \\
                x \to \infty, t \to \infty.
            \end{align*}
            
            Logo,
            \begin{align*}
                \int_0^\infty \left( x^2 \right)^s \exp \left( -a x^2 \right) x dx &= \int_0^\infty \left( t / a \right)^s \exp \left( -t \right) \frac{dt}{2a} \\
                &= \frac{1}{2 a^{s + 1}} \int_0^\infty \exp \left( -t \right) t^s \\
                &= \frac{1}{2 a^{s + 1}} \Gamma \left( s + 1 \right).
            \end{align*}
        \end{solution}

        \part $\int_0^1 x^k \ln x dx = -1 / \left( k + 1 \right)^2$.
        \begin{solution}
            Fazendo $x = \exp \left( -w \right)$, temos $w = \ln \left( 1 / x \right) = - \ln \left( x \right)$ e consequentemente $dw = \frac{-1}{x} dx = - \exp \left( w \right) dw$ que implica em $dx = - \exp \left( -w \right) dw$. Avaliando os extremos,
            \begin{align*}
                x \to 0, w \to \infty, \\
                x \to 1, w \to 0.
            \end{align*}

            Logo,
            \begin{align*}
                \int_\infty^0 \exp \left( -w k \right) \left( -w \right) \left( -\exp \left( -w \right) \right) dw &= - \int_0^\infty \exp \left( -w \left( k + 1 \right) \right) w dw.
            \end{align*}
            Fazendo $w \left( k + 1 \right) = t$ temos $w = t / \left( k + 1 \right)$ e consequentemente $dt = \left( k + 1 \right) dw \to dw = dt / \left( k + 1 \right)$. Avaliando os extremos
            \begin{align*}
                w \to 0, t \to 0, \\
                w \to \infty, t \to \infty.
            \end{align*}

            Logo,
            \begin{align*}
                - \int_0^\infty \exp \left( -t \right) \left( t / \left( k + 1 \right) \right) \left( 1 / \left( k + 1 \right) \right) dt &= \left( -1 / \left( k + 1 \right) \right) \int_0^\infty \exp \left( -t \right) t dt \\
                &= \frac{-1}{\left( k + 1 \right)^2} \Gamma \left( 2 \right) \\
                &= \frac{-1}{\left( k + 1 \right)^2} \Gamma \left( 1 + 1 \right) \\
                &= - \Gamma \left( 1 \right) / \left( k + 1 \right)^2 \\
                &= - 1 / \left( k + 1 \right)^2.
            \end{align*}
        \end{solution}
    \end{parts}

    \question Mostre que $\lim_{x \to 0} \Gamma(a x) / \Gamma(x) = a^{-1}$.
    \begin{solution}
        Usando a defini\c{c}\~{a}o de Weierstrass, tem-se
        \begin{align*}
            \frac{\Gamma(a x)}{\Gamma(x)} &= \frac{x \exp(\gamma x) \prod_{n = 1}^\infty \left( 1 + x / n \right) \exp(-a x / n)}{a x \exp(\gamma a x) \prod_{n = 1}^\infty \left( 1 + a x / n \right) \exp(-a x / n)} \\
            &= \frac{\exp(\gamma x) \prod_{n = 1}^\infty \left( 1 + x / n \right) \exp(-x / n)}{\left( \exp(\gamma x) \right)^a a \prod \left( 1 + a x / n \right) \exp(-a x / n}, \\
            \lim_{x \to 0} \frac{\Gamma(a x)}{\Gamma(x)} &= \lim_{x \to 0} \left( \frac{\exp(\gamma x (1 - a)}{a} \right) \lim_{x \to 0} \left( \frac{\prod_{n = 1}^\infty (1 + x / n)v \exp(- x / n)}{\prod_{n = 1}^\infty (1 + a x q n) \exp(-x / n)} \right) \\
            &= \left( 1 / a \right).
        \end{align*} \\
    \end{solution}

    \question Mostre que $\text{Res }_{z = -n} \Gamma(z) = \left( -1 \right)^n / n!, \quad n = 0, 1, 2, \ldots$
    \begin{solution}
        Temos que
        \begin{align*}
            \text{Res}_{z = -n} \Gamma(z) &= \lim_{z \to - n} \left( z - (-n) \right) \Gamma(z) \\
            &= \lim_{z \to -n} \left( z + n \right) \lim_{k \to \infty} \left( k! k^z \right) / \left( z (z + 1) \ldots \right) \\
            &= \lim_{k \to \infty} \lim_{z \to -n} (z + n) k! k^z / \left( z (z + 1) \ldots (z + n - 1) (z + n) (z + n + 1) \ldots \right) \\
            &= \lim_{k \to \infty} k! / \left( k^n (-1)^n (n) (n - 1) \ldots (1) (1) \ldots (k - n) \right) \\
            &= \left( (-1)^ / n! \right) \lim_{k \to \infty} \left( k (k - 1) \ldots (k - n) \ldots (1) \right) / \left( k^n (k - n) \ldots (1) \right) \\
            &= \left( (-1)^n / n! \right) \lim_{k \to \infty} \left( 1 - 1 /k \right) \ldots \left( 1 - (n - 1) / k \right) \\
            &= (-1)^n / n!
        \end{align*}
    \end{solution}

    \question Mostre que $| \left( i x \right)! |^2 = \left( \pi x \right) / \left( \sinh \pi x \right)$ onde $x! = \Gamma(x + 1)$.
    \begin{solution}
        Como $x! = \Gamma(x + 1) = x \Gamma(x)$, tem-se
        \begin{align*}
            (i x)! &= \Gamma(ix + 1) \\
            &= ix \Gamma(i x), \\
            | (i x)! |^2 &= \left( i x \Gamma(i x) \right) \left( -i x \Gamma(-i x) \right) \\
            &= i x \Gamma(i x) \Gamma(-i x + 1) \\
            &= (i x) \left( 2 \pi / \left( i ( \exp(-i \pi i x) - \exp(i \pi i x) \right) \right) && \text{pela f\'{o}rmula de reflex\~{a}o} \\
            &=\left( 2 \pi x \right) / \left( \exp(\pi x) - \exp(-\pi x) \right) \\
            &= \left( 2 \pi x \right) / \left( 2 \sinh(\pi x) \right).
        \end{align*}

        A f\'{o}rmula de reflex\~{a}o diz que
        \begin{align*}
            \Gamma(z) \Gamma(1 -z) &= \pi / \sin(\pi z) = 2 \pi / \left( i (\exp(-i \pi z) - \exp(i \pi z)) \right).
        \end{align*}
    \end{solution}

    \question Mostre que $|x!| \geq |\left( x + iy \right)!|$, onde $x, y \in \mathbb{R}$, $x \neq -1, -2, \ldots$.
    \begin{solution}
        Temos que $x! = \Gamma(x + 1) = x \Gamma(x)$, portanto $| x! | = x \Gamma(x) = | \Gamma(x + 1) |$. E
        \begin{align*}
            (x + i y)! &= \Gamma(x + i y + 1), \\
            | (x + i y)! | &= | \Gamma(x + i y + 1) | \\
            &= \left| \int_0^\infty \exp(-t) t^{x + iy} dt \right| \\
            &\leq \int_0^\infty | \exp(-t) t^{x + i y} | dy \\
            &\leq \int_0^\infty | \exp(-t) t^x \exp(i y \ln t) | dt \\
            &\leq \int_0^\infty | \exp(-t) t^x | | \exp(i y \ln t) | dt \\
            &\leq \int_0^\infty | \exp(-t) t^x dt && | \exp(i y \ln t) | = 1 \\
            &= x! \leq | x! |.
        \end{align*}
        Como $| \int_0^\infty \exp(-t) t^x dt | = | \Gamma(x + 1) | = | x! |$, tem-se que $ | (x + i y)! | \leq | x! |$.
    \end{solution}

    \question Seja $f(z) = \left( 1 + z \right)^\alpha$.
    \begin{parts}
        \part Mostre que
        \begin{align*}
            \left. \frac{d^n f}{d z^n} \right|_{z = 0} &= \frac{\Gamma(\alpha + 1)}{\Gamma(\alpha - n + 1}
        \end{align*}
        e use esse resultado para mostrar que $\left( 1 + z \right)^\alpha = \sum_{n= 0}^\infty \binom{\alpha}{n} z^n$, onde $\binom{\alpha}{n} = \alpha! / \left( n! \left( \alpha - n \right)! \right) = \Gamma(\alpha + 1) / \left( n! \Gamma(\alpha - n + 1) \right)$.
        \begin{solution}
            Temos que
            \begin{align*}
                \left[ \frac{d^n f}{d z^n} \right]_{z = 0} &= \left[ \frac{d^n \left[ (1 + z)^\alpha \right]}{d z^n} \right]_{z = 0} \\
                &= \left[ \alpha (\alpha - 1) \ldots (\alpha - n + 1) (1 + z)^{\alpha - n} \right]_{z = 0} \\
                &= \alpha (\alpha - 1) \ldots (\alpha - n + 1) && z = 0.
            \end{align*}
            Como $\Gamma(\alpha + 1) = \alpha \Gamma(\alpha)$ ent\~{a}o $\alpha = \Gamma(\alpha + 1) / \Gamma(\alpha)$ e portanto
            \begin{align*}
                \left[ \frac{d^n f}{d z^n} \right]_{z = 0} &= \frac{\Gamma(\alpha + 1)}{\Gamma(\alpha)} \frac{\Gamma(\alpha)}{\Gamma(\alpha - 1)} \ldots \frac{\Gamma(\alpha - n - 1}{\Gamma(\alpha - n)} \frac{\Gamma(\alpha - n)}{\Gamma(\alpha - n + 1} \\
                &= \Gamma(\alpha + 1) / \Gamma(\alpha - n + 1).
            \end{align*}
            Expandindo $(1 + z)^\alpha$ em s\'{e}rie de Taylor (em torno de $z = 0$):
            \begin{align*}
                \left( 1 + z \right)^\alpha &= \sum_{n = 0}^\infty (1 / n!) z^n \left[ \frac{d^n (1 + z)^\alpha}{d z^n} \right]_{z = 0}\\
                &= \sum_{n = 0}^\infty \frac{\Gamma(\alpha + 1)}{\Gamma(\alpha - n + 1)} \frac{z^n}{n!} \\
                &= \sum_{n = 0}^\infty (\alpha / n) z^n,
            \end{align*}
            onde $(\alpha / n) = \Gamma(\alpha + 1) / \left( n! \Gamma(\alpha - n + 1 \right)$.
        \end{solution}

        \part Generalize esse resultado para $a, b \in \mathbb{C}$ mostrando que $\left( a + b \right)^\alpha = \sum_{n = 0}^\infty \binom{\alpha}{n} a^n b^{\alpha - n}$.
        \begin{solution}
            Para $a, b \in \mathbb{C}$ temos
            \begin{align*}
                \left( a + b \right)^\alpha &= \left[ (1 + a / b) b \right]^\alpha.
            \end{align*}
            Fazendo $z = a / b$,
            \begin{align*}
                (a + b)^\alpha = (1 + z)^\alpha b^\alpha \\
                &= \sum_{n = 0}^\infty \binom{\alpha}{n} z^n b^\alpha \\
                &= \sum_{n = 0}^\infty \binom{\alpha}{n} (a^n / b^n) b^\alpha \\
                &= \sum_{n = 0}^\infty \binom{\alpha}{n} a^n b^{\alpha - b}.
            \end{align*}
        \end{solution}
        
        \part Mostre que se $\alpha = m \in \mathbb{Z}$ ent\~{a}o essa s\'{e}rie \'{e} truncada no termo $n = m$.
        \begin{solution}
            Para $\alpha = m \in \mathbb{Z}$, no termo $n = m$, tem-se
            \begin{align*}
                \binom{m}{m} b^m \alpha^{m - m} = b^m.
            \end{align*}
            Analisando o termo seguinte da s\'{e}rie, ou seja, $n = m + 1$, com $\alpha = m$, tem-se
            \begin{align*}
                \binom{\alpha}{n} &= \binom{m}{m + 1} \\
                &= \Gamma(m + 1) / \left( (m + 1)! \Gamma(m - (m + 1) + 1 \right) \\
                &= \Gamma(m + 1) / \Gamma(0).
            \end{align*}
            Como $\Gamma(0)$ diverge, pois $0$ \'{e} um dos p\'{o}los temos que a s\'{e}rie \'{e} truncada em $n = m$.
        \end{solution}
    \end{parts}

    \question Mostre que $\int_{-1}^n \left( 1 + x \right)^a \left( 1 - x \right)^b dx = 2^{a + b + 1} B(a + 1, b + 1)$, onde $B(a, b)$ \'{e} a fun\c{c}\~{a}o beta.
    \begin{solution}
        Temos que a fun\c{c}\~{a}o beta corresponde a
        \begin{align*}
            B(a, b) &= \int_0^1 t^{a - 1} (1 - t)^{b - 1} dt.
        \end{align*}
        Fazendo $t = (1 + x)/2$ temos que $x = 2t - 1$ e $dt = dx/2$. E os extremos ficam $t = 0 \to x = -1$ e $ = 1 \to x = 1$. Substituindo na dfini\c{c}\~{a}o da fun\c{c}\~{a}o beta
        \begin{align*}
            B(a, b) &= \int_{-1}^1 \left( \frac{1 + x}{2} \right)^{a - 1} \left( \frac{1 - x}{2} \right)^{b - 1} \frac{dx}{2} \\
            &= \int_{-1}^1 \frac{(1 + x)^a}{2^2} \frac{(1 - x)^b}{2^b} \frac{dx}{2} \\
            &= \frac{1}{2^{a + b + 1}} \int_{-1}^1 (1 + x)^a (1 - x)^b dx.
        \end{align*}
        Portanto,
        \begin{align*}
            \int_{-1}^1 (1 + x)^a (1 - x)^b dx &= 2^{a + b + 1} B(a + 1, b + 1).
        \end{align*}
    \end{solution}

    \question Mostre que
    \begin{parts}
        \part $B(a, b) = B(a + 1, b) + B(a, b + 1)$.
        \begin{solution}
            Temos que
            \begin{align*}
                B(a + 1, b) + B(a, b + 1) &= \frac{\Gamma(a + 1) \Gamma(b)}{\Gamma(a + b + 1)} + \frac{\Gamma(a) \Gamma(b + 1)}{\Gamma(a + b + 1)} \\
                &= \frac{a \Gamma(a) \Gamma(b)}{(a + b) \Gamma(a + b)} + \frac{\Gamma(a) b \Gamma(b)}{(a + b) \Gamma(a + b)} \\
                &= \frac{(a + b) \Gamma(a) \Gamma(b)}{(a + b) \Gamma(a + b)} \\
                &= \frac{\Gamma(a) \Gamma(b)}{\Gamma(a + b)} \\
                &= B(a, b).
            \end{align*}
        \end{solution}

        \part $B(a, b) = \left( b - 1 \right)/a B(a + 1, b - 1)$.
        \begin{solution}
            Temos que
            \begin{align*}
                \frac{b - 1}{a} B(a + 1, b - 1) &= \left( \frac{b - 1}{a} \right) \frac{\Gamma(a + 1) \Gamma(b - 1)}{\Gamma(a + 1 + b - 1)} \\
                &= \left( \frac{b - 1}{a} \right) \frac{\Gamma(a + 1) \Gamma(b - 1)}{\Gamma(a + b)} \\
                &= \frac{(b - 1)\Gamma(a + 1) \Gamma(b - 1)}{a \Gamma(a + b)} \\
                &= \frac{(b - 1) \Gamma(b - 1) a \Gamma(a)}{a \Gamma(a + b)} \\
                &= \frac{(b - 1) \Gamma(b - 1) \Gamma(a)}{\Gamma(a + b)} \\
                &= \frac{\Gamma(b - 1 + 1) \Gamma(a)}{\Gamma(a + b)} \\
                &= \frac{\Gamma(b) \Gamma(a)}{\Gamma(a + b)} \\
                &= B(a, b).
            \end{align*}
        \end{solution}
    \end{parts}

    \question Mostre que
    \begin{align*}
        \int_{-1}^1 \left( 1 - x^2 \right)^{1/2} x^{2n} dx = \begin{cases}
            \pi / 2, & n = 0; \\
            \pi \left( 2n - 1 \right)!! / \left( 2n + 2 \right)!!, & n = 1, 2, \ldots
        \end{cases}
    \end{align*}
    \begin{solution}
        Temos que
        \begin{align*}
            \int_{-1}^1 (1 - x^2)^{1/2} x^{2n} dx &= 2 \int_0^1 (1 - x^2)^{1/2} x^2 dx.
        \end{align*}
        Fazendo $x^2 = t$, $dx = dt/2x = dt/2t^{1/2}$. Os extremos ficam com $x = 0 \to t = 0$ e $x = 1 \to t  = 1$. Logo,
        \begin{align*}
            2 \int_0^1 (1 - t)^{1/2} t^b t^{1/2} /2 dt = \int_0^1 (1 - t)^{1/2} t^{n - 1/2} dt \\
            &= B(3/2, n + 1/2) \\
            &= B(n + 1/2, 3/2).
        \end{align*}
        TODO
    \end{solution}

    \question Mostre que $\int_t^z dx \ \left( \left( z - x \right)^{1 - \alpha} \left( x - t \right)^{\alpha} \right) = \pi / \left( \sin \pi \alpha \right)$.
    \begin{solution}
        
    \end{solution}

    \question Mostre que $d \left( z \right)_n / d z = \left( z \right)_n \left[ \psi_0 \left( z + n \right) - \psi_0\left( z \right) \right]$, onde $\left( z \right)_n = z \left( z + 1 \right) \cdots \left( z + n - 1 \right)$ \'{e} o s\'{i}mbolo de Pochammer e $\psi_0$ \'{e} a fun\c{c}\~{a}o digama.
    \begin{solution}
        
    \end{solution}

    \question Mostre que $\ln \sin \pi z = \ln \pi z - \sum_{n = 1}^\infty \zeta(2n) z^{2n} / n$, onde $\zeta(n)$ \'{e} a fun\c{c}\~{a}o zeta de Riemann.
    \begin{solution}
        
    \end{solution}

    \question Mostre que $\psi_n(1) = \left( -1 \right)^{n + 1} n! \zeta(n + 1)$, onde $\psi_n(z)$ \'{e} a fun\c{c}\~{a}o poligama (de ordem n).
    \begin{solution}
        A fun\c{c}\~{a}o poligama correspondente a
        \begin{align*}
            \psi_n(z + 1) = (-1)^{n + 1} n! \sum_{k = 1}^\infty 1/(z + k)^{n + 1}, n = 1, 2, \ldots
        \end{align*}
        Para $z = 0$ temos
        \begin{align*}
            \psi_n(1) = (-1)^{n + 1}n ! \sum_{k = 1}^\infty 1/k^{n + 1}, n = 1, 2, \ldots
        \end{align*}
        Como $\sum_{k = 1}^\infty 1/k^{n + 1} = \zeta(n + 1)$ temos $\psi_n(1) = (-1)^{k + 1} n! \zeta(n + 1)$.
    \end{solution}

    \question Mostre que $\psi_0(z) = -\gamma + \int_0^\infty dt / \left( e^t - 1 \right) - \int_0^\infty \left( e^{\left( 1 - z \right) t} \right) dt / \left( e^t - 1 \right)$.
    \begin{solution}
        
    \end{solution}

    \question Mostre que $n! \zeta(n + 1) = \int_0^\infty t^n dt / \left( e^t - 1 \right)$.
    \begin{solution}
        
    \end{solution}

    \question[E de 2011] Mostre que
    \begin{align*}
        \int_{-1}^1 \left( \frac{1 + x}{1 - x} \right)^{1/4} dx &= \frac{\pi}{\sqrt{2}}.
    \end{align*}
    \begin{solution}
        Temos que
        \begin{align*}
            \int_{-1}^1 (1 + x)^{1/4} (1 - x)^{-1/4} dx &= \int_0^1 2^{1/4} t^{1/4} (1 - t)^{-1/4} 2 dt && t = (1 + x)/2 \\
            &= 2 \int_0^1 t^{5/4 - 1} (1 - t)^{3/4 - 1} dt \\
            &= 2 B(5/4, 3/4) \\
            &= \frac{2 \Gamma(5/4) \Gamma(3/4)}{\Gamma(2)} \\
            &= \frac{2(1/4) \Gamma(1/4) \Gamma(1 - 1/4)}{1 \Gamma(1)} \\
            &= \frac{1}{2} \frac{\pi}{\sin(\pi/4)} \\
            &= \pi / \sqrt{2}.
        \end{align*}
    \end{solution}
\end{questions}
\end{document}
