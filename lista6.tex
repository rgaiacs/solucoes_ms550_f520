% Filename: lista6.tex
% This code is part of 'Solutions for MS550, M\'{e}todos de Matem\'{a}tica Aplicada I, and F520, M\'{e}todos Matem\'{a}ticos da F\'{i}sica I'
% 
% Description: This file corresponds to the solutions of homework sheet 6.
% 
% Created: 25.03.12 10:25:19 AM
% Last Change: 01.06.12 06:37:49 PM
% 
% Authors:
% - Raniere Silva (2012): initial version
% 
% Copyright (c) 2012 Raniere Silva <r.gaia.cs@gmail.com>
% 
% Permission is granted to copy, distribute and/or modify this document under the terms of the GNU Free Documentation License, Version 1.3 or any later version published by the Free Software Foundation; with no Invariant Sections, no Front-Cover Texts, and no Back-Cover Texts.
% This document is distributed in the hope that it will be useful, but WITHOUT ANY WARRANTY; without even the implied warranty of MERCHANTABILITY or FITNESS FOR A PARTICULAR PURPOSE.
% More details at <http://www.gnu.org/licenses/>
%
\documentclass[a4paper,12pt, leqno, answers]{exam}
% Customiza\c{c}\~{a}o da classe exam
\newcommand{\mycheader}{Lista 6 - Fun\c{c}\~{a}o de Bessel}
\header{MS550, F520}{\mycheader}{\thepage/\numpages}
\headrule
\footer{Dispon\'{i}vel em \\% Filename: repository.tex
% 
% This code is part of 'Solutions for MS550, M\'{e}todos de Matem\'{a}tica Aplicada I, and F520, M\'{e}todos Matem\'{a}ticos da F\'{i}sica I'
% 
% Description: This file keeps the url of the repository.
% 
% Created: 07.03.12 04:00:00 PM
% Last Change: 30.05.12 04:40:25 PM
% 
% Authors:
% - Raniere Silva (2012): initial version
% 
% Copyright (c) 2012 Raniere Silva <r.gaia.cs@gmail.com>
% 
% This work is licensed under the Creative Commons Attribution-ShareAlike 3.0 Unported License. To view a copy of this license, visit http://creativecommons.org/licenses/by-sa/3.0/ or send a letter to Creative Commons, 444 Castro Street, Suite 900, Mountain View, California, 94041, USA.
%
% This work is distributed in the hope that it will be useful, but WITHOUT ANY WARRANTY; without even the implied warranty of MERCHANTABILITY or FITNESS FOR A PARTICULAR PURPOSE.
%
\url{https://github.com/r-gaia-cs/solucoes_listas_metodos}
}{}{Reportar erros para \\\href{mailto:r.gaia.cs@gmail.com}{r.gaia.cs@gmail.com}
}
\footrule 
\pagestyle{headandfoot}
\renewcommand{\solutiontitle}{\noindent\textbf{Solu\c{c}\~{a}o:}\enspace}
\SolutionEmphasis{\slshape}
\unframedsolutions
\pointname{}

% Para impress\~{a}o
\usepackage[top=3cm, bottom=3cm, left=2cm, right=2cm]{geometry}

% Para ereaders (Kindle, Nook, Kobo, ...) and tablets (iPad, GalaxyTab, ...)
% \usepackage[papersize={180mm,240mm},margin=2mm]{geometry}
% \sloppy

% Pacotes
\usepackage[utf8]{inputenc}
\usepackage[brazil]{babel}
\usepackage{amsmath}
\usepackage{amsfonts}
\usepackage{amssymb}
\usepackage{hyperref}
\usepackage{graphicx}

% Customiza\c{c}\~{a}o do pacote amsmath
\allowdisplaybreaks[4]

% Novos ambientes
% \newenvironment{fwsolution}{\begin{EnvFullwidth}\begin{TheSolution}}{\end{TheSolution}\end{EnvFullwidth}}

% Novos comandos
\newcommand{\devp}[2]{\frac{\partial #1}{\partial #2}}
\newcommand{\grad}{\mbox{grad }}
\newcommand{\diver}{\mbox{div }}
\newcommand{\rot}{\mbox{rot }}

\begin{document}
%cover
\thispagestyle{empty}
% Filename: cover.tex
% This code is part of 'Solutions for MS550, M\'{e}todos de Matem\'{a}tica Aplicada I, and F520, M\'{e}todos Matem\'{a}ticos da F\'{i}sica I'
% 
% Description: This file corresponds to the cover.
% 
% Created: 30.05.12 04:40:25 PM
% Last Change: 31.05.12 10:11:55 PM
% 
% Authors:
% - Raniere Silva (2012): initial version
% 
% Copyright (c) 2012 Raniere Silva <r.gaia.cs@gmail.com>
% 
% Permission is granted to copy, distribute and/or modify this document under the terms of the GNU Free Documentation License, Version 1.3 or any later version published by the Free Software Foundation; with no Invariant Sections, no Front-Cover Texts, and no Back-Cover Texts.
% This document is distributed in the hope that it will be useful, but WITHOUT ANY WARRANTY; without even the implied warranty of MERCHANTABILITY or FITNESS FOR A PARTICULAR PURPOSE.
% More details at <http://www.gnu.org/licenses/>
%
\begin{center}
    \LARGE{Solu\c{c}\~{o}es para MS550, M\'{e}todos de Matem\'{a}tica Aplicada I, e F520, M\'{e}todos Matem\'{a}ticos da F\'{i}sica I}
    
    \Large{\mycheader}
\end{center}
\vspace{.5\textheight}

\begin{tabular}{|p{.9\textwidth}|}
\hline
\'{E} garantida a permiss\~{a}o para copiar, distribuir e/ou modificar este documento sob os termos da Licen\c{c}a de Documenta\c{c}\~{a}o Livre GNU (GNU Free Documentation License), Vers\~{a}o 1.2 ou qualquer vers\~{a}o posterior publicada pela Free Software Foundation; sem Se\c{c}\~{o}es Invariantes, Textos de Capa Frontal, e sem Textos de Quarta Capa.

Este documento \'{e} distribuido na esperança que possa ser \'{u}til, mas SEM NENHUMA GARANTIA; sem uma garantia implicita de ADEQUA\c{C}\~{A}O a qualquer MERCADO ou APLICA\c{C}\~{A}O EM PARTICULAR.

Mais detalhes em \url{http://www.gnu.org/licenses/}.
\\ \hline
\end{tabular}

\newpage
\setcounter{page}{1}
Equa\c{c}\~{o}es eventualmente útil:
\begin{align}
    & f(x) = \sum_{n = 0}^\infty \frac{f^{(n)}(a)}{n!} (x - a)^n \tag{ST} \label{eq:ser_taylor} \\
    & \Gamma(z) = \int_0^\infty e^{-t} t^{z - 1} dt \tag{GE} \label{eq:gamma_euler} \\
    & \Gamma(z + 1) = z \Gamma(z), \ \Gamma(z) \Gamma(1 - z) = \pi / \sin(\pi z) \label{eq:gamma_rel} \\
    & 2^{2 z - 1} \Gamma(z) \Gamma(z + 1/2) = \sqrt{\pi} \Gamma(2 z) \label{eq:gamma_dup_legendre} \\
    & B(z, w) = \frac{\Gamma(z) \Gamma(w)}{\Gamma(z + w)} \tag{BG} \label{eq:beta} \\
    & B(z, w) = 2 \int_0^{\pi / 2} \cos^{2z - 1} \theta \sin^{2w - 1} \theta d\theta \tag{BT} \label{eq:beta_trig} \\
    & B(z, w) = \int_0^1 t^{z - 1} (1 - t)^{w - 1} dt \tag{BI} \label{eq:beta_int} \\
    & (\alpha)_n = \alpha (\alpha + 1) \ldots (\alpha + n - 1), \ (\alpha)_0 = 1 \tag{SP} \label{eq:sim_poch} \\
    & (\alpha)_n = \frac{\Gamma(\alpha + n)}{\Gamma(\alpha)} \label{eq:sim_poch_gamma} \\
    & \frac{(\alpha)_n}{m!} = \binom{\alpha + n - 1}{n}, \ \frac{(-\alpha)_n}{n!} = (-1)^n \binom{\alpha}{n} \label{eq:sim_poch_binom} \\
    & x^2 y'' + x y' + (x^2 - \nu^2) y = 0 \tag{FB} \label{eq:bessel} \\
    & J_\nu(x) = \sum_{k = 0}^\infty \frac{(-1)^k}{k! \Gamma(k + \nu + 1)} \left( \frac{x}{2} \right)^{2k + \nu} \label{eq:bessel_pri_esp} \\
    & J_\nu(x) = \frac{x^\nu \exp(-i x)}{2^\nu \Gamma(\nu + 1)} \,_1F_1(\nu + 1/2, 2\nu + 1; 2 i x) \label{eq:bessel_pri_esp_hiperg_con} \\
    & J_{-\nu}(x) = (-1)^\nu J_\nu(x) \label{eq:bessel_prim_esp_neg} \\
    & \exp(x (t - t^{-1}) / 2) = \sum_{k = -\infty}^{+\infty} t^k J_k(x) \tag{GFB} \label{eq:bessel_pri_esp_geratriz} \\
    & x^2 y'' + x y' - (x^2 + v^2) y = 0 \tag{FBM} \label{eq:bessel_mod} \\
    & I_\nu(x) = i^{-\nu} J_\nu(x) \label{eq:bessel_mod_pri_esp}
    % & z(1 - z)y'' + \left[ \gamma - (\alpha + \beta + 1) z \right] y' - \alpha \beta y = 0 \tag{EH} \label{eq:hiperg} \\
    % & {}_2F_1(\alpha, \beta, \gamma; z) = \sum_{n = 0}^\infty \frac{(\alpha)_n (\beta)_n}{(\gamma)_n} \frac{z^n}{n!} \tag{SH} \label{eq:ser_hiperg} \\
    % & {}_2F_1(\alpha, \beta, \gamma; z) = \frac{1}{B(\beta, \gamma - \beta)} \int_0^1 t^{\beta - 1} (1- t)^{\gamma - \beta - 1} (1 - tz)^{-\alpha} dt \label{eq:hiperg_int} \\
    % & {}_2F_1(\alpha, \beta, \gamma; z) = \frac{\alpha \beta}{\gamma} \,{}_2F_1(\alpha + 1, \beta + 1, \gamma + 1; z) \label{eq:hiperg_der} \\
    % & zy'' + (\gamma - z)y' - \alpha y = 0 \tag{EHC} \label{eq:hiper_con} \\
    % & {}_1F_1(\alpha, \gamma; z) = \sum_{n = 0}^\infty \frac{(\alpha)_n}{(\gamma)_n} \frac{z^n}{n!} \tag{SHC} \label{eq:ser_hiperg_con}
\end{align}
\thispagestyle{headandfoot}
\begin{questions}
    \question Mostre, diretamente a partir da defini\c{c}\~{a}o, que
    \begin{parts}
        \part[Equa\c{c}\~{a}o 18.92 do Riley\nocite{Riley:2006:Mathematical}] $\mathrm{d}[x^\nu J_\nu(x)]/\mathrm{d}x = x^\nu J_{\nu - 1}(x)$
        \begin{solution}
            Temos que
            \begin{align*}
                \frac{\mathrm{d}}{\mathrm{d}x} \left[ x^\nu J_\nu(x) \right] &= \frac{\mathrm{d}}{\mathrm{d}x} x^\nu \sum_{k = 0}^\infty \frac{(-1)^k }{k! \Gamma(k + \nu + 1)} \frac{x^{2k + \nu}}{2^{2k + \nu}} && \text{por \eqref{eq:bessel_pri_esp}} \\
                &= \frac{\mathrm{d}}{\mathrm{d}x} \sum_{k = 0}^\infty x^\nu \frac{(-1)^k x^{2k + \nu}}{k! \Gamma(k + \nu + 1) 2^{2k + \nu}} \\
                &= \frac{\mathrm{d}}{\mathrm{d}x} \sum_{k = 0}^\infty \frac{(-1)^k x^{2 k + 2 \nu}}{k! \Gamma(k + \nu + 1) 2^{2 k + \nu}} \\
                &= \sum_{k = 0}^\infty \frac{(-1)^k (2 k + 2 \nu) x^{2 k + 2 \nu - 1}}{k! \Gamma(k + \nu + 1) 2^{2k + \nu}} \\
                &= \sum_{k = 0}^\infty \frac{(-1)^k (k + \nu) x^{2 k + 2 \nu - 1}}{k! \Gamma(k + \nu + 1) 2^{2k + \nu - 1}} \\
                &= \sum_{n = 0}^\infty \frac{(-1)^k (k + \nu) x^{2 k + 2 \nu - 1}}{k! (k + \nu) \Gamma(k + \nu) 2^{2k + \nu - 1}} \\
                &= \sum_{k = 0}^\infty \frac{(-1)^k x^{2 k + 2 \nu - 1}}{k! \Gamma(k + \nu) 2^{2k + \nu - 1}} \\
                &= x^\nu \sum_{k = 0}^\infty \frac{(-1)^k x^{2k + (\nu - 1)}}{k! \Gamma( k + ( \nu - 1) + 1 ) ) 2^{2k + (\nu - 1)}} \\
                &= x^\nu J_{\nu - 1}(x) && \text{por \eqref{eq:bessel_pri_esp}}.
            \end{align*}
        \end{solution}

        \part $\mathrm{d}[x^{-\nu} J_\nu(x)]/\mathrm{d}x = -x^{-\nu} J_{\nu + 1}(x)$
        \begin{solution}
            Temos que
            \begin{align*}
                \frac{\mathrm{d}}{\mathrm{d}x} \left[ x^\nu J_\nu(x) \right] &= \frac{\mathrm{d}}{\mathrm{d}x} x^{-\nu} \sum_{k = 0}^\infty \frac{(-1)^k }{k! \Gamma(k + \nu + 1)} \frac{x^{2k + \nu}}{2^{2k + \nu}} && \text{por \eqref{eq:bessel_pri_esp}} \\
                &= \frac{\mathrm{d}}{\mathrm{d}x} \sum_{k = 0}^\infty x^{-\nu} \frac{(-1)^k x^{2k + \nu}}{k! \Gamma(k + \nu + 1) 2^{2k + \nu}} \\
                &= \frac{\mathrm{d}}{\mathrm{d}x} \sum_{k = 0}^\infty \frac{(-1)^k x^{2 k}}{k! \Gamma(k + \nu + 1) 2^{2 k + \nu}} \\
                &= \sum_{k = 0}^\infty \frac{(-1)^k (2 k) x^{2 k - 1}}{k! \Gamma(k + \nu + 1) 2^{2k + \nu}} \\
                &= \sum_{k = 0}^\infty \frac{(-1)^k (2 k) x^{-\nu} x^{2 k + \nu - 1}}{k! \Gamma(k + \nu + 1) 2^{2k + \nu}} \\
                &= x^{-\nu} \sum_{k = 0}^\infty \frac{(-1)^k (2 k) x^{2 k + \nu - 1}}{k! \Gamma(k + \nu + 1) 2^{2k + \nu}} \\
                &= x^{-\nu} \sum_{k = 0}^\infty \frac{(-1)^k k x^{2 k + \nu - 1}}{k! \Gamma(k + \nu + 1) 2^{2k + \nu - 1}} \\
                &= x^{-\nu} \sum_{k = 0}^\infty \frac{(-1)^k x^{2 k + \nu - 1}}{(k - 1)! \Gamma(k + \nu + 1) 2^{2k + \nu - 1}} \\
                &= -x^{-\nu} \sum_{k = 0}^\infty \frac{(-1)^{k - 1} x^{2 (k - 1) + \nu + 1}}{(k - 1)! \Gamma( (k - 1) + (\nu + 1) + 1) 2^{2(k - 1) + \nu + 1}} \\
                &= x^{-\nu} J_{\nu + 1}(x) && \text{por \eqref{eq:bessel_pri_esp}}.
            \end{align*}
        \end{solution}

        \part $J_{\nu - 1}(x) + J_{\nu + 1}(x) = (2\nu/x) J_\nu(x)$
        \begin{solution}
            Temos que
            \begin{align*}
                \frac{\mathrm{d}}{\mathrm{d}x} \left[ x^\nu J_\nu(x) \right] &= \nu x^{\nu - 1} J_\nu(x) + x^\nu J'_\nu(x) = x^\nu J_{\nu - 1}(x), \\
                \frac{\mathrm{d}}{\mathrm{d}x} \left[ x^{-\nu} J_\nu(x) \right] &= -\nu x^{-\nu - 1} J_\nu(x) + x^{-\nu} J'_\nu(x) = - x^{-\nu} J_{\nu + 1}(x).
            \end{align*}
            Ent\~{a}o, para $x \neq 0$, temos que
            \begin{align*}
                \nu J_\nu(x) + x J'_\nu(x) &= x J_{\nu - 1}(x), \\
                -\nu J_\nu(x) + x J'_\nu(x) &= - x J_{\nu + 1}(x).
            \end{align*}
            Logo,
            \begin{align*}
                \nu J_\nu(x) + x J'_\nu(x) - \left( -\nu J_\nu(x) + x J'_\nu(x) \right) &= x J_{\nu - 1}(x) - \left( - x J_{\nu + 1}(x) \right) \\
                2 \nu J_\nu(x) &= x J_{\nu - 1}(x) + x J_{\nu + 1}(x) \\
                2 \nu x^{-1} J_\nu(x) &= J_{\nu - 1}(x) + J_{\nu + 1}(x).
            \end{align*}

            Tamb\'{e}m \'{e} poss\'{i}vel utilizar \eqref{eq:bessel_pri_esp_geratriz} para mostrar a igualdade pedida (ver detalhes em Riley\nocite{Riley:2006:Mathematical}).
        \end{solution}

        \part $J_{\nu - 1}(x) - J_{\nu + 1}(x) = 2 J'_\nu(x)$
        \begin{solution}
            Temos que
            \begin{align*}
                \frac{\mathrm{d}}{\mathrm{d}x} \left[ x^\nu J_\nu(x) \right] &= \nu x^{\nu - 1} J_\nu(x) + x^\nu J'_\nu(x) = x^\nu J_{\nu - 1}(x), \\
                \frac{\mathrm{d}}{\mathrm{d}x} \left[ x^{-\nu} J_\nu(x) \right] &= -\nu x^{-\nu - 1} J_\nu(x) + x^{-\nu} J'_\nu(x) = - x^{-\nu} J_{\nu + 1}(x).
            \end{align*}
            Ent\~{a}o, para $x \neq 0$, temos que
            \begin{align*}
                \nu J_\nu(x) + x J'_\nu(x) &= x J_{\nu - 1}(x), \\
                -\nu J_\nu(x) + x J'_\nu(x) &= - x J_{\nu + 1}(x).
            \end{align*}
            Logo,
            \begin{align*}
                \nu J_\nu(x) + x J'_\nu(x) + \left( -\nu J_\nu(x) + x J'_\nu(x) \right) &= x J_{\nu - 1}(x) + \left( - x J_{\nu + 1}(x) \right) \\
                2 x J'_\nu(x) &= x J_{\nu - 1}(x) - x J_{\nu + 1}(x) \\
                2 J'_\nu(x) &= J_{\nu - 1}(x) - J_{\nu + 1}(x).
            \end{align*}
        \end{solution}
    \end{parts}

    \question[Exerc\'{i}cio 11.1.4 do Arfken\nocite{Arfken:2005:Mathematical}] Mostre que vale a express\~{a}o de Jacobi-Anger
    \begin{align*}
        \exp(i x \cos \theta) = \sum_{k = -\infty}^{+\infty} i^k J_k(x) \exp(i k \theta).
    \end{align*}
    \begin{solution}
        Fazendo que $t = i \exp(i \theta)$, temos que
        \begin{align*}
            \frac{t - t^{-1}}{2} &= \frac{i \exp(i \theta)}{2} - \frac{\exp(-i \theta)}{2} \\
            &= \frac{i \exp(i \theta)}{2} + \frac{i \exp(-i \theta)}{2} \\
            &= i \left( \frac{\exp(i \theta) + \exp(-i \theta)}{2} \right) \\
            &= i \cos \theta.
        \end{align*}
        Logo, por \eqref{eq:bessel_pri_esp_geratriz} para $t = i \exp(i \theta)$ temos
        \begin{align*}
            \exp(i x \cos \theta) &= \sum_{k = -\infty}^{+\infty} (i \exp(i \theta))^k J_k(x) \\
            &= \sum_{k = -\infty}^{+\infty} i^k \exp(i k \theta) J_k(x).
        \end{align*}
    \end{solution}

    \question[Exerc\'{i}cio 11.1.5 do Arfken\nocite{Arfken:2005:Mathematical}] Mostre que
    \begin{parts}
        \part $\cos x = J_0(x) + 2 \sum_{k = 1}^\infty (-1)^k J_{2k}(x)$,
        \begin{solution}
            Fazendo $(t - t^{-1}) 2^{-1} = i$ temos que
            \begin{align*}
                \exp(i x) = \sum_{k = -\infty}^{+\infty}  i^k J_k(x) && \text{por \eqref{eq:bessel_pri_esp_geratriz}},
            \end{align*}
            e fazendo $(t - t^{-1}) 2^{-1} = -i$ temos que
            \begin{align*}
                \exp(-i x) = \sum_{k = -\infty}^{+\infty} (-i)^k J_k(z). && \text{por \eqref{eq:bessel_pri_esp_geratriz}}
            \end{align*}

            Ent\~{a}o,
            \begin{align*}
                \cos(x) &= \left( \exp(i x) + \exp(-i x) \right) 2^{-1} \\
                &= \left[ \sum_{k = -\infty}^{+\infty} J_k(x) (-1)^k i^k + \sum_{k = -\infty}^{+\infty} J_k(x) i^k \right] 2^{-1} \\
                &= \left[ \sum_{k = -\infty}^{+\infty} J_{-k}(x) i^k + \sum_{k = -\infty}^{+\infty} J_k(x) i^k \right] 2^{-1} \\
                &= \left[ \sum_{k = -\infty}^{+\infty} \left( J_{-k}(x) + J_{k}(x) \right) i^k \right] 2^{-1} && \text{por $\bigstar$} \\
                &= \left[ 2 \sum_{k = -\infty}^{+\infty} J_{k}(x) i^k \right] 2^{-1} \\
                &= \sum_{k = -\infty}^{+\infty} J_{k}(x) i^k \\
                &= J_0(x) + 2 \sum_{k = 1}^\infty (-1)^k J_{2k}(x),
            \end{align*}
            onde $\bigstar$ corresponde a soma de $k$'s sim\'{e}tricos.

            Para $k$ \'{i}mpar, i.e., $k = 2 \bar{k}$, temos
            \begin{align*}
                \sum_{k = 0}^\infty J_{2k + 1}(x) i^{-2k + 1} + \sum_{k = 0}^\infty J_{2k + 1}(x) i^{2k + 1} = 0
            \end{align*}
            e para $k$ par, i.e., $k = 2 \bar{k}$, temos
            \begin{align*}
                \sum_{k = 1}^\infty J_{2k}(x) i^{-2k} + \sum_{n = 1}^\infty J_{2k}(x) i^{2k} = 2 \sum_{n = 1}^\infty J_{2k}(z) i^{2k}.
            \end{align*}

            Logo,
            \begin{align*}
                \cos(x) &= J_0(x) + 2 \sum_{k = 0}^\infty (-1)^n J_{2n}(z).
            \end{align*}
        \end{solution}

        \part $\sin x = 2 \sum_{n = 0}^\infty (-1)^n J_{2n + 1}(x)$.
        \begin{solution}
            Fazendo $t = i$ em \eqref{eq:bessel_pri_esp_geratriz} temos que
            \begin{align*}
                \exp(i x^m) &= \sum_{k = -\infty}^{+\infty} J_k(x) i^k,
            \end{align*}
            e fazendo $t = -i$ temos que
            \begin{align*}
                \exp(-i x) &= \sum_{k = -\infty}^{+\infty} J_k(x) (-1)^k i^k.
            \end{align*}

            Ent\~{a}o temos que
            \begin{align*}
                \sin(x) &= \left( \exp(i x) - \exp(-i x) \right) (2 i)^{-1} \\
                &=  (-i) 2^{-1} \left( \exp(i x) - \exp(-i x) \right) \\
                &= (-i) 2^{-1} \left[ \sum_{k = -\infty}^{+\infty} J_k(x) i^k - \sum_{k = -\infty}^{+\infty} J_k(x) (-1)^k i^k \right] \\
                &= (-i) 2^{-1} \left[ J_0(x) + 2 \sum_{k = 1}^{+\infty} J_k(x) i^k - J_0(x) - 2 \sum_{k = 1}^{+\infty} J_k(x) (-1)^k i^k \right] \\
                &= (-i) 2^{-1} \left[ 2 \sum_{k = 1}^{+\infty} J_k(x) i^k - 2 \sum_{k = 1}^{+\infty} J_k(x) (-1)^k i^k \right] \\
                &= - \sum_{k = 1}^{+\infty} J_k(x) i^{k + 1} + \sum_{k = 1}^{+\infty} J_k(x) (-1)^k i^{k + 1}.
            \end{align*}

            Para $k$ par, i.e., $k = 2 \bar{k}$, temos
            \begin{align*}
                - \sum_{k = 1}^\infty J_{2k}(x) i^{2k + 1} + \sum_{k = 1}^\infty J_{2k}(x) i^{2k + 1} = 0
            \end{align*}
            e para $k$ \'{i}mpar, i.e., $k = 2 \bar{k} + 1$, temos
            \begin{align*}
                \begin{split}
                    - \sum_{k = 0}^\infty J_{2k + 1}(x) i^{2k + 2} + \sum_{k = 0}^\infty J_{2k + 1}(x) (-1)^{2k + 1} i^{2k + 2} =& - \sum_{n = 0}^\infty J_{2k + 1} (-1)^k (-1)^k \\
                    &\quad {}+ \sum_{k = 0}^\infty J_{2k + 1}(x) (-1) (-1)^k (-1)
                \end{split} \\
                &= 2 \sum_{k = 0}^\infty (-1)^k J_{2k + 1}(x).
            \end{align*}
            Logo,
            \begin{align*}
                \sin(x) &= 2 \sum_{k = 0}^\infty (-1)^k J_{2k + 1}(x).
            \end{align*}
        \end{solution}
    \end{parts}

    \question[Exerc\'{i}cio 11.1.2 do Arfken\nocite{Arfken:2005:Mathematical}] Mostre, a partir da fun\c{c}\~{a}o geratriz, que vale a f\'{o}rumla de adi\c{c}\~{a}o para as fun\c{c}\~{o}es de Bessel: $J_n(u + v) = \sum_{m = -\infty}^{+\infty} J_m(u) J_{n - m}(v)$.
    \begin{solution}
        Temos que
        \begin{align*}
            \sum_{n = -\infty}^{+\infty} t^n J_n(u + v) &= \exp\left( (u + v)(t - t^{-1})/2 \right) && \text{por \eqref{eq:bessel_pri_esp_geratriz}} \\
            &=\exp(u(t - t^{-1})/2) \exp(v(t - t^{-1})/2) \\
            &= \sum_{m = -\infty}^{+\infty} t^m J_m(u) \sum_{l = -\infty}^{+\infty} t^l J_l(v) && \text{por \eqref{eq:bessel_pri_esp_geratriz}} \\
            &= \sum_{m = -\infty}^{+\infty} \sum_{l = -\infty}^{+\infty} t^{m + l} J_m(u) J_l(v) \\
            &= \sum_{m = -\infty}^{+\infty} \sum_{n = -\infty}^{+\infty} t^{n} J_m(u) J_{n - m}(v) && n = l + m  \\
            &= \sum_{n = -\infty}^{+\infty} t^{n} \sum_{m = -\infty}^{+\infty} J_m(u) J_{n - m}(v)
        \end{align*}
        Por compara\c{c}\~{a}o termo a termos, concluimos que
        \begin{align*}
            J_n(u + v) &= \sum_{m = - \infty}^{+\infty} J_m(u) J_{n - m}(v).
        \end{align*}
    \end{solution}

    \question[Exerc\'{i}cio 11.1.10 do Arfken\nocite{Arfken:2005:Mathematical}] Mostre que
    \begin{align*}
        J_n(x) &= (-1)^n z^n \left( x^{-1} \,\mathrm{d}/\mathrm{d}x \right)^n J_0(x).
    \end{align*}
    \begin{solution}
        Iremos mostrar por indu\c{c}\~{a}o finita.

        Para $n = 0$ temos que
        \begin{align*}
            J_0(x) &= (-1)^0 (x)^0 \left( x^{-1} \,\mathrm{d}/\mathrm{d}x \right)^0 J_0(x) \\
            &= J_0(x).
        \end{align*}

        Como hip\'{o}tese de indu\c{c}\~{a}o temos que
        \begin{align*}
            J_n(x) &= (-1)^n x^n \left( x^{-1} \,\mathrm{d}/\mathrm{d}x \right)^n J_0(x).
        \end{align*}

        E como tese de indu\c{c}\~{a}o
        \begin{align*}
            J_{n + 1}(x) &= (-1)^{n + 1} x^{n + 1} \left( x^{-1} \,\mathrm{d}/\mathrm{d}x \right)^{n + 1} J_0(x).
        \end{align*}
        Mas pelo exerc\'{i}cio 1(b) desta lista temos que
        \begin{align*}
            \left( \mathrm{d}/\mathrm{d}x \right) \left( x^{-n} J_n(x) \right) = - x^{-n} J_{n + 1}(x) && \nu = n
        \end{align*}
        e manipulando a express\~{a}o acima temos que
        \begin{align*}
            J_{n + 1}(x) &= - x^n \left( \mathrm{d}/\mathrm{d}x \right) \left( x^{-n} J_n(x) \right) \\
            &= - x^n \left( \mathrm{d}/\mathrm{d}x \right) \left[ x^{-n} (-1)^n x^n \left( x^{-1} \mathrm{d}/\mathrm{d}x \right)^n J_0(x) \right] && \text{por $\bigstar$} \\
            &= (-1)^{n + 1} x^n \left( \mathrm{d}/\mathrm{d}x \right) \left[ \left( x^{-1} \mathrm{d}/\mathrm{d}x \right)^n J_0(x) \right] \\
            &= (-1)^{n + 1} x^{n + 1} z^{-1} \left( \mathrm{d}/\mathrm{d}z \right) \left[ \left( z^{-1} \mathrm{d}/\mathrm{d}z \right)^n J_0(z) \right] \\
            &= (-1)^{n + 1} x^{n + 1} \left( z^{-1} \mathrm{d}/\mathrm{d}z \right)^{n + 1} J_0(z),
        \end{align*}
        onde a \'{u}ltima express\~{a}o corresponde a nossa tese de indu\c{c}\~{a}o e $\bigstar$ a hip\'{o}tese de indu\c{c}\~{a}o.
    \end{solution}

    \question[E de 2006, Exerc\'{i}cio 11.1.9 do Arfken\nocite{Arfken:2005:Mathematical}] Seja $J_0(z)$ a fun\c{c}\~{a}o de Bessel de primeira esp\'{e}cie e ordem zero. Mostre que
    \begin{align*}
        J_0(z) &= \frac{2}{\pi} \int_0^1 \frac{\cos(zt)}{\sqrt{1 - t^2}} \,\mathrm{d}t.
    \end{align*}
    \begin{solution}
        Temos que
        \begin{align*}
            \frac{2}{\pi} \int_0^1 \frac{\cos(zt)}{(1 - t^2)^{1/2}} \,\mathrm{d}t &= \frac{2}{\pi} \int_0^1 (1 - t^2)^{-1/2} \left[ \sum_{n = 0}^\infty \frac{(-1)^n (zt)^{2n}}{(2n)!} \right] \,\mathrm{d}t && \text{expans\~{a}o em s\'{e}rie} \\
            &= \frac{2}{\pi} \sum_{n = 0}^\infty \frac{(-1)^n z^{2n}}{(2n)!} \int_0^1 (1 - t^2)^{-1/2} (t^2)^n \,\mathrm{d}t \\
            &= \frac{2}{\pi} \sum_{n = 0}^\infty \frac{(-1)^n z^{2n}}{(2n)!} \int_0^1 (1 - y)^{-1/2} y^{n - 1/2} \,\mathrm{d}y && t^2 = y \\
            &= \frac{1}{\pi} \sum_{n = 0}^\infty \frac{(-1)^n z^{2n}}{(2n)!} B(n + 1/2, 1/2) && \text{por \eqref{eq:beta_int}} \\
            &= \frac{1}{\pi} \sum_{n = 0}^\infty \frac{(-1)^n z^{2n}}{(2n)!} \frac{\Gamma(n + 1/2) \Gamma(1/2)}{\Gamma(n + 1)} && \text{por \eqref{eq:beta}} \\
            &= \frac{1}{\pi} \sum_{n = 0}^\infty \frac{(-1)^n z^{2n}}{(2n)!} \frac{\sqrt{\pi} \sqrt{\pi} \Gamma(2n)}{n! \Gamma(n) 2^{2n - 1}} \\
            &= \sum_{n = 0}^\infty \frac{(-1)^n z^{2n}}{(2n)!} \frac{\Gamma(2n)}{n! \Gamma(n) 2^{2n - 1}} \\
            &= \sum_{n = 0}^\infty \frac{(-1)^n z^{2n}}{2^{2n} n!} \frac{2n \Gamma(2n)}{(2n)! n \Gamma(n)} \\
            &= \sum_{n = 0}^\infty \frac{(-1)^n z^{2n}}{2^{2n} n!} \frac{\Gamma(2n + 1)}{(2n)! n \Gamma(n)} \\
            &= \sum_{n = 0}^\infty \frac{(-1)^n z^{2n}}{2^{2n} n!} \frac{\Gamma(2n + 1)}{(2n)! \Gamma(n + 1)} \\
            &= \sum_{n = 0}^\infty \frac{(-1)^n z^{2n}}{2^{2n} n!} \frac{1}{\Gamma(n + 1)} \\
            &= \sum_{n = 0}^\infty \frac{(-1)^n z^{2n}}{2^{2n} n! n!} \\
            &= \sum_{n = 0}^\infty \frac{(-1)^n}{n! \Gamma(n + 0 + 1)} \frac{z^{2n}}{2^{2n}} \\
            &= J_0(z) && \text{por \eqref{eq:bessel_pri_esp}}.
        \end{align*}
    \end{solution}

    \question Mostre que
    \begin{align*}
        J_v(x) &= \frac{2(x/2)^\nu}{\sqrt{\pi} \Gamma(\nu + 1/2)} \int_0^{\pi/2} \cos(x \sin \theta) \cos^{2\nu} \theta \,\mathrm{d}\theta.
    \end{align*}
    \begin{solution}
        Temos que
        \begin{align*}
            \int_0^{\pi/2} \cos(x \sin \theta) \cos^{2 \nu} \theta \,\mathrm{d}\theta &= \int_0^{\pi/2} \sum_{k = 0}^\infty \frac{(-1)^k (x \sin \theta)^{2k}}{(2k)!} \cos^{2\nu} \theta \,\mathrm{d}\theta && \text{s\'{e}rie de Taylor} \\
            &= \sum_{k = 0}^\infty \int_0^{\pi/2} \frac{(-1)^k x^{2k} \sin^{2k} \theta}{(2k)!} \cos^{2\nu} \theta \,\mathrm{d}\theta \\
            &= \sum_{k = 0}^\infty \frac{(-1)^k x^{2k}}{(2k)!} \int_0^{\pi/2} \sin^{2k} \theta \cos^{2\nu} \theta \,\mathrm{d}\theta \\
            &= \sum_{k = 0}^\infty \frac{(-1)^k x^{2k}}{(2k)!} 2^{-1} B(k + 1/2, \nu + 1/2) && \text{por \eqref{eq:beta_trig}} \\
            &= \sum_{k = 0}^\infty \frac{(-1)^k x^{2k}}{(2k)!} 2^{-1} \frac{\Gamma(k + 1/2) \Gamma(\nu + 1/2)}{\Gamma(k + \nu + 1)} && \text{por \eqref{eq:beta}} \\
            &= \sum_{k = 0}^\infty \frac{(-1)^k x^{2k}}{(2k)!} 2^{-1} \frac{\sqrt{\pi} \Gamma(2k)}{2^{2k - 1} \Gamma(k)} \frac{\Gamma(\nu + 1/2)}{\Gamma(k + \nu + 1)} && \text{por \eqref{eq:gamma_dup_legendre}} \\
            &= \sum_{k = 0}^\infty \frac{(-1)^k x^{2k}}{(2k)!} \frac{\sqrt{\pi} \Gamma(2k)}{2^{2k} \Gamma(k)} \frac{\Gamma(\nu + 1/2)}{\Gamma(k + \nu + 1)} && \text{por \eqref{eq:gamma_dup_legendre}} \\
            &= \sum_{k = 0}^\infty \frac{(-1)^k (x/2)^{2k}}{(2k)!} \frac{\sqrt{\pi} \Gamma(2k)}{\Gamma(k)} \frac{\Gamma(\nu + 1/2)}{\Gamma(k + \nu + 1)} \\
            &= \sum_{k = 0}^\infty \frac{(-1)^k (x/2)^{2k}}{\Gamma(2k + 1)} \frac{\sqrt{\pi} \Gamma(2k)}{\Gamma(k)} \frac{\Gamma(\nu + 1/2)}{\Gamma(k + \nu + 1)} \\
            &= \sum_{k = 0}^\infty \frac{(-1)^k (x/2)^{2k}}{2k \Gamma(2k)} \frac{\sqrt{\pi} \Gamma(2k)}{\Gamma(k)} \frac{\Gamma(\nu + 1/2)}{\Gamma(k + \nu + 1)} \\
            &= \sum_{k = 0}^\infty \frac{(-1)^k (x/2)^{2k}}{2k} \frac{\sqrt{\pi}}{\Gamma(k)} \frac{\Gamma(\nu + 1/2)}{\Gamma(k + \nu + 1)} \\
            &= \sum_{k = 0}^\infty \frac{(-1)^k (x/2)^{2k}}{2} \frac{\sqrt{\pi}}{\Gamma(k + 1)} \frac{\Gamma(\nu + 1/2)}{\Gamma(k + \nu + 1)} \\
            &= \sum_{k = 0}^\infty \frac{(-1)^k (x/2)^{2k}}{2} \frac{\sqrt{\pi}}{\Gamma(k + 1)} \frac{\Gamma(\nu + 1/2)}{\Gamma(k + \nu + 1)}
        \end{align*}
        Logo,
        \begin{align*}
            \frac{2 (x/2)^\nu}{\sqrt{\pi} \Gamma(\nu + 1/2)} \int_0^{\pi/2} \cos(x \sin \theta) \cos^{2 \nu} \,\mathrm{d}\theta &= \frac{2 (x/2)^\nu}{\sqrt{\pi} \Gamma(\nu + 1/2)} \sum_{k = 0}^\infty \frac{(-1)^k (x/2)^{2k} \sqrt{\pi} \Gamma(\nu + 1/2)}{2 \Gamma(k + 1) \Gamma(k + \nu + 1)} \\
            &= \sum_{k = 0}^\infty \frac{2 (x/2)^\nu (-1)^k (x/2)^{2k} \sqrt{\pi} \Gamma(\nu + 1/2)}{\sqrt{\pi} \Gamma(\nu + 1/2) 2 \Gamma(k + 1) \Gamma(k + \nu + 1)} \\
            &= \sum_{k = 0}^\infty \frac{2 (x/2)^\nu (-1)^k (x/2)^{2k} \sqrt{\pi}}{\sqrt{\pi} 2 \Gamma(k + 1) \Gamma(k + \nu + 1)} \\
            &= \sum_{k = 0}^\infty \frac{(x/2)^\nu (-1)^k (x/2)^{2k}}{\Gamma(k + 1) \Gamma(k + \nu + 1)} \\
            &= \sum_{k = 0}^\infty \frac{(-1)^k (x/2)^{2k + \nu}}{k! \Gamma(k + \nu + 1)} \\
            &= J_\nu(x)
        \end{align*}
    \end{solution}

    \question[Exerc\'{i}cio V-33 do Farrell\nocite{Farrell:1971:Solved}] Mostre que $\int_0^\infty J_0(x) \,\mathrm{d}x = 1$, assumindo que a integral $\int_0^\infty J_n(x) \,\mathrm{d}x$, $n = 0, 1, 2, \ldots$ \'{e} convergente e dado que
    \begin{align}
        \int_0^\pi \frac{\cos(n \phi) \,\mathrm{d}\phi}{a - i b \cos \phi} &= \frac{\pi i^n}{\sqrt{a^2 + b^2}} \left[ \frac{\sqrt{a^2 + b^2} - a}{b}^n \right], & b \neq 0, i = \sqrt{-1}, \tag{$\bigstar$} \label{eq:farrell:v-33.2} \\
        J_n(x) &= \frac{(-i)^n}{\pi} \int_0^\pi \exp(i x \cos \phi) \cos(n \phi) \,\mathrm{d}\phi, & i = \sqrt{-1}, n = 0, 1, 2, \ldots. \tag{$\bigstar\bigstar$} \label{eq:farrell:v-33.3}
    \end{align}
    \begin{solution}
        Se $a$ \'{e} qualquer constante maior que zero, ent\~{a}o a integral $\int_0^\infty \exp(-a x) J_n(bx) \,\mathrm{d}x$ \'{e} convergente. Ent\~{a}o, utilizando \eqref{eq:farrell:v-33.3} temos
        \begin{align*}
            \int_0^\infty \exp(-a x) J_n(bx) \,\mathrm{d}x &= \int_0^\infty \exp(-a x) \left[ \frac{(-i)^n}{n} \int_0^\pi \exp(i b x \cos \phi) \cos(n \phi) \,\mathrm{d}\phi \right] \,\mathrm{d}x \\
            &= \frac{(-i)^n}{n} \int_0^\infty \exp(-a x) \left[ \int_0^\pi \exp(i b x \cos \phi) \cos(n \phi) \,\mathrm{d}\phi \right] \,\mathrm{d}x 
        \end{align*}
        Se considerarmos $x$ e $\phi$ como um plano de coordenadas retangulares, ent\~{a}o podemos considerar a integral multipla anterior equivalente a integral dupla na regi\~{a}o $R$ do plano $x\phi$ correspondente aos pontos $(x, \phi)$ tal que $x \geq 0$ e $0 \leq \phi \leq \phi$. E assim a integral dupla na regi\~{a}o $R$ pode ser igual a integral m\'{u}ltipla na qual a ordem de integra\c{c}\~{a}o encontra-se invertida. Ent\~{a}o
        \begin{align*}
            \int_0^\infty \exp(-a x) J_n(bx) \,\mathrm{d}x &= \frac{(-i)^n}{n} \int_0^\pi \left[ \int_0^\infty \exp(-a x) \exp(i b x \cos \phi) \cos(n \phi) \,\mathrm{d}x \right] \,\mathrm{d}\phi \\
            &= \frac{(-i)^n}{n} \int_0^\pi \left[ \int_0^\infty \exp(-a x) \exp(i b x \cos \phi) \,\mathrm{d}x \right] \cos(n \phi) \,\mathrm{d}\phi \\
            &= \frac{(-i)^n}{n} \int_0^\pi \left[ \int_0^\infty \exp(-(a - i b \cos \phi)x) \,\mathrm{d}x \right] \cos(n \phi) \,\mathrm{d}\phi \\
            &= \frac{(-i)^n}{n} \int_0^\pi \left[ \left. \frac{\exp(-(a - ib \cos\phi)x)}{-(a - i b \cos\phi)} \right|_{x = 0}^{x = \infty} \right] \cos(n \phi) \,\mathrm{d}\phi \\
            &= \frac{(-i)^n}{n} \int_0^\pi \frac{1}{a - i b \cos\phi} \cos(n \phi) \,\mathrm{d}\phi \\
            &= \frac{1}{\sqrt{a^2 + b^2}} \left[ \frac{\sqrt{a^2 + b^2} - a}{b} \right]^n,
        \end{align*}
        onde a \'{u}ltima passagem corresponde a aplica\c{c}\~{a}o de \eqref{eq:farrell:v-33.2}.

        A express\~{a}o
        \begin{align}
            \int_0^\infty \exp(-ax) J_n(bx) \,\mathrm{d}x &= \frac{1}{\sqrt{a^2 + b^2}} \left[ \frac{\sqrt{a^2 + b^2} - a}{b} \right]^n \label{eq:farrell:v-33.6}
        \end{align}
        foi obtida sob a hip\'{o}tese de que $a > 0$, mas a integral do lado esquerdo \'{e} convergente para $a \geq 0$ e fazendo $a = 0$ verificamos a validade da express\~{a}o. Tomando $a = 0$, $b = 1$ e $n = 0$ obtemos a igualdade desejada.
    \end{solution}

    \question Mostre que
    \begin{align*}
        \mathcal{L}[J_0(x)](p) &= \int_0^\infty \exp(-px) J_0(x) dx = (1 + p^2)^{-1/2}.
    \end{align*}
    \begin{solution}
        A rela\c{c}\~{a}o desejada decorre de \eqref{eq:farrell:v-33.6} ao tomar $a = p$, $b = 0$ e $n = 0$.

        Tamb\'{e}m temos que
        \begin{align*}
            \mathcal{L}[J_0(x)](p) &= \int_0^\infty \exp(-p x) J_0(x) ,\mathrm{d}x \\
            &= \int_0^\infty \exp(-p x) \sum_{k = 0}^\infty \frac{(-1)^k}{k! \Gamma(k + 1)} \left( \frac{x}{2} \right)^{2k} \,\mathrm{d}x && \text{por \eqref{eq:bessel_pri_esp}} \\
            &= \sum_{k = 0}^\infty \frac{(-1)^k}{k! k! 2^{2k}} \int_0^\infty x^{2k} \exp(-p x) \,\mathrm{d}x && \text{por \eqref{eq:gamma_rel}} \\
            &= \frac{p^{2k + 1}}{p^{2k + 1}} \sum_{k = 0}^\infty \frac{(-1)^k}{k! k! 2^{2k}} \int_0^\infty x^{2k} \exp(-p x) \,\mathrm{d}x \\
            &= \sum_{k = 0}^\infty \frac{(-1)^k}{k! k! 2^{2k} p^{2k + 1}} \int_0^\infty (px)^{(2k + 1) - 1} \exp(-p x) \,\mathrm{d}(px) \\
            &= \sum_{k = 0}^\infty \frac{(-1)^k}{k! k! 2^{2k} p^{2k + 1}} \Gamma(2 k + 1) && \text{por \eqref{eq:gamma_euler}} \\
            &= \sum_{k = 0}^\infty \frac{(-1)^k}{k! k! 2^{2k} p^{2k + 1}} \frac{1}{\sqrt{\pi}} 2^{2k} \Gamma(k + 1/2) \Gamma(k + 1) && \text{por \eqref{eq:gamma_dup_legendre}} \\
            &= \sum_{k = 0}^\infty \frac{(-1)^k}{k! k! 2^{2k} p^{2k + 1}} \frac{1}{\sqrt{\pi}} 2^{2k} (1/2)_k \Gamma(1/2) k! \\
            &= \sum_{k = 0}^\infty \frac{(-1)^k}{k! k! 2^{2k} p^{2k + 1}} 2^{2k} (1/2)_k k! \\
            &= \frac{1}{p} \sum_{k = 0}^\infty \frac{(-1)^k (1/2)_k}{K!} \left( \frac{1}{p} \right)^{2k} \\
            &= \frac{1}{p} \sum_{k = 0}^\infty \frac{(-1)^k [(2k - 1)!!]^2}{(2k)!} \left( \frac{1}{p} \right)^{2k} \\
            &= \frac{1}{p} \sum_{k = 0}^\infty (-1)^k [(2k - 1)!!]^2 \frac{(1/p)^{2k}}{(2k)!} \\
            &= \frac{1}{\sqrt{1 + p^2}}
        \end{align*}
        onde o \'{u}ltimo passo \'{e} justificado pela expans\~{a}o de $(1 + z^2)^{-1/2}$, $z = 1/p$, em s\'{e}rie de Taylor.
    \end{solution}

    \question Mostre que
    \begin{align*}
        \int_0^\infty J_n(ax) dx &= a^{-1}, n = 0, 1, 2, \ldots
    \end{align*}
    \begin{solution}
        A rela\c{c}\~{a}o desejada decorre de \eqref{eq:farrell:v-33.6} ao tomar $a = 0$, $b = a$ e $n = 0$.

        Tamb\'{e}m temos que
        \begin{align*}
            \int_0^\infty J_n(a x) \,\mathrm{d}x &= a^{-1} \int_0^\infty J_n(z) \,\mathrm{d}z.
        \end{align*}
        Vamos mostrar que $\int_0^\infty J_n(z) \,\mathrm{d}z = 1$, ent\~{a}o
        \begin{align*}
            \int_0^\infty J_n(z) \,\mathrm{d}z &= \int_0^\infty \left[ J_{n - 2}(z) - 2 J'_{n - 1}(z) \right] \,\mathrm{d}z \\
            &= \int_0^\infty J_{n - 2}(z) \,\mathrm{d}z - \left. 2 J_{n - 1}(z) \right|_0^\infty \\
            &= \int_0^\infty J_{n - 2}(z) \,\mathrm{d}z.
        \end{align*}
        A express\~{a}o acima nos fornece uma rela\c{c}\~{a}o de recorr\^{e}ncia de modo que precisamos mostrar apenas que $\int_0^\infty J_0(z) \,\mathrm{d}z = 1$, ver gr\'{a}fico, e que $\int_0^\infty J_1(z) \,\mathrm{d}z = 1$, usar $J_1(x) = - J'_0(x)$.
    \end{solution}

    \question Mostre que a equa\c{c}\~{a}o $x^4 y'' + (\exp(2/x) - v^2)y = 0$ \'{e} satisfeita por $y = x J_v(\exp(1/x))$.
    \begin{solution}
        Temos que $J_\nu(z)$ satisfaz \eqref{eq:bessel}. Escrevendo $y(x) = x J_\nu(z)$, onde $z = \exp(1/x)$, temos
        \begin{align*}
            \frac{\mathrm{d}y}{\mathrm{d}x} &= J_\nu(z) \frac{\mathrm{d}x}{\mathrm{d}x} + x J'\nu(z) \frac{\mathrm{d}z}{\mathrm{d}x} \\
            &= J_\nu(z) + z x^{-1} J'_\nu(z) && \frac{\mathrm{d}z}{\mathrm{d}x} = \frac{-\exp(1/x)}{x^2} = \frac{-z}{x^2}, \\
            \frac{\mathrm{d}^2y}{\mathrm{d}x^2} &= J'_\nu(z) \frac{\mathrm{d}z)}{\mathrm{d}x} - \frac{1}{x} J'_\nu(z) \frac{\mathrm{d} z}{\mathrm{d}x} + \frac{z}{x^2} J'_\nu(z) - \frac{z}{x} J''_v(z) \frac{\mathrm{d}z}{\mathrm{d}x} \\
            &= \frac{z^2}{x^3} J''_\nu(z) + \frac{z}{x^3} J'_\nu(z).
        \end{align*}
        Substituindo na equa\c{c}\~{a}o desejada temos
        \begin{align*}
            x^4 y'' + (\exp(2/x) - \nu^2) y &= x z^2 J''_\nu(z) + x \ J'_\nu(z) + (z^2 - \nu^2) x J_\nu(z) \\
            &= x \left[ z^2 J''_\nu(z) + z J'_\nu(z) + (z^2 - \nu^2) J_\nu(z) \right] \\
            &= x \left[ 0 \right] && \eqref{eq:bessel} \\
            &= 0.
        \end{align*}
    \end{solution}

    \question[Exerc\'{i}cio VI-2 do Farrell\nocite{Farrell:1971:Solved}] Mostre que a solu\c{c}\~{a}o geral de
    \begin{align*}
        x^2 \frac{d^2y}{dy^2} + (1 - 2\alpha)x\frac{dy}{dx} + \left[ \beta^2 \gamma^2 x^{2\gamma} + (\alpha^2 - n^2 \gamma^2) \right]y &= 0
    \end{align*}
    \'{e} dada por $A x^\alpha J_n(\beta x^\gamma) + B x^\alpha Y_n(\beta x^\gamma)$.
    \begin{solution}
        Devemos mostrarmos que $x^\alpha J_n(\beta x^\gamma)$ e $x^\alpha Y_n(\beta x^\gamma)$ s\~{a}o solu\c{c}\~{o}es da equa\c{c}\~{a}o desejada. Comecemos com $y = x^\alpha J_n(\beta x^\gamma)$. Ent\~{a}o
        \begin{align*}
            y' &= \frac{\mathrm{d}}{\mathrm{d}x} \left[ x^\alpha J_n \right] \\
            &= x^\alpha \frac{\mathrm{d}}{\mathrm{d}x} J_n + J_n \frac{\mathrm{d}}{\mathrm{d}x} x^\alpha \\
            &= x^\alpha (\beta \gamma x^{\gamma - 1}) J'_n + J_n \alpha x^{\alpha - 1} \\
            &= \beta \gamma x^{\alpha + \gamma - 1} J'_n + \alpha x^{\alpha - 1} J_n \\
            y'' &= \beta^2 \gamma^2 x^{\alpha + 2 \gamma - 2} J''_n + \beta \gamma \left[ (\alpha + \gamma - 1) x^{\alpha + \gamma - 2} + \alpha x^{\alpha + \gamma - 2} \right] J'_n + \alpha (\alpha - 1) x^{\alpha - 2} J_n
        \end{align*}
        Substituindo na equa\c{c}\~{a}o temos
        \begin{align*}
            \gamma^2 x^\alpha \left[ \beta^2 x^{2 \gamma} J''_n + \beta x^\gamma J'_n + (\beta^2 x^{2\gamma} - n^2) J_n \right] &= 0,\\
            \gamma^2 x^\alpha \left[ (\beta^2 x^\gamma)^2 J''_n + \beta x^\gamma J'_n + ( (\beta x^\gamma)^2 - n^2) J_n \right] &= 0.
        \end{align*}
        Notamos que a express\~{a}o entre colchetes \'{e} \eqref{eq:bessel} e assim concluimos que $y = x^\alpha J_n(\beta x^\gamma)$ satisfaz a equa\c{c}\~{a}o desejada.

        Para $y = x^\alpha Y_n(\beta x^\gamma)$ o processo \'{e} semelhante.
    \end{solution}

    As fun\c{c}\~{o}es de Bessel esf\'{e}ricas $j_n(x)$ e $y_n(x)$ s\~{a}o definidas como
    \begin{align}
        j_{n}(x) &= \left( \frac{\pi}{2x} \right)^{-1/2} J_{n + 1/2}(x), \label{eq:bessel_esf_prim} \\
        y_n(x) &= \left( \frac{\pi}{2x} \right)^{-1/2} Y_{n + 1/2}(x) = (-1)^{n + 1} \left( \frac{\pi}{2x} \right)^{-1/2} J_{-n - 1/2}(x). \label{eq:bessel_esf_seg}
    \end{align}

    \question Mostre que a $n$-\'{e}sima fun\c{c}\~{a}o de Bessel esf\'{e}rica \'{e} dado por
    \begin{align*}
        f_l(x) &= (-1)^l x^l \left( \frac{1}{x} \frac{\mathrm{d}}{\mathrm{d}x} \right)^l f_0(x),
    \end{align*}
    onde $f_l(x)$ denota tanto $j_l(x)$ como $y_n(x)$.
    \begin{solution}
        Pelo exerc\'{i}cio 1(a) desta lista temos que
        \begin{align*}
            J_{\nu + 1}(x) &= -x^\nu \frac{\mathrm{d}}{\mathrm{d}x} \left[ x^{-\nu} J_\nu(x) \right].
        \end{align*}
        Tomando $\nu = l + 1/2$ na express\~{a}o acima temos
        \begin{align*}
            J_{l + 1/2 + 1}(x) &= -x^{l + 1/2} \frac{\mathrm{d}}{\mathrm{d}x} \left[ x^{-(l + 1/2)} J_{l + 1/2}(x) \right] \\
            J_{l + 3/2}(x) &= -x^{l + 1/2} \frac{\mathrm{d}}{\mathrm{d}x} \left[ x^{-l - 1/2)} J_{l + 1/2}(x) \right] \\
            J_{l + 3/2}(x) &= - x^{l + 1/2} \frac{\mathrm{d}}{\mathrm{d}x} \left[ \frac{x^{-1/2} J_{l + 1/2}(x)}{x^l} \right] \\
            x^{-1/2} J_{l + 3/2}(x) &= - x^l \frac{\mathrm{d}}{\mathrm{d}x} \left[ \frac{x^{-1/2} J_{l + 1/2}(x)}{x^l} \right] \\
            j_{l + 1}(x) &= - x^l \frac{\mathrm{d}}{\mathrm{d}x} \left[ x^{-l} j_{l}(x) \right] && \text{por \eqref{eq:bessel_esf_prim}} \\
            j_{l}(x) &= - x^{l - 1} \frac{\mathrm{d}}{\mathrm{d}x} \left[ x^{-l + 1} j_{l - 1}(x) \right] && l + 1 \to l 
        \end{align*}
        Aplicando a express\~{a}o acima recursivamente temos
        \begin{align*}
            j_{l}(x) &= - x^{l - 1} \frac{\mathrm{d}}{\mathrm{d}x} \left[ x^{-l + 1} j_{l - 1}(x) \right] \\
            &= - x^{l - 1} \frac{\mathrm{d}}{\mathrm{d}x} \left[ x^{-l + 1} (-1) x^{l - 2} \frac{\mathrm{d}}{\mathrm{d}x} \left[ x^{-l + 2} j_{l - 2}(x) \right] \right] \\
            &= (-1)^2 x^{l - 1} \frac{\mathrm{d}}{\mathrm{d}x} \left[ x^{-l + 1} x^{l - 2} \frac{\mathrm{d}}{\mathrm{d}x} \left[ x^{-l + 2} j_{l - 2}(x) \right] \right] \\
            &= (-1)^2 x^{l - 1} \frac{\mathrm{d}}{\mathrm{d}x} \left[ x^{-1} \frac{\mathrm{d}}{\mathrm{d}x} \left[ x^{-l + 2} j_{l - 2}(x) \right] \right] \\
            &= (-1)^2 \frac{x^l}{x} \frac{\mathrm{d}}{\mathrm{d}x} \left[ x^{-1} \frac{\mathrm{d}}{\mathrm{d}x} \left[ x^{-l + 2} j_{l - 2}(x) \right] \right] \\
            &= \ldots \\
            &= (-1)^l x^l \left( \frac{1}{x} \frac{\mathrm{d}}{\mathrm{d}x} \right)^l j_0(x).
        \end{align*}

        Para $y_l(x)$ o processo \'{e} an\'{a}logo tomando no in\'{i}cio $\nu = l - 1/2$ e utilizando que $Y_{l + 1/2}(x) = (-1)^{l + 1} J_{-l - 1/2}(x)$.
    \end{solution}

    \question Mostre que
    \begin{parts}
        \part $j_n(x) = (-1)^n x^n \left( x^{-1} d/dx \right)^n x^{-1} \sin x$,
        \begin{solution}
            Sabendo que $j_0(x) = \sin(x) x^{-1}$ e pelo exerc\'{i}cio anterior utilizamos uma simples substitui\c{c}\~{a}o para concluir que $j_n(x) = (-1)^n x^n \left( x^{-1} d/dx \right)^n x^{-1} \sin x$.
        \end{solution}

        \part $y_n(x) = (-1)^{n + 1} x^n \left( x^{-1} d/dx \right)^n x^{-1} \cos x$.
        \begin{solution}
            Sabendo que $y_0(x) = - \cos(x) x^{-1}$ e pelo exerc\'{i}cio anterior utilizamos uma simples substitui\c{c}\~{a}o para concluir que $y_n(x) = (-1)^{n + 1} x^n \left( x^{-1} d/dx \right)^n x^{-1} \cos x$.
        \end{solution}
    \end{parts}

    \question Mostre que
    \begin{parts}
        \part $j_{n - 1}(x) + j_{n + 1}(x) = x^{-1} (2 n + 1) j_n(x)$,
        \begin{solution}
            
        \end{solution}

        \part $n j_{n - 1}(x) - (n _ 1)j_{n + 1}(x) = (2n + 1)j'_n(x)$,
        \begin{solution}
            
        \end{solution}

        \part $y_{n - 1}(x) + y_{n + 1}(x) = x^{-1} (2n + 1) y_n(x)$,
        \begin{solution}
            
        \end{solution}

        \part $n y_{n - 1}(x) - (n + 1)y_{n + 1}(x) = (2n + 1) y'_n(x)$.
        \begin{solution}
            
        \end{solution}
    \end{parts}

    As fun\c{c}\~{o}es de Bessel esf\'{e}ricas modificadas $i_n(x)$ e $k_n(x)$ s\~{a}o definidas como
    \begin{align}
        i_n(x) &= \sqrt{\frac{\pi}{2x}} I_{n + 1/2}(x), \label{eq:bessel_esf_mod_prim} \\
        k_n(x) &= \sqrt{\frac{2}{\pi x}} K_{n + 1/2}(x), \label{eq:bessel_esf_mod_seg}
    \end{align}
    onde $I_\nu(x)$ e $K_\nu(x)$ s\~{a}o as fun\c{c}\~{o}es de Bessel modificadas de primeira e segunda esp\'{e}cie.
    \question Mostre que
    \begin{parts}
        \part $i_{n + 1}(x) = x^n \,\mathrm{d}(x^{-n} i_n(x)) / \mathrm{d}x$,
        \begin{solution}
            
        \end{solution}

        \part $k_{n + 1}(x) = -x^n \,\mathrm{d}(x^{-n} k_n(x)) / \mathrm{d}x$.
        \begin{solution}
            
        \end{solution}
    \end{parts}

    \question Mostre que
    \begin{parts}
        \part $i_n(x) = x^{n} (x^{-1} \,\mathrm{d} / \mathrm{d}x )^n (\sinh(x) x^{-1})$,
        \begin{solution}
            
        \end{solution}

        \part $k_n(x) = (-1)^n x^{n} (x^{-1} \,\mathrm{d} / \mathrm{d}x )^n (\exp(-x) x^{-1})$.
        \begin{solution}
            
        \end{solution}
    \end{parts}

    \question Mostre que
    \begin{parts}
        \part $i_{n - 1}(x) - i_{n + 1}(x) = (2n + 1) x^{-1} i_n(x)$,
        \begin{solution}
            
        \end{solution}

        \part $n i_{n - 1}(x) + (n + 1) i_{n + 1}(x) = (2n + 1) i'_n(x)$,
        \begin{solution}
            
        \end{solution}

        \part $k_{n - 1}(x) - k_{n + 1}(x) = - (2n + 1) x^{-1} k_n(x)$,
        \begin{solution}
            
        \end{solution}

        \part $n k_{n - 1}(x) + (n + 1) k_{n + 1}(x) = - (2n + 1) k'_n(x)$.
        \begin{solution}
            
        \end{solution}
    \end{parts}

    \question[P2 de 2006] Fa\c{c}a um esbo\c{c}o do gr\'{a}fico das fun\c{c}\~{o}es de Bessel de primeira esp\'{e}cie e de ordem zero, um e dois.
    \begin{solution}
        Como apresentado nas notas de aula, as fun\c{c}\~{o}es de Bessel de primeira esp\'{e}cie apresentam comportamento oscilat\'{o}rio e que apenas $J_0(x)$ n\~{a}o se anula para $x = 0$.
        \begin{center}
            \includegraphics[width=0.8\textwidth]{BesselJPlot.png}
        \end{center}
    \end{solution}

    \question[P3 de 2006] As fun\c{c}\~{o}es de Bessel esf\'{e}ricas $j_n(x)$ s\~{a}o definidas atrav\'{e}s de
    \begin{align*}
        j_n(x) &= (\pi / (2x) )^{1/2} J_{n + 1/2}(x),
    \end{align*}
    onde $J_\nu(x)$ s\~{a}o as fun\c{c}\~{o}es de Bessel de primeira esp\'{e}cie e de ordem $\nu$. Mostre que
    \begin{align*}
        j_0(x) &= \sin(x) / x.
    \end{align*}
    \begin{solution}
        Temos que
        \begin{align*}
            j_n(x) &= (\pi / (2x) )^{1/2} J_{n + 1/2}(x) \\
            &= (\pi / (2x) )^{1/2} \sum_{k = 0}^\infty \frac{(-1)^k x^{2k + 1/2}}{2^{2k + 1/2} k! \Gamma(k + 1/2 + 1)} && \text{por \eqref{eq:bessel_pri_esp}} \\
            &= \sqrt{\pi} \sum_{k = 0}^\infty \frac{(-1)^k x^{2k}}{2^{2k + 1} \Gamma(k + 1) \Gamma(k + 1 + 1/2)} \\
            &= \sqrt{\pi} \sum_{k = 0}^\infty \frac{(-1)^k x^{2k}}{\sqrt{\pi} \Gamma(2k + 2)} && \text{por \eqref{eq:gamma_dup_legendre}} \\
            &= \sum_{k = 0}^\infty \frac{(-1)^k x^{2k}}{\Gamma(2k + 2)} \\
            &= \sum_{k = 0}^\infty \frac{(-1)^k x^{2k}}{(2k + 1)!} \\
            &= x^{-1} \sum_{k = 0}^\infty \frac{(-1)^k x^{2k + 1}}{(2k + 1)!} \\
            &= x^{-1} \sin(x) && \text{pela s\'{e}rie de Taylor}
        \end{align*}
    \end{solution}

    \question[P2 de 2010] Considere a equa\c{c}\~{a}o de Bessel $y''(x) + x^{-1} y'(x) + (1 - v^2 x^{-2}) y(x) = 0$ e suas solu\c{c}\~{o}es de primeira esp\'{e}cie $J_\nu, \nu \geq 0$.
    \begin{parts}
        \part Mostre que $J_0$ possui um n\'{u}mero infinito de zeros no intervalo $(0, \infty)$ (n\~{a}o vale usar a express\~{a}o assint\'{o}tica de $J_0$ sem deduz\'{i}-la). Dica: transforma\c{c}\~{a}o de vari\'{a}veis $y(x) = x^{-1/2} u(x)$.
        \begin{solution}
            
        \end{solution}

        \part Deduza uma f\'{o}ruma para $J_{1/2}$ em termos de fun\c{c}\~{o}es elementares. Fa\c{c}a o mesmo para $J_{3/2}$.
        \begin{solution}
            
        \end{solution}
    \end{parts}

    \question[T5 de 2011] Mostre que
    \begin{align*}
        J_\nu(z) &= \frac{(z/2)^\nu}{\sqrt{\pi} \Gamma(\nu + 1/2)} \int_{-1}^1 (1 - t^2)^{\nu - 1/2} \exp(i z t) \, \mathrm{d}t.
    \end{align*}
    \begin{solution}
        Temos que
        \begin{align*}
            I &= \int_{-1}^1 (1 - t^2)^{\nu - 1/2} \exp(i z t) \,\mathrm{d}t \\
            &= \int_{-1}^1 (1 - t^2)^{\nu - 1/2} \left( \cos(zt) + i \sin(zt) \right) \,\mathrm{d}t \\
            &= \int_{-1}^1 \underbrace{(1 - t^2)^{\nu - 1/2} \cos(zt)}_{\text{fun\c{c}\~{a}o par}} \,\mathrm{d}t + i \int_{-1}^1 \underbrace{(1 - t^2)^{\nu - 1/2} \sin(zt)}_{= 0 \text{, fun\c{c}\~{a}o \'{i}mpar}} \,\mathrm{d}t \\
            &= 2 \int_0^1 (1 - t^2)^{\nu - 1/2} \cos(zt) \,\mathrm{d}t \\
            &= 2 \sum_{n = 0}^\infty \frac{(-1)^n z^{2n}}{(2n)!} \int_0^1 (1 - t^2)^{\nu - 1/2} t^{2n} \,\mathrm{d}t && \text{expans\~{a}o em s\'{e}rie} \\
            &= \sum_{n = 0}^\infty \frac{(-1)^n z^{2n}}{(2n)!} \int_0^1 (1-y)^{\nu - 1/2} y^{n - 1/2} \,\mathrm{d}t && t^2 = y \\
            &= \sum_{n = 0}^\infty \frac{(-1)^n z^{2n}}{(2n)!} B(\nu + 1/2, n + 1/2) && \text{por \eqref{eq:beta_int}} \\
            &= \sum_{n = 0}\infty \frac{(-1)^n z^{2n}}{(2n)!} \frac{\Gamma(\nu + 1/2) \Gamma(n + 1/2)}{\Gamma(\nu + n + 1)} && \text{por \eqref{eq:beta}} \\
            &= \sqrt{\pi} \Gamma(\nu + 1/2) \sum_{n = 0}^\infty \frac{(-1)^n z^{2n}}{(2n)!} \frac{\Gamma(2n)}{\Gamma(n) 2^{2n - 1} \Gamma(\nu + n + 1)} && \text{por \eqref{eq:gamma_dup_legendre}} \\
            &= \sqrt{\pi} \Gamma(\nu + 1/2) \sum_{n = 0}^\infty \frac{(-1)^n z^{2n}}{(2n) \Gamma(2n)} \frac{\Gamma(2n)}{\Gamma(n) 2^{n - 1} \Gamma(\nu + n + 1)} \\
            &= \sqrt{\pi} \Gamma(\nu + 1/2) \sum_{n = 0}^\infty \frac{(-1)^n z^{2n}}{(2n)} \frac{1}{\Gamma(n) 2^{n - 1} \Gamma(\nu + n + 1)} \\
            &= \sqrt{\pi} \Gamma(\nu + 1/2) \sum_{n = 0}^\infty \frac{(-1)^n}{\Gamma(n + 1) \Gamma(\nu + n + 1)} \left( \frac{z}{2} \right)^{2n} \\
            &= \sqrt{\pi} \Gamma(\nu + 1/2) \left( \frac{z}{2} \right)^{-\nu} \sum_{n = 0}^\infty \frac{(-1)^n}{n! \Gamma(\nu + n + 1)} \left( \frac{z}{2} \right)^{2 n + \nu} \\
            &= \sqrt{\pi} \Gamma(\nu + 1/2) (z/2)^{- \nu} J_\nu(z).
        \end{align*}
    \end{solution}

    \question[P2 de 2011, E de 2011] Seja $J_\nu(x)$ as fun\c{c}\~{o}es de Bessel de primeira esp\'{e}cie e ordem $n$ ($n = 0, 1, 2, \ldots$). Mostre que
    \begin{align*}
        \sum_{n = 0}^\infty \frac{t^n}{n!} J_n(x) = J_0(\sqrt{x^2 - 2xt}).
    \end{align*}
    \begin{solution}
        Temos que
        \begin{align*}
            J_0(\sqrt{x^2 - 2 x t}) &= \sum_{m = 0}^\infty \frac{(-1)^m \left( 2^{-1} \sqrt{x^2 - 2 x t} \right)^{2m}}{(m!)^2} && \text{por \eqref{eq:bessel_pri_esp}} \\
            &= \sum_{m = 0}^\infty \frac{(-1)^m}{2^{2m} (m!)^2} (x^2 - 2x t)^m \\
            &= \sum_{m = 0}^\infty \frac{(-1)^m}{2^{2m} (m!)^2} \sum_{n = 0}^m \binom{m}{n} (x^2)^{m - n} (-2 x t)^n && \text{expans\~{a}o binomial} \\
            &= \sum_{m = 0}^\infty \frac{(-1)^m}{2^{2m} (m!)^2} \sum_{n = 0}^m \frac{m!}{n! (m - n)!} (x^2)^{m - n} (-2 x t)^n \\
            &= \sum_{m = 0}^\infty \frac{(-1)^m}{2^{2m} (m!)^2} \sum_{n = 0}^m \frac{m!}{n! (m - n)!} x^{2m - 2n} (-1)^n 2^n x^n t^n \\
            &= \sum_{m = 0}^\infty \sum_{n = 0}^m \frac{(-1)^m}{2^{2m} (m!)^2} \frac{m!}{n! (m - n)!} x^{2m - 2n} (-1)^n 2^n x^n t^n \\
            &= \sum_{m = 0}^\infty \sum_{n = 0}^m \frac{(-1)^m m! 2^n x^{2m - 2n} (-1)^n x^n t^n}{2^{2m} (m!)^2 n! (m - n)!} \\
            &= \sum_{m = 0}^\infty \sum_{n = 0}^m \frac{(-1)^{m + n} x^{2m - n} t^n}{2^{2m - n} m! n! (m - n)!} \\
            &= \sum_{n = 0}^\infty \sum_{m = n}^\infty \frac{(-1)^{m + n} x^{2m - n} t^n}{2^{2m - n} m! n! (m - n)!} \\
            &= \sum_{n = 0}^\infty \sum_{k = 0}^\infty \frac{(-1)^{k + 2n} x^{2(n + k) - n} t^n}{2^{2(n + k) - n} (n + k)! n! k!} && m - n = k \\
            &= \sum_{n = 0}^\infty \frac{t^n}{n!} \sum_{k = 0}^\infty \frac{(-1)^{k} x^{2k + n}}{2^{2k + n} (n + k)! k!} \\
            &= \sum_{n = 0}^\infty \frac{t^n}{n!} J_n(x) && \text{por \eqref{eq:bessel_pri_esp}}.
        \end{align*}
    \end{solution}
\end{questions}
\bibliographystyle{plain}
\bibliography{bibliography}
\end{document}
