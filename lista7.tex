% Filename: lista7.tex
% 
% This code is part of 'Solutions for MS550, M\'{e}todos de Matem\'{a}tica Aplicada I, and F520, M\'{e}todos Matem\'{a}ticos da F\'{i}sica I'
% 
% Description: This file corresponds to the solutions of homework sheet 7.
% 
% Created: 30.05.12 09:39:27 AM
% Last Change: 19.06.12 08:51:34 AM
% 
% Authors:
% - Raniere Silva (2012): initial version
% 
% Copyright (c) 2012 Raniere Silva <r.gaia.cs@gmail.com>
% 
% This work is licensed under the Creative Commons Attribution-ShareAlike 3.0 Unported License. To view a copy of this license, visit http://creativecommons.org/licenses/by-sa/3.0/ or send a letter to Creative Commons, 444 Castro Street, Suite 900, Mountain View, California, 94041, USA.
%
% This work is distributed in the hope that it will be useful, but WITHOUT ANY WARRANTY; without even the implied warranty of MERCHANTABILITY or FITNESS FOR A PARTICULAR PURPOSE.
%
\documentclass[a4paper,12pt, leqno, answers]{exam}
% Customiza\c{c}\~{a}o da classe exam
\newcommand{\mycheader}{Lista 7 - Fun\c{c}\~{a}o de Legendre}
\header{MS550, F520}{\mycheader}{\thepage/\numpages}
\headrule
\footer{Dispon\'{i}vel em \\% Filename: repository.tex
% 
% This code is part of 'Solutions for MS550, M\'{e}todos de Matem\'{a}tica Aplicada I, and F520, M\'{e}todos Matem\'{a}ticos da F\'{i}sica I'
% 
% Description: This file keeps the url of the repository.
% 
% Created: 07.03.12 04:00:00 PM
% Last Change: 30.05.12 04:40:25 PM
% 
% Authors:
% - Raniere Silva (2012): initial version
% 
% Copyright (c) 2012 Raniere Silva <r.gaia.cs@gmail.com>
% 
% This work is licensed under the Creative Commons Attribution-ShareAlike 3.0 Unported License. To view a copy of this license, visit http://creativecommons.org/licenses/by-sa/3.0/ or send a letter to Creative Commons, 444 Castro Street, Suite 900, Mountain View, California, 94041, USA.
%
% This work is distributed in the hope that it will be useful, but WITHOUT ANY WARRANTY; without even the implied warranty of MERCHANTABILITY or FITNESS FOR A PARTICULAR PURPOSE.
%
\url{https://github.com/r-gaia-cs/solucoes_listas_metodos}
}{}{Reportar erros para \\\href{mailto:r.gaia.cs@gmail.com}{r.gaia.cs@gmail.com}
}
\footrule 
\pagestyle{headandfoot}
\renewcommand{\solutiontitle}{\noindent\textbf{Solu\c{c}\~{a}o:}\enspace}
\SolutionEmphasis{\slshape}
\unframedsolutions
\pointname{}

% Filename: paper_size.tex
%
% This code is part of 'Solutions for MS550, M\'{e}todos de Matem\'{a}tica Aplicada I, and F520, M\'{e}todos Matem\'{a}ticos da F\'{i}sica I'
% 
% Description: This file corresponds to the paper size output.
% 
% Created: 07.03.12 04:00:00 PM
% Last Change: 30.05.12 04:40:25 PM
% 
% Authors:
% - Raniere Silva (2012): initial version
% 
% Copyright (c) 2012 Raniere Silva <r.gaia.cs@gmail.com>
% 
% This work is licensed under the Creative Commons Attribution-ShareAlike 3.0 Unported License. To view a copy of this license, visit http://creativecommons.org/licenses/by-sa/3.0/ or send a letter to Creative Commons, 444 Castro Street, Suite 900, Mountain View, California, 94041, USA.
%
% This work is distributed in the hope that it will be useful, but WITHOUT ANY WARRANTY; without even the implied warranty of MERCHANTABILITY or FITNESS FOR A PARTICULAR PURPOSE.
%
% Para impress\~{a}o
\usepackage[top=3cm, bottom=3cm, left=2cm, right=2cm]{geometry}

% Para ereaders (Kindle, Nook, Kobo, ...) and tablets (iPad, GalaxyTab, ...)
% \usepackage[papersize={180mm,240mm},margin=2mm]{geometry}
% \sloppy


% Pacotes
\usepackage[utf8]{inputenc}
\usepackage[T1]{fontenc}
\usepackage[brazil]{babel}
\usepackage{amsmath}
\usepackage{amsfonts}
\usepackage{amssymb}
\usepackage{hyperref}
\usepackage{graphicx}
\usepackage{tikz}

% Customiza\c{c}\~{a}o do pacote amsmath
\allowdisplaybreaks[4]

% Novos ambientes
% \newenvironment{fwsolution}{\begin{EnvFullwidth}\begin{TheSolution}}{\end{TheSolution}\end{EnvFullwidth}}

% Novos comandos
\newcommand{\devp}[2]{\frac{\partial #1}{\partial #2}}
\newcommand{\grad}{\mbox{grad }}
\newcommand{\diver}{\mbox{div }}
\newcommand{\rot}{\mbox{rot }}

\begin{document}
%cover
\thispagestyle{empty}
% Filename: cover.tex
% This code is part of 'Solutions for MS550, M\'{e}todos de Matem\'{a}tica Aplicada I, and F520, M\'{e}todos Matem\'{a}ticos da F\'{i}sica I'
% 
% Description: This file corresponds to the cover.
% 
% Created: 30.05.12 04:40:25 PM
% Last Change: 31.05.12 10:11:55 PM
% 
% Authors:
% - Raniere Silva (2012): initial version
% 
% Copyright (c) 2012 Raniere Silva <r.gaia.cs@gmail.com>
% 
% Permission is granted to copy, distribute and/or modify this document under the terms of the GNU Free Documentation License, Version 1.3 or any later version published by the Free Software Foundation; with no Invariant Sections, no Front-Cover Texts, and no Back-Cover Texts.
% This document is distributed in the hope that it will be useful, but WITHOUT ANY WARRANTY; without even the implied warranty of MERCHANTABILITY or FITNESS FOR A PARTICULAR PURPOSE.
% More details at <http://www.gnu.org/licenses/>
%
\begin{center}
    \LARGE{Solu\c{c}\~{o}es para MS550, M\'{e}todos de Matem\'{a}tica Aplicada I, e F520, M\'{e}todos Matem\'{a}ticos da F\'{i}sica I}
    
    \Large{\mycheader}
\end{center}
\vspace{.5\textheight}

\begin{tabular}{|p{.9\textwidth}|}
\hline
\'{E} garantida a permiss\~{a}o para copiar, distribuir e/ou modificar este documento sob os termos da Licen\c{c}a de Documenta\c{c}\~{a}o Livre GNU (GNU Free Documentation License), Vers\~{a}o 1.2 ou qualquer vers\~{a}o posterior publicada pela Free Software Foundation; sem Se\c{c}\~{o}es Invariantes, Textos de Capa Frontal, e sem Textos de Quarta Capa.

Este documento \'{e} distribuido na esperança que possa ser \'{u}til, mas SEM NENHUMA GARANTIA; sem uma garantia implicita de ADEQUA\c{C}\~{A}O a qualquer MERCADO ou APLICA\c{C}\~{A}O EM PARTICULAR.

Mais detalhes em \url{http://www.gnu.org/licenses/}.
\\ \hline
\end{tabular}

\newpage
\setcounter{page}{1}
Equa\c{c}\~{o}es eventualmente útil:
\begin{align}
    & f(x) = \sum_{n = 0}^\infty \frac{f^{(n)}(a)}{n!} (x - a)^n \tag{ST} \label{eq:ser_taylor} \\
    & \Gamma(z) = \int_0^\infty e^{-t} t^{z - 1} dt \tag{GE} \label{eq:gamma_euler} \\
    & \Gamma(z + 1) = z \Gamma(z), \ \Gamma(z) \Gamma(1 - z) = \pi / \sin(\pi z) \label{eq:gamma_rel} \\
    & 2^{2 z - 1} \Gamma(z) \Gamma(z + 1/2) = \sqrt{\pi} \Gamma(2 z) \label{eq:gamma_dup_legendre} \\
    & B(z, w) = \frac{\Gamma(z) \Gamma(w)}{\Gamma(z + w)} \tag{BG} \label{eq:beta} \\
    & B(z, w) = 2 \int_0^{\pi / 2} \cos^{2z - 1} \theta \sin^{2w - 1} \theta d\theta \tag{BT} \label{eq:beta_trig} \\
    & B(z, w) = \int_0^1 t^{z - 1} (1 - t)^{w - 1} dt \tag{BI} \label{eq:beta_int} \\
    & (\alpha)_n = \alpha (\alpha + 1) \ldots (\alpha + n - 1), \ (\alpha)_0 = 1 \tag{SP} \label{eq:sim_poch} \\
    & (\alpha)_n = \frac{\Gamma(\alpha + n)}{\Gamma(\alpha)} \label{eq:sim_poch_gamma} \\
    & \frac{(\alpha)_n}{m!} = \binom{\alpha + n - 1}{n}, \ \frac{(-\alpha)_n}{n!} = (-1)^n \binom{\alpha}{n} \label{eq:sim_poch_binom} \\
    & z(1 - z)y'' + \left[ \gamma - (\alpha + \beta + 1) z \right] y' - \alpha \beta y = 0 \tag{EH} \label{eq:hiperg} \\
    & {}_2F_1(\alpha, \beta, \gamma; z) = \sum_{n = 0}^\infty \frac{(\alpha)_n (\beta)_n}{(\gamma)_n} \frac{z^n}{n!} \tag{SH} \label{eq:ser_hiperg} \\
    & {}_2F_1(\alpha, \beta, \gamma; z) = \frac{1}{B(\beta, \gamma - \beta)} \int_0^1 t^{\beta - 1} (1- t)^{\gamma - \beta - 1} (1 - tz)^{-\alpha} dt \label{eq:hiperg_int} \\
    % & {}_2F_1(\alpha, \beta, \gamma; z) = \frac{\alpha \beta}{\gamma} \,{}_2F_1(\alpha + 1, \beta + 1, \gamma + 1; z) \label{eq:hiperg_der} \\
    % & zy'' + (\gamma - z)y' - \alpha y = 0 \tag{EHC} \label{eq:hiper_con} \\
    % & {}_1F_1(\alpha, \gamma; z) = \sum_{n = 0}^\infty \frac{(\alpha)_n}{(\gamma)_n} \frac{z^n}{n!} \tag{SHC} \label{eq:ser_hiperg_con}
    % & x^2 y'' + x y' + (x^2 - \nu^2) y = 0 \tag{EB} \label{eq:bessel} \\
    % & J_\nu(x) = \sum_{k = 0}^\infty \frac{(-1)^k}{k! \Gamma(k + \nu + 1)} \left( \frac{x}{2} \right)^{2k + \nu} \label{eq:bessel_pri_esp} \\
    % & J_\nu(x) = \frac{x^\nu \exp(-i x)}{2^\nu \Gamma(\nu + 1)} \,_1F_1(\nu + 1/2, 2\nu + 1; 2 i x) \label{eq:bessel_pri_esp_hiperg_con} \\
    % & J_{-\nu}(x) = (-1)^\nu J_\nu(x) \label{eq:bessel_prim_esp_neg} \\
    % & \exp(x (t - t^{-1}) / 2) = \sum_{k = -\infty}^{+\infty} t^k J_k(x) \tag{GFB} \label{eq:bessel_pri_esp_geratriz} \\
    % & x^2 y'' + x y' - (x^2 + v^2) y = 0 \tag{EBM} \label{eq:bessel_mod} \\
    % & I_\nu(x) = i^{-\nu} J_\nu(x) \label{eq:bessel_mod_pri_esp} \\
    & (1 - x^2) y'' - 2 x y' + \nu (\nu + 1) y = 0 \tag{EL} \label{eq:legendre} \\
    & P_\nu(x) = \,_2F_1(- \nu, \nu + 1, 1; (1 - x)/2) \tag{FL} \label{eq:legendre_funcao_pri_esp} \\
    & g(-x, t) = (1 - 2 x t + t^2)^{-1/2} \tag{FGL} \label{eq:legendre_funcao_geratriz} \\
    & g(-x, t) = \sum_{n = 0}^\infty P_n(x) t^n \tag{FGLS} \label{eq:legendre_funcao_geratriz_serie} \\
    & \int_{-1}^1 P_n(x) P_m(x) \,\mathrm{d}x = 2 \left( 2n + 1 \right)^{-1} \delta_{mn} \label{eq:legendre_ortog} \\
    & Q_n(x) = \frac{1}{2^{n + 1}} \int_{-1}^1 \frac{(1 - y^2)^n}{(x - y)^{n + 1}} \,\mathrm{d}y \label{eq:legendre_funcao_seg_esp_integral} \\
    & (1 - x^2) y'' - 2 x y' + \left( \nu (\nu + 1) - m^2 (1 - x^2)^{-1} \right) y = 0 \tag{ELA} \label{eq:legendre_ass} \\
    & P_\nu^m(x) = (1 - x^2)^{m/2} \frac{\mathrm{d}^m}{\mathrm{d}x^m}P_\nu(x) \tag{FLA} \label{eq:legendre_ass_ funcao_pri_esp} \\
    & \sum_{n = 0}^\infty P_{n + m}(x) t^n = \frac{(2m)!}{2^n m!} \frac{(1 - x^2)^{m/2}}{(1 - 2 x t + t^2)^{m + 1/2}} \label{eq:legendre_ass_funcao_geratriz} \\
    & Q_\nu^m(x) = (1 - x^2)^{m/2} \frac{\mathrm{d}^m}{\mathrm{d}x^m}Q_\nu(x) \label{eq:legendre_ass_funcao_seg_esp} \\
    & F_{n}(x) = \frac{1}{\rho(x)} \frac{\mathrm{d}^n}{\mathrm{d} x^n}\left[ \rho(x) s^n(x) \right] \tag{FR} \label{eq:form_rodrigues}
\end{align}

\begin{questions}
    \question Mostre as rela\c{c}\~{o}es de recorr\^{e}ncia que seguem derivando a fun\c{c}\~{a}o geratriz.
    \begin{solution}
        Seja $g(x, t)$ a fun\c{c}\~{a}o geratriz de Legendre. Para $g(1, t)$ temos
        \begin{align*}
            g(1, t) &= \left( 1 - 2 t + t^2 \right)^{-1/2} && \text{por \eqref{eq:legendre_funcao_geratriz}} \\
            &= \left[ \left( 1 - t \right)^2 \right]^{-1/2} \\
            &= \left( 1 - t \right)^{-1} \\
            &= \sum_{n = 0}^\infty t^n && \text{pela expans\~{a}o em Taylor}, \\
            g(1, t) &= \sum_{n = 0}^\infty P_n(1) t^n && \text{por \eqref{eq:legendre_funcao_geratriz_serie},}
        \end{align*}
        de onde segue que
        \begin{align}
            P_n(1) = 1. \label{eq:legendre_rel_rec_n1}
        \end{align}
        Tomando agora $g(-x, t)$ temos
        \begin{align*}
            g(-x, t) &= \left( 1 - 2 (-x) t + t^2 \right)^{-1/2} && \text{por \eqref{eq:legendre_funcao_geratriz}} \\
            &= \left( 1 - 2 x (-t) + (-t)^2 \right)^{-1/2} \\
            &= \sum_{n = 0}^\infty P_n(x) (-1)^n t^n, \\
            g(-x, t) &= \sum_{n = 0}^\infty P_n(-x) t^n && \text{por \eqref{eq:legendre_funcao_geratriz_serie}}
        \end{align*}
        de onde segue que
        \begin{align}
            P_n(-x) = (-1)^n P_n(x). \label{eq:legendre_rel_rec_nx}
        \end{align}
        Utilizando \eqref{eq:legendre_rel_rec_n1} e \eqref{eq:legendre_rel_rec_nx} segue que
        \begin{align*}
            P_n(-1) = (-1)^n.
        \end{align*}

        Derivando $g(x, t)$ em rela\c{c}\~{a}o a $t$ temos
        \begin{align*}
            \frac{\partial g(x,t)}{\partial t} &= \frac{\partial}{\partial t}\left( 1 - 2 x t + t^2 \right)^{-1/2} && \text{por \eqref{eq:legendre_funcao_geratriz}} \\
            &= \left( - 2 x + 2 t \right) (-1/2) \left( 1 - 2 x t + t^2 \right)^{-3/2} \\
            &= \left( x - t \right) \left( 1 - 2 x t + t^2 \right)^{-1/2} \left( 1 - 2 x t + t^2 \right)^{-1} \\
            &= \left( x - t \right) g(x, t) \left( 1 - 2 x t + t^2 \right)^{-1}, \\
            \left( 1 - 2 x t + t^2 \right) \frac{\partial g(x, t)}{\partial t} &= (x - t) g(x, t).
        \end{align*}
        Usando \eqref{eq:legendre_funcao_geratriz_serie} na equa\c{c}\~{a}o anterior temos
        \begin{align*}
            \left( 1 - 2 x t + t^2 \right) \frac{\partial}{\partial t}\left[ \sum_{n = 0}^\infty P_n(x) t^n \right] &= (x - t) \left[ \sum_{n = 0}^\infty P_n(x) t^n \right] \\
            \left( 1 - 2 x t + t^2 \right) \left[ \sum_{n = 0}^\infty P_n(x) n t^{n - 1} \right] &= (x - t) \left[ \sum_{n = 0}^\infty P_n(x) t^n \right] \\
            \sum_{n = 0}^\infty P_n(x) n \left( t^{n - 1} - 2 x t^{n} + t^{n + 1} \right) &= \sum_{n = 0}^\infty P_n(x) (x t^n - t^{n + 1}).
        \end{align*}
        Agrupando as pot\^{e}ncias de $t$ segue que
        \begin{align}
            (2n + 1) x P_n(x) = (n + 1) P_{n + 1}(x) + n P_{n - 1}(x). \label{eq:legendre_rel_rec_2n+1}
        \end{align}
        Derivando $g(x, t)$ em rela\c{c}\~{a}o a $x$ temos
        \begin{align*}
            \frac{\partial g(x,t)}{\partial x} &= \frac{\partial}{\partial x}\left( 1 - 2 x t + t^2 \right)^{-1/2} && \text{por \eqref{eq:legendre_funcao_geratriz}} \\
            &= \left( - 2 t \right) (-1/2) \left( 1 - 2 x t + t^2 \right)^{-3/2} \\
            &= t \left( 1 - 2 x t + t^2 \right)^{-1/2} \left( 1 - 2 x t + t^2 \right)^{-1} \\
            &= t g(x, t) \left( 1 - 2 x t + t^2 \right)^{-1}, \\
            \left( 1 - 2 x t + t^2 \right) \frac{\partial g(x, t)}{\partial t} &= -t g(x, t).
        \end{align*}
        Usando \eqref{eq:legendre_funcao_geratriz_serie} na equa\c{c}\~{a}o anterior temos
        \begin{align*}
            \left( 1 - 2 x t + t^2 \right) \frac{\partial}{\partial x}\left[ \sum_{n = 0}^\infty P_n(x) t^n \right] &= t \left[ \sum_{n = 0}^\infty P_n(x) t^n \right] \\
            \left( 1 - 2 x t + t^2 \right) \left[ \sum_{n = 0}^\infty P'_n(x) t^{n} \right] &= t \left[ \sum_{n = 0}^\infty P_n(x) t^n \right] \\
            \sum_{n = 0}^\infty P'_n(x) \left( t^{n} - 2 x t^{n + 1} + t^{n + 2} \right) &= \sum_{n = 0}^\infty P_n(x) t^{n + 1}.
        \end{align*}
        Agrupando as pot\^{e}ncias de $t$ segue que
        \begin{align*}
            P_n(x) = P'_{n + 1}(x) - 2 x P'_n(x) + P'_{n - 1}(x). %  \label{eq:legendre_rel_rec_2der}
        \end{align*}
        Derivando \eqref{eq:legendre_rel_rec_2n+1} temos
        \begin{align*}
            (2n + 1) (P_n(x) + x P'_n(x)) = (n + 1) P'_{n + 1}(x) + n P'_{n - 1}(x)
        \end{align*}
        e usando o resultado na express\~{a}o anterior segue que
        \begin{align}
            (2n + 1)P_n(x) = P'_{n + 1}(x) - P'_{n - 1}(x). \label{eq:legendre_rel_rec_2n+1der}
        \end{align}

        TODO
        \begin{align}
            P'_{n - 1}(x) = -n P_n(x) + x P_n'(x). \label{eq:legendre_rel_rec_devn-1}
        \end{align}

        Substituindo \eqref{eq:legendre_rel_rec_devn-1} em \eqref{eq:legendre_rel_rec_2n+1der} segue que
        \begin{align*}
            (2n + 1) P_n(x) &= P'_{n + 1}(x) - \left[ -n P_n(x) + x P'_n(x) \right] \\
            &= P'_{n + 1}(x) + n P_n(x) - x P'_n(x), \\
            (n + 1) P_n(x) &= P'_{n + 1}(x) - x P'_n(x).
        \end{align*}

        Tomando $n \to n - 1$ na equa\c{c}\~{a}o anterior temos
        \begin{align*}
            n P_{n - 1}(x) &= P'_{n}(x) - x P'_{n - 1}(x)
        \end{align*}
        e somando com \eqref{eq:legendre_rel_rec_devn-1} multiplicado por $(-x)$ segue que
        \begin{align*}
            n P_{n - 1}(x) - x P'_{n - 1} &= P'_n(x) - x P'_{n - 1}(x) - \left[ -n x P_n(x) + x^2 P'_n(x) \right], \\
            n P_{n - 1}(x) &= P'_n(x) + n x P_n(x) - x^2 P'_n(x), \\
            n P_{n - 1}(x) - n x P_n(x) &= (1 - x^2) P'_n(x).
        \end{align*}

        Outras rela\c{c}\~{o}es:
        \begin{align*}
            & (1 - x^2) P'_n(x) = (n + 1) x P_n(x) - (n + 1) P_{n + 1}(x), \\
            & P'_{n + 1}(x) = (n + 1) P_n(x) + x P'_n(x).
        \end{align*}
    \end{solution}

    \question[Exerc\'{i}cio 12.1.7 do Arfken\nocite{Arfken:2005:Mathematical}] Mostre que
    \begin{align*}
        P_n(\cos \theta) = (-1)^n \frac{r^{n + 1}}{n!} \frac{\partial^n}{\partial z^n} \left( \frac{1}{r} \right).
    \end{align*}
    \begin{solution}
        Considere o diagrama abaixo \\ \hspace*{.2\textwidth}
        \begin{tikzpicture}
            \draw[->] (-.2,0) -- (6,0) node[below right]{$z$};
            \draw[->] (0,-.2) -- (0,3);
            \draw (0,0) -- (25:5) node[midway, above left]{$\vec{r_1}(x, y, z)$} -- (3,0) node[midway, below right]{$\vec{r_2}(x, y, z - \Delta z)$};
            \draw (0,0) -- (3,0) node[midway, below]{$\vec{r_3}(x, y, z)$};
            \draw (0:1) arc (0:25:1) node[below right]{$\theta$};
        \end{tikzpicture} \\
        Ent\~{a}o
        \begin{align*}
            |\vec{r_1} - \vec{r_3}| &= \sqrt{r_1^2 + r_3^2 - 2 r_1 r_3 \cos(\theta)}, \\
            |\vec{r_1} - \vec{r_3}|^{-1} &= (r_1^2 + r_2^3 - 2 r_1 r_3 \cos(\theta))^{-1/2} \\
            &= r_1^{-1} \sum_{n = 0}^\infty P_n(\cos(\theta)) (r_3 r_1^{-1})^n && \text{por \eqref{eq:legendre_funcao_geratriz} e \eqref{eq:legendre_funcao_geratriz_serie}} \\
            &= \sum_{n = 0}^\infty P_n(\cos(\theta)) (\Delta z^n (-1)^n) r^{-n + 1}.
        \end{align*}
        Por outro lado
        \begin{align*}
            \left[ r (z - \Delta z) \right]^{-1} &= \sum_{n = 0}^\infty \frac{1}{n!} \frac{\mathrm{d}^n}{\mathrm{d}z^n}\left( \frac{1}{r} \right) (\Delta z)^n.
        \end{align*}
        Portanto,
        \begin{align*}
            P_n(\cos(\theta)) = \frac{(-1)^n r^{n + 1}}{n!} \frac{\mathrm{d}}{\mathrm{d}z^n}\left( \frac{1}{r} \right).
        \end{align*}
    \end{solution}

    \question Mostre, usando explicitamente coordenadas esf\'{e}ricas $(r, \theta, \phi)$, que
    \begin{align*}
        \frac{\partial}{\partial z} \left[ \frac{P_n(\cos \theta)}{r^{n + 1}} \right] &= - (n + 1) \frac{P_{n + 1}(\cos \theta)}{r^{n + 2}}.
    \end{align*}
    \begin{solution}
        Como
        \begin{align*}
            P_n(\cos \theta) &= (-1)^n \frac{r^{n + 1}}{n!} \frac{\partial^n}{\partial z^n}\left( \frac{1}{r} \right)
        \end{align*}
        j\'{a} est\'{a} em coordenadas esf\'{e}ricas, temos:
        \begin{align*}
            \frac{\partial}{\partial z}\left[ \frac{P_n(\cos \theta)}{r^{n + 1}} \right] &= \frac{\partial}{\partial z}\left[ \frac{(-1)^n r^{n + 1}}{n! r^{n + 1}} \frac{\partial^n}{\partial z^n}\left( \frac{1}{r} \right) \right] \\
            &= \frac{\partial}{\partial z}\left[ \frac{(-1)^n}{n!} \frac{\partial^n}{\partial z^n}\left( \frac{1}{r} \right) \right] \\
            &= \frac{(-1)^n}{n!} \frac{\partial^{n + 1}}{\partial z^{n + 1}}\left( \frac{1}{r} \right). 
        \end{align*}
        Utilizando a express\~{a}o para $P_{n + 1}(\cos \theta)$ de um exerc\'{i}cio anterior temos
        \begin{align*}
            P_{n + 1}(\cos \theta) &= \frac{r^{n + 2} (-1)^{n + 1}}{(n + 1)!} \frac{\partial^{n + 1}}{\partial z^{n + 1}}\left( \frac{1}{r} \right), \\
            \frac{\partial^{n + 1}}{\partial z^{n + 1}}\left( \frac{1}{r} \right) &= \frac{(n + 1)! P_{n + 1}(\cos \theta)}{(-1)^{n + 1} r^{n + 2}}.
        \end{align*}
        E substituindo na express\~{a}o anterior,
        \begin{align*}
            \frac{\partial}{\partial z}\left[ \frac{P_n(\cos \theta)}{r^{n + 1}} \right] &= \frac{(-1)}{n!} \frac{(n + 1)!}{(-1)^{n + 1}} \frac{P_{n + 1}(\cos \theta)}{r^{n + 2}} \\
            &= \frac{-(n + 1) P_{n + 1}(\cos \theta)}{r^{n + 2}}.
        \end{align*}
    \end{solution}

    \question Mostre que
    \begin{parts}
        \part[T6 de 2010] $\int_{-1}^1 P_n(x) \,\mathrm{d}x = 0, n \geq 1$,
        \begin{solution}
            Utilizando \eqref{eq:legendre_ortog} e fazendo $P_m(x) = 1 = P_0(x)$ temos
            \begin{align*}
                \int_{-1}^1 P_n(x) \,\mathrm{d}x = \frac{2}{2 n + 1} \delta_{0n} = \begin{cases}
                    2, & \text{se $n = 0$,} \\
                    0, & \text{se $n \geq 1$.}
                \end{cases}
            \end{align*}
        \end{solution}

        \part $\int_{-1}^1 x^{2n + 1} P_{2m}(x) \,\mathrm{d}x = 0, m \neq n$,
        \begin{solution}
            Sabemos que $x^{2n + 1}$ \'{e} uma fun\c{c}\~{a}o \'{i}mpar, $(-x)^{2n + 1} = (-1) x^{2n + 1}$, e que $P_{2m}(x)$ \'{e} uma fun\c{c}\~{a}o par, $P_{2m}(-x) = P_{2m}(x)$. Logo, $f(x) = x^{2n + 1} P_{2m}(x)$ \'{e} uma fun\c{c}\~{a}o \'{i}mpar, $m \neq n$, e portanto
            \begin{align*}
                \int_{-1}^1 f(x) \,\mathrm{d}x &= 0, m \neq n.
            \end{align*}
        \end{solution}

        \part $\int_{-1}^1 x^m P_n(x) \,\mathrm{d}x = 0, m < n$,
        \begin{solution}
            Temos que
            \begin{align*}
                \int_{-1}^1 x^{m} P_n(x) \,\mathrm{d}x &= \int_{-1}^1 \sum_{k = 0}^m a_k P_k(x) P_n(x) \,\mathrm{d}x \\
                &= \sum_{k = 0}^m a_k \int_{-1}^1 P_k(x) P_n(x) \,\mathrm{d}x \\
                &= \sum_{k = 0}^m a_k \left[ \frac{2}{2n + 1} \delta_{mn} \right] && \text{por \eqref{eq:legendre_ortog}} \\
                &=\sum_{k = 0}^m a_k \left[ 0 \right] \\
                &= 0.
            \end{align*}
        \end{solution}

        \part $\int_{-1}^1 x^n P_n(x) \,\mathrm{d}x = \left[ 2^{n + 1} (n!)^2 \right] / (2n + 1)!$.
        \begin{solution}
            Por \eqref{eq:form_rodrigues} temos que
            \begin{align*}
                P_n(x) &= \frac{1}{2^n n!} \frac{\mathrm{d}^n}{\mathrm{d} x^n}\left[ (x^2 - 1)^n \right]
            \end{align*}
            e portanto o $n$-\'{e}simo termo de $P_n$ corresponde a
            \begin{align*}
                \frac{\mathrm{d}^n}{\mathrm{d}x^n}(x^{2n}) &= (2n) (2n - 1) \ldots (2n - n + 1) x^n \\
                &= (2n) \ldots (n + 1) x^n \\
                &= \frac{(2n)!}{n!} x^n.
            \end{align*}
            Utilizando a express\~{a}o acima temos que
            \begin{align*}
                P_n &= \frac{1}{2^n n!} \frac{(2n)!}{n!} x^n, \\
                x^n &= \frac{2^n (n!)^2}{(2n)!} P_n.
            \end{align*}
            Logo,
            \begin{align*}
                \int_{-1}^1 x^n P_n(x) \,\mathrm{d}x &= \int_{-1}^1 \frac{2^n (n!)^2}{(2n)!} P_n P_n \,\mathrm{d}x \\
                &= \frac{2^n (n!)^2}{(2n)!} \int_{-1}^1 P_n(x) P_n(x) \,\mathrm{d}x \\
                &= \frac{2^n (n!)^2}{(2n)!} \frac{2}{2n + 1} \delta_{nm} && \text{por \eqref{eq:legendre_ortog}} \\
                &= \frac{2^n (n!)^2}{(2n)!} \frac{2}{2n + 1} \\
                &= \frac{2^{n + 1} (n!)}{(2n + 1)!}
            \end{align*}
        \end{solution}
    \end{parts}

    \question Mostre que
    \begin{parts}
        \part[P3 de 2006, T6 de 2010, T6 de 2011] $P'_n(1) = n (n + 1) / 2$,
        \begin{solution}
            \textbf{M\'{e}todo 1}

            Pela fun\c{c}\~{a}o geratriz temos que
            \begin{align*}
                \frac{1}{\sqrt{1 - 2 x t + t^2}} &= \sum_{n = 0}^\infty P_n(x) t^n.
            \end{align*}
            Derivando em rela\c{c}\~{a}o a $x$ obtemos
            \begin{align*}
                \frac{t}{\left( 1 - 2 x t + t^2 \right)^{3/2}} &= \sum_{n = 0}^\infty P'_n(x) t^n.
            \end{align*}
            Para $x = 1$ verifica-se que
            \begin{align*}
                \frac{t}{\left( 1 - 2 t + t^2 \right)^{3/2}} &= \frac{1}{\left[ \left( 1 - t \right)^2 \right]^{3/2}} \\
                &= \frac{t}{\left( 1 - t \right)^3} \\
                &= \sum_{n = 0}^\infty P'_n(1) t^n.
            \end{align*}
            Sabendo que
            \begin{align*}
                \left( 1 - t \right)^{-3} &= \sum_{k = 0}^\infty \frac{(3)_k}{k!} t^k \\
                &= \sum_{k = 0}^\infty \frac{(k + 2)!}{2! k!} t^k \\
                &= \sum_{k = 0}^\infty \frac{(k + 2) (k + 1)}{2} t^k
            \end{align*}
            temos ent\~{a}o que
            \begin{align*}
                \frac{t}{\left( 1 - t \right)^3} &= \sum_{k = 0}^\infty \frac{(k + 2) (k + 1)}{2} t^{k + 1} \\
                &= \sum_{n = 1}^\infty \frac{(n + 1) n}{2} t^n \\
                &= \sum_{n = 0}^\infty P'_n(1) t^n.
            \end{align*}
            Concluimos ent\~{a}o que $P'_0(1) = 0$ e $P'_n(1) = n \left( n + 1 \right) / 2$, $n = 1, 2, \ldots$, que equivale a $P'_n(1) = n \left( n + 1 \right) / 2$, $n = 0, 1, 2, \ldots$.

            \textbf{M\'{e}todo 2}

            Temos que
            \begin{align*}
                P_n(x) &= \,_2F_1(-n, n + 1, 1; (1 - x)/2) \\
                P'_n(x) &= \,_2F'_1(-n, n + 1, 1; (1 - x)/2) \left( -1 / 2 \right)\\
                &= \left( -1 / 2 \right) ( -n ) ( n + 1 ) ( 1 )^{-1} \,_2F_1(-n + 1, n + 2m, 2; (1 - x)/2) \\
                P'_n(1) &= n (n + 1) (2)^{-1} \,_2F_1(-n + 1, n + 2, 2; 0) \\
                &= n (n + 1) / 2.
            \end{align*}
        \end{solution}

        \part[E de 2006] $P'_n(-1) = (-1)^{n + 1} n (n + 1) / 2$.
        \begin{solution}
            Temos que
            \begin{align*}
                \frac{1}{\sqrt{1 - 2 x t + t^2}} &= \sum_{n = 0}^\infty P_n(x) t^n,
            \end{align*}
            derivando em rela\c{c}\~{a}o a $x$ obtemos
            \begin{align*}
                \frac{t}{\left[ 1 - 2 x t + t^2 \right]^{3/2}} &= \sum_{n = 0}^\infty P'_n(x) t^n
            \end{align*}
            e fazendo $x = -1$
            \begin{align*}
                \frac{t}{\left[ 1 + 2 t + t^2 \right]^{3/2}} &= \frac{t}{\left[ \left( 1 + t \right)^2 \right]^{3/2}} \\
                &= \frac{t}{(1 + t)^3} \\
                &= \sum_{n = 0}^\infty P'_n(-1) t^n.
            \end{align*}
            Sabendo que
            \begin{align*}
                \left( 1 + t \right)^{-3} &= \sum_{k = 0}^\infty \frac{(3)_k}{k!} (-t)^k \\
                &= \sum_{k = 0}^\infty \frac{(k + 2)!}{2! k!} (-1)^k t^k \\
                &= \sum_{k = 0}^\infty \frac{(k + 2) (k + 1)}{2} (-1)^k t^k
            \end{align*}
            temos ent\~{a}o que
            \begin{align*}
                \frac{t}{\left( 1 - t \right)^3} &= \sum_{k = 0}^\infty \frac{(k + 2) (k + 1)}{2} (-1)^k t^{k + 1} \\
                &= \sum_{n = 1}^\infty \frac{(n + 1) n}{2} (-1)^{n - 1} t^n \\
                &= \sum_{n = 0}^\infty P'_n(-1) t^n.
            \end{align*}
            Concluimos ent\~{a}o que $P'_0(-1) = 0$ e $P'_n(-1) = (-1)^{n - 1} n \left( n + 1 \right) / 2 = (-1)^{n + 1} n (n + 1) / 2$, $n = 1, 2, \ldots$, que equivale a $P'_n(1) = (-1)^{n + 1} n \left( n + 1 \right) / 2$, $n = 0, 1, 2, \ldots$.
        \end{solution}
    \end{parts}

    \question Construa (a menos de uma normaliza\c{c}\~{a}o) os polinômios de Legendre $P_n(x)$ utilizando o processo de ortogonaliza\c{c}\~{a}o de Gram-Schmidt aplicado \`{a} base $\left\{ 1, x, x^2, x^3, \ldots \right\}$ do espa\c{c}o dos polinômios no intervalo $(-1, 1)$ equipado com o produto escalar $(f, g) = \int_{-1}^1 f(x) g(x) \,\mathrm{d} x$.
    \begin{solution}
        Seja $\left\{ v_0, v_1, v_2, \ldots \right\} = \left\{ 1, x, x^2, \ldots \right\}$. Ent\~{a}o pelo processo de ortogonaliza\c{c}\~{a}o de Gram-Schmidt temos que
        \begin{align*}
            u_0 &= v_0 = 1 = P_0(x) \\
            u_1 &= v_1 - \frac{v_1^t u_0}{\| u_0 \||^2} u_0 \\
            &= x - \frac{1 \int_{-1}^1 x \,\mathrm{d}x}{\int_{-1}^1 \,\mathrm{d}x} \\
            &= x - 0 / 2 = x = P_1(x) \\
            u_2 &= v_2 - \frac{v_2^t u_0}{\| u_0 \|^2} u_0 - \frac{v_2^t u_1}{\| u_1 \|^2} u_1 \\
            &= x^2 - \frac{1 \int_{-1}^1 x^2 \,\mathrm{d}x}{\int_{-1}^1 \,\mathrm{d}x} - \frac{x \int_{-1}^1 x^2 \,\mathrm{d}x}{\int_{-1}^1 \,\mathrm{d}x} \\
            &= x^2 - 1/3 = (2/3) P_2(x) \\
            \vdots
        \end{align*}
    \end{solution}

    \question Mostre que as fun\c{c}\~{o}es de Legentre de segunda esp\'{e}cie $Q_n(x)$ satisfazem as seguintes rela\c{c}\~{o}es de recorr\^{e}ncia:
    \begin{parts}
        \part $(2n + 1) x Q_n(x) = (n + 1) Q_{n + 1}(x) + n Q_{n - 1}(x)$,
        \begin{solution}
            A partir de \eqref{eq:legendre_funcao_seg_esp_integral} temos que
            \begin{align*}
                (n + 1) Q_{n + 1} + n Q_{n - 1}(x) &= (n + 1) \frac{1}{2} \int_{-1}^1 \frac{P_{n + 1}(y)}{x - 1y} \,\mathrm{d}y + n \frac{1}{2} \int_{-1}^1 \frac{P_{n - 1}(y)}{x - y} \\
                &= \frac{1}{2} \int_{-1}^1 \frac{(n + 1) P_{n + 1} + n P_{n - 1}}{x - y} \,\mathrm{d}y \\
                &= (2n + 1) x Q_n(x),
            \end{align*}
            onde a \'{u}ltima passagem decorre de $(2n + 1) P_n(x) = (n + 1) P_{n + 1}(x) + n P_{n - 1}(x)$ (ver notas de aula).
        \end{solution}

        \part $(2n + 1) Q_n(x) = Q'_{n + 1}(x) - Q'_{n - 1}(x)$.
        \begin{solution}
            A partir de \eqref{eq:legendre_funcao_seg_esp_integral} temos que
            \begin{align*}
                Q'_{n + 1}(x) - Q'_{n - 1}(x) &= \frac{1}{2} \int_{-1}^1 \frac{P'_{n + 1}(y)}{x - y} \,\mathrm{d}y - \frac{1}{2} \int_{-1}^1 \frac{P'_{n - 1}(y)}{x - y} \,\mathrm{d}y \\
                &= \frac{1}{2} \int_{-1}^1 \frac{P1_{n + 1} - P'_{n - 1}}{x - y} \,\mathrm{d}y \\
                &= \frac{1}{2} \int_{-1}^1 \frac{(2n + 1) P_n(y)}{x - y} \,\mathrm{d}y \\
                &= (2n + 1) Q_n(x).
            \end{align*}
        \end{solution}
    \end{parts}

    \question Mostre que $Q_n(-x) = (-1)^{n + 1} Q_n(x)$.
    \begin{solution}
        Temos que
        \begin{align*}
            Q_n(-x) &= \frac{1}{2} \int_{-1}^1 \frac{P_n(y)}{-x - y} \,\mathrm{d}y \\
            &= \frac{1}{2} \int_{-1}^1 \frac{P_n(-u)}{- (x - u)} \,\mathrm{d}u && u = -y \\
            &= \frac{1}{2} \int_{-1}^1 \frac{(-1)^n P_n(u)}{- (x - u)} \,\mathrm{d}u \\
            &= (-1)^{n + 1} Q_n(x).
        \end{align*}
    \end{solution}

    \question Mostre que
    \begin{align*}
        n \left[ P_n(x) Q_{n - 1}(x) - Q_n(x) Q_{n - 1}(x) \right] = P_1(x) Q_0(x) - P_0(x) Q_1(x) = 1.
    \end{align*}
    \begin{solution}
        Temos que
        \begin{align*}
            I &= n \left[ P_n(x) Q_{n - 1}(x) - Q_n(x) P_{n - 1}(x) \right] \\
            &= P_1(x) Q_0(x) - Q_1(x) P_0(x) && n = 1 \\
            &= P_1(x) \frac{1}{2} \int_{-1}^1 \frac{P_0(y)}{x - y} \,\mathrm{d}y - \frac{1}{2} \int_{-1}^1 \frac{P_1(y)}{x - y} \,\mathrm{d}y P_0(x) \\
            &= \frac{x}{2} \int_{-1}^1 \frac{1}{x - y} \,\mathrm{d}y - \frac{1}{2} \int_{-1}^1 \frac{y}{x - y} \,\mathrm{d}y \\
            &= \frac{1}{2} \left[ x \left( \left. -\log(x - y) \right|_{y = -1}^1 \right) - \left. \left( x \log(y - x) - y \right) \right|_{y = -11}^1 \right] \\
            &= 2^{-1} \left[ -x \log(x - 1) + x \log(x + 1) - x \log(1 - x) + 1 + x \log(-1 - x) + 1 \right] \\
            &= 1.
        \end{align*}
    \end{solution}

    \question Mostre que os polinômios de Legendre associados $P_n^m(x)$ satisfazem as seguintes rela\c{c}\~{o}es de recorr\^{e}ncia:
    \begin{parts}
        \part $P_n^{m + 1}(x) - 2 m x \left( 1 - x^2 \right)^{-1/2} P_n^m(x) + \left[ n (n _ 1) - m (m - 1) \right] P_n^{m - 1}(x) = 0$,
        \begin{solution}
            Sabemos que os polin\^{o}mios de Legendre associados $P_n^m(x)$ satisfazem \eqref{eq:legendre_ass} e portanto
            \begin{align*}
                0 &= (1 - x^2) \frac{\mathrm{d}^2}{\mathrm{d}x^2}P_n^m(x) - 2 x \frac{\mathrm{d}}{\mathrm{d}x}P_n^m(x) + \left( n (n + 1) - \frac{m^2}{1 - x^2} \right) P_n^m(x) \\
                \begin{split}
                    &= (1 - x^2) \frac{\mathrm{d}^2}{\mathrm{d}x^2}\left[ (1 - x^2)^{m/2} \frac{\mathrm{d}^m}{\mathrm{d}x^m}P_n(x) \right] - 2 x \frac{\mathrm{d}}{\mathrm{d}x}P_n^m(x) \\ &\quad {}+ \left( n (n + 1) - \frac{m^2}{1 - x^2} \right) P_n^m(x)
                \end{split} \\
                % \begin{split}
                %     &= (1 - x^2)
                % \end{split}
                \begin{split}
                    &= -m P_n^m + m (m - 2) x^2 (1 - x^2)^{-1} P_n^m - 2 m x (1 - x^2)^{-1/2} P_n^{m + 1} + \\ &\quad {}+ P_n^{m + 2} + 2 x^2 m (1 - x^2)^{-1} P_n^m - 2 x (1 - x^2)^{-1/2} P_n^{m + 1} \\ &\quad {} + \left( n + (n + 1) - m^2 (1 - x^2)^{-1} \right) P_n^m
                \end{split} \\
                \begin{split}
                    &= P_n^{m + 2} + \left[ -2 x (m + 1) (1 - x^2)^{-1/2} \right] P_n^{m + 1} \\ &\quad {}+ \left[ -m (1 - x^2) + m (m + 2) x^2 + 2 x^2 m + n (n + 1) (1 - x^2) - m^2 \right] (1 - x)^{-1} P_n^m
                \end{split} \\
                \begin{split}
                    &= P_n^{m + 2} + \left[ -2 x (m + 1) (1 - x^2)^{-1/2} \right] P_n^{m + 1} \\ &\quad {}+ \left[ -m + m x^2 + m^2 x - 2 m x^2 + 2 m x^2 + n (n + 1) (1 - x^2) - m^2 \right] (1 - x)^{-1} P_n^m
                \end{split} \\
                \begin{split}
                    &= P_n^{m + 2} + \left[ -2 x (m + 1) (1 - x^2)^{-1/2} \right] P_n^{m + 1} \\ &\quad {}+ \left[ -m (1 - x^2) - m^2 (1 - x^2) + n (n + 1) (1 - x^2) \right] (1 - x)^{-1} P_n^m
                \end{split} \\
                \begin{split}
                    &= P_n^{m + 2} + \left[ -2 x (m + 1) (1 - x^2)^{-1/2} \right] P_n^{m + 1} \\ &\quad {}+ \left[ m (m + 1) + n (n _ 1) \right] P_n^m.
                \end{split}
            \end{align*}
            Fazendo $m \to m - 1$ segue que
            \begin{align*}
                0 = P_n^{m + 1} - 2 x m (1 - x^2)^{-1/2} P_n^m + \left[ n (n + 1) - (m - 1) m \right] P_n^{m - 1}.
            \end{align*}
        \end{solution}

        \part $(2 n + 1) x P _n^m(x) = (n + m) P_{n - 1}^m(x) + (n - m + 1) P_{n + 1}^m (x) = 0$,
        \begin{solution}
            Derivando \eqref{eq:legendre_ass_funcao_geratriz} em rela\c{c}\~{a}o a $t$ temos que
            \begin{align*}
                \frac{(2 m)!}{2^m m!} \frac{(2m + 1) (x - t)}{(1 - 2 x t + t^2)^{m + 3/2}} &= \sum_{n = 0}^\infty P_{n + m}^m(x) n t^{n - 1}, \\
                (2m + 1) (x - t) \sum_{n = 0}^\infty P_{n + m}^m(x) t^n &= \sum_{n = 0}^\infty P_{n + m}^m(x) n t^{n - 1} (1 - 2 x t + t^2) \\
                \begin{split}
                    (2m + 1) x P_{n + m}^m - (2m + 1) P_{n + m - 1}^m &= (n + 1) P_{n + m + 1}^m \\ &\quad {}- 2 n x P_{n + m}^m + (n + 1) P_{n + m - 1}^n
                \end{split} \\
                \left[ (2m + 1) x + 2 n x \right] P_{n + m}^m &= \left[ (2m + 1) + (n - 1) \right] P_{n + m - 1}^m + (n + 1) P_{n + m + 1}^m.
            \end{align*}
            Fazendo $n = k - m$ segue que
            \begin{align*}
                \left[ x (2 k + 1) \right] P_k^m &= \left( m + k \right) P_{k = 1}^m + \left( k - m + 1 \right) P_{k + 1}.
            \end{align*}
        \end{solution}

        \part $(2n + 1) (1 - x^2)^{1/2} P_n^m(x) = P_{n + 1}^{m + 1}(x) - P_{n - 1}^{m + 1}(x)$,
        \begin{solution}
            Por \eqref{eq:legendre_rel_rec_2n+1der} segue que
            \begin{align*}
                (2 n + 1) P_n^m(x) &= P_{n + 1}^{m + 1}(x) - P_{n - 1}^{m + 1}(x)
            \end{align*}
            TODO
        \end{solution}

        \part $(1 - x^2)^{1/2} {P_n^m}'(x) = 2^{-1} P_n^{m + 1}(x) - 2^{-1} (n + m) (n - m + 1) P_n^{m - 1}(x)$.
        \begin{solution}
            Segue de \eqref{eq:legendre_ass} ao efetuar a mudan\c{c}a de vari\'{a}vel representada por $y(x) = (1 - x^2)^{m/2} w(x)$ que
            \begin{align*}
                (1 - x^2) w'' - 2 (m + 1) x w' + (\nu - m) (\nu + m + 1) w &= 0.
            \end{align*}
            Como $P_n^{m - 1}(x)$ satisfaz a equa\c{c}\~{a}o acima temos que
            \begin{align*}
                0 &= (1 - x^2) P_n^{m + 1}(x) - 2 m x P_n^m(x) + (n - m + 1) (m + m) P_n^{m - 1}(x) \\
                \begin{split}
                    &= (1 - x^2)^{m/2} P_n^{m + 1}(x) - 2 m x (1 - x^2)^{m/2 - 1} P_n^m(x) \\ &\quad {}+ (n + m) (n - m + 1) (1 - x^2)^{m/2 - 1} P_n^{m - 1}
                \end{split} \\
                \begin{split}
                    &= (1 - x^2)^{-1/2} P_n^{m + 1}(x) - 2\left[ -P_n^{m \prime}(x) + (1 - x^2)^{-1/2} P_n^{m + 1}(x) \right] \\ &\quad +{} (n + m) (n - m + 1) (1 - x^2)^{-1/2} P_n^{m - 1}(x)
                \end{split} \\
                &= (1 - x^2)^{1/2} 2 P_n^{m\prime}(x) - P_n^{m + 1}(x) + (n + m) (n - m + 1) P_n^{m - 1}(x).
            \end{align*}
            Da equa\c{c}\~{a}o acima segue que
            \begin{align*}
                \sqrt{1 - x^2} P_n^{m\prime} &= 2^{-1} P_n^{m + 1}(x) - 2^{-1} (n + m) (n - m + 1) P_n^{m - 1}(x).
            \end{align*}
        \end{solution}
    \end{parts}

    \question Definia $P_n^{-m}(x)$ ($m \geq 0$) usando a f\'{o}rmula de Rodrigues, ou seja,
    \begin{align*}
        P_n^{-m}(x) &= \frac{(1 - x^2)^{-m/2}}{2^n n!} \frac{\mathrm{d}^{n - m}}{\mathrm{d}x^{n - m}}\left( x^2 - 1 \right)^n.
    \end{align*}
    Mostre que
    \begin{align*}
        P_n^{-m}(x) &= (-1)^m \frac{(n - m)!}{(n + m)!} P_n^m(x).
    \end{align*}
    \begin{solution}
        Temos que
        \begin{align*}
            P_n^{-m}(x) &= \frac{(1 - x^2)^{-m/2}}{2^n n!} \frac{\mathrm{d}^{n - m}}{\mathrm{d}x^{n - m}}(x^2 - 1)^n \\
            &= \frac{(1 - x^2)^{-m/2}}{2^n n!} \frac{\mathrm{d}^{n - m}}{\mathrm{d}x^{n - m}}\left[ (x - 1) (x + 1) \right]^n \\
            &= \frac{(1 - x^2)^{-m/2}}{2^n n!} \frac{\mathrm{d}^{n - m}}{\mathrm{d}x^{n - m}}\left[ (x - 1)^n (x + 1)^n \right] \\
            &= \frac{(1 - x^2)^{-m/2}}{2^n n!} \sum_{k = 0}^{n - m} \frac{(n - m)!}{(n - m - k)! k!} \frac{\mathrm{d}^k}{\mathrm{d}x^k}(x - 1)^n \frac{\mathrm{d}^{n - m - k}}{\mathrm{d}x^{m - n - k}}(x + 1)^n \\
            &= \frac{(1 - x^2)^{-m/2}}{2^n n!} \sum_{k = 0}^{n - m} \frac{(n - m)!}{(n - m - k)! k!} \left[ \frac{n!}{(n - k)!}(x - 1)^{n - k} \frac{n!}{(m - k)!} (x + 1)^{m + k} \right], \\
            P_n^m(x) &= \frac{(1 - x^2)^{m/2}}{2^n n!} \sum_{k = m}^n \frac{(n + m)!}{(n - m - k)! k!} \left[ \frac{n!}{(n - k)!} (x - 1)^{n - k} \frac{n!}{(k - m)!} (x + 1)^{k - m} \right] \\
            &= \frac{(1 - x^2)^{-m/2}}{2^n n!} (1 - x^2)^m \sum_{j = 0}^{n - m} \frac{(n + m)!}{(n - j)! (m + j)!} \left[ \frac{n!}{(n - m - j)!} (x - 1)^{n - m - j} \frac{n!}{j!} (x + 1)^j \right] \\
            &= \frac{(-1)^m (1 - x^2)^{-m/2}}{2^n n!} \sum_{j = 0}^{n - m} \frac{(n + m)!}{(n -m - j)! j!} \left[ \frac{n!}{(n - j)!} (x - 1)^{n - j} \frac{n!}{(m + j)!} (x + 1)^{m + j} \right] \\
            &= \frac{(n + m)!}{(n - m)!} (-1)^m P_n^{-m}.
        \end{align*}
        Logo, segue que
        \begin{align*}
            P_n^{-m}(x) &= (-1)^m \frac{(n - m)!}{(n + m)!} P_n^m(x).
        \end{align*}
    \end{solution}

    \question Mostre que
    \begin{align*}
        P_n^m(0) &= \begin{cases}
            (-1)^{(n - m)/2} (n + m - 1)!! \left[ (n - m)!! \right]^{-1}, & \text{$n + m$ \'{e} par,} \\
            0, & \text{$n + m$ \'{e} \'{i}mpar.}
        \end{cases}
    \end{align*}
    \begin{solution}
        Para $x = 0$ temos que \eqref{eq:legendre_ass_funcao_geratriz} assume a forma
        \begin{align*}
            \frac{(2 m)!}{2^m m!} \frac{1}{(1 + t^2)^{m + 1/2}} &= \sum_{n = 0}^\infty P_{n + m}^m(0) t^n \\
            &= \sum_{k = m}^\infty P_k^m(0) t^{k - m} \\
            &= \frac{(2 m)!}{2^m m!} \sum_{n = 0}^\infty \frac{(-1)^n}{n!} (m + 1/2)_n t^{2n} \\
            &= \frac{(2 m)!}{2^m m!} \sum_{n = 0}^\infty \frac{(-1)^n}{n!} \frac{(2 m + 2 n - 1)!!}{(2 m - 1)!!} \frac{t^{2n}}{2^n} \\
            &= \frac{(2 m)!}{2^m m!} \sum_{k = 0}^\infty \frac{(-1)^{(k - m)/2}}{\left( (k - m)/2 \right)!} \frac{(m + k - 1)!!}{(2 m - 1)!!} \frac{t^{k + m}}{2^{k - m/2}} \\
            &= \frac{(2 m)!}{2^m m!} \sum_{k = 0}^\infty \frac{(-1)^{(k - m)/2}}{(k - m)!!} \frac{2^{(k - m)/2}}{2^{k - m/2}} \frac{(m + k - 1)!!}{(2 m - 1)!!} t^{k - m} \\
            &= \sum_{k = 0}^\infty \frac{(-1)^{(k - m)/2}}{(k - m)!} (m + k - 1)!! t^{k - m}.
        \end{align*}
        Logo, segue que
        \begin{align*}
            P_k^m(0) = (-1)^{(k - m)/2} \frac{(m + k - 1)!!}{(k - m)!!}.
        \end{align*}
    \end{solution}

    \question Sejam $P_n^m(x)$ as fun\c{c}\~{o}es associadas de Legendere de primeira es\'{e}cie. Mostre que
    \begin{align*}
        P_n^n(\cos \theta) &= \left( 2n - 1 \right)!! \sin^n \theta, & n = 0, 1, 2, \ldots.
    \end{align*}
    \begin{solution}
        Temos que
        \begin{align*}
            P_n^n(x) &= (1 - x^2)^{n / 2} \frac{\mathrm{d}^n}{\mathrm{d}x^n} P_n(x) \\
            &= \frac{(1 - x^2)^{n / 2}}{2^n n!} \frac{\mathrm{d}^{2n}}{\mathrm{d}x^{2n}} (x^2 - 1)^n.
        \end{align*}
        Mas
        \begin{align*}
            (x^2 - 1)^n &= \sum_{k = 0}^n \binom{n}{k} (x^2)^{n - k} (-1)^k \\
            &= x^{2n} \underbrace{- \binom{n}{1} x^{2n - 2} + \binom{n}{2} x^{2n - 4} - \ldots + (-1)^n}_{\text{nulo ao calcular $\mathrm{d}^2n/\mathrm{d}x^{2n}$}} \\
            \frac{\mathrm{d} ^{2n}}{\mathrm{d}x^{2n}} \left[ \left( x^2 - 1 \right)^n \right] &= \frac{\mathrm{d}^{2n}}{\mathrm{d}x^{2n}} \left( x^{2n} \right) \\
            &= 2n (2n - 1) \ldots 1 \\
            &= (2n)!
        \end{align*}
        e portanto $P_n^n(x) = \left( 1 - x^2 \right)^{n / 2} (2n)! \left( 2^n n! \right)^{-1}$.
        Como
        \begin{align*}
            \frac{(2n)!}{2^n n!} &= \frac{(2n) (2n - 1) (2n - 2) (2n - 3) \ldots 2 \cdot 1}{2^n n (n - 1) \ldots 1} \\
            &= \frac{(2n) (2n - 1) (2n - 2) (2n - 3) \ldots 2 \cdot 1}{(2n - 2) \ldots 2} \\
            &= (2n - 1) (2n - 3) \ldots 1 \\
            &= (2n - 1)!!
        \end{align*}
        temos que $P_n^n(x) = \left( 1 - x^2 \right)^{n / 2} (2n - 1)!!$.

        Fazendo $x = \cos \theta$ temos que $(1 - x^2)^{1/2} = \sin \theta$, $\theta \in [0, \pi]$. Se $x \in [-1, 1]$ ent\~{a}o $\theta \in [0, \pi]$ e portanto $P_n^n(\cos \theta) = (2n - 1)!! \sin^n \theta$.
    \end{solution}

    \question Seja $P_n$ o $n$-\'{e}simo polin\^{o}mio de Legendre ($n = 0, 1, 2, \ldots$). Calcule $P_n(0)$.
    \begin{solution}
        
    \end{solution}

    \question[E de 2010] Seja $P_n$ o $n$-\'{e}simo polin\^{o}mio de Legendre. Mostre que
    \begin{align*}
        \sum_{n = 0}^\infty \frac{2n + 1}{2} P_n(x) P_n(x') &= \delta(x - x'),
    \end{align*}
    $x, x' \in (-1, 1)$. Dica: lembre-se que os polin\^{o}mios de Legendere forma um conjunto ortogonal completo, com $\int_{-1}^1 P_n(x) P_m(x) \,\mathrm{d}x = 2 (2 n + 1)^{-1} \delta_{nm}$.
    \begin{solution}
        
    \end{solution}

    \question[T6 de 2011] Seja $P_n(x)$ ($n = 0, 1, 2, \ldots$) os polin\^{o}mios de Legendere. Mostre que
    \begin{align*}
        P'_n(x) &= (2n - 1) P_{n + 1}(x) + (2n - 5) P_{n - 3} + (2n - 9) P_{n - 5}(x) + \ldots + \Delta_n,
    \end{align*}
    onde $\Delta_n = 3 P_1(x)$ se $n$ for par e $\Delta_n = P_0(x)$ se $n$ for \'{i}mpar.
    \begin{solution}
        Temos \eqref{eq:legendre_rel_rec_2n+1der}, 
        \begin{align*}
            (2n + 1) P_n(x) &= P'_{n + 1}(x) - P'_{n - 1}(x),
        \end{align*}
        e fazendo $n \to n - 1$ obtemos
        \begin{align*}
            P'_n(x) &= (2n - 1) P_{n - 1}(x) + P'_{n - 2}(x) \\
            P'_{n - 2}(x) &= (2n - 5) P_{n - 3}(x) + P'_{n - 4}(x) && n \to n - 2 \\
            P'_{n - 4}(x) &= (2n - 9) P_{n - 5} (x)+ P'_{n - 6}(x) && n \to n - 2 \\
            \vdots
        \end{align*}
        Somando as equa\c{c}\~{o}es acima obtemos
        \begin{align*}
            P'_n(x) &= (2n - 1) P_{n + 1}(x) + (2n - 5) P_{n - 3}(x) + (2n - 9) P_{n - 5}(x) + \ldots.
        \end{align*}
        Se $n$ \'{e} par, a \'{u}ltima rela\c{c}\~{a}o \'{e}
        \begin{align*}
            P_2(x) &= 3 P_1(x) + P_0(x) = 3 P_1(x)
        \end{align*}
        e se $n$ \'{e} \'{i}mpar, a \'{u}ltima rela\c{c}\~{a}o \'{e}
        \begin{align*}
            P_1(x) &= P_0(x) + 0 = P_0(x).
        \end{align*}
        Logo,
        \begin{align*}
            \Delta_n &= \begin{cases}
                3 P_1(x), & \text{n par}, \\
                P_0(x), & \text{n \'{i}mpar}.
            \end{cases}
        \end{align*}
    \end{solution}

    \question[P2 de 2011] Sejam $P_n(x)$ os polin\^{o}mios de Legendre ($n = 0, 1, 2, \ldots$). Mostre que
    \begin{parts}
        \part $P_n(1) = 1$,
        \begin{solution}
            Temos \eqref{eq:legendre_funcao_geratriz} e \eqref{eq:legendre_funcao_geratriz_serie}. Logo para $x = 1$ temos
            \begin{align*}
                \sum_{n = 0}^\infty P_n(1) t^n &= \frac{1}{\sqrt{1 - 2 t + t^2}} \\
                &= \frac{1}{1 - t} \\
                &= \sum_{n = 0}^\infty t^n \\
                &= 1.
            \end{align*}
        \end{solution}

        \part $\int_0^1 P_{2n}(x) \,\mathrm{d}x = 0$ ($n \neq 0$).
        \begin{solution}
            Utilizando \eqref{eq:legendre_rel_rec_2n+1der} temos que
            \begin{align*}
                \int_0^1 P_{2n}(x) \,\mathrm{d}x &= \int_0^1 \left[ \frac{P'_{2n + 1}(x) - P'_{2n - 1}(x)}{2 (2n) + 1} \right] \,\mathrm{d}x \\
                &= \frac{1}{4 n + 1} \left[ \left. P_{2n + 1}(x) \right|_0^1 - \left. P_{2n - 1}(x) \right|_0^1 \right] \\
                &= \frac{1}{(4n + 1)} \left[ P_{2n + 1}(1) - P_{2n - 1}(0) - P_{2n - 1}(1) + P_{2n - 1}(0) \right] \\
                &= \frac{1}{(4n + 1)} \left[ - P_{2n - 1}(0) + P_{2n - 1}(0) \right] \\
                &= \frac{1}{(4n + 1)} \left[ - 0 + 0 \right]  && \bigstar \\
                &= 0,
            \end{align*}
            onde $\bigstar$ decorre de
            \begin{align*}
                \sum_{n = 0}^\infty P_n(0) t^n &= \frac{1}{\sqrt{1 + t^2}} \\
                &= \sum_{m = 0}^\infty \frac{(1/2)_m (-1)^m (t^2)^m}{m!} \\
                P_{2n + 1}(0) &= 0.
            \end{align*}
        \end{solution}
    \end{parts}

    \question[E de 2011] Sejam $P_n(x)$ os polin\^{o}mios de Legendre ($n = 0, 1, 2, \ldots$). Mostre que
    \begin{align*}
        (2n + 1) x P_n(x) &= (n + 1) P_{n + 1}(x) + n P_{n - 1}(x).
    \end{align*}
    \begin{solution}
        Temos que
        \begin{align*}
            g(x, t) &= \frac{1}{\sqrt{1 - 2 x t + t^2}} \\
            &= \sum_{n = 0}^\infty P_n(x) t^n.
        \end{align*}
        Derivando em rela\c{c}\~{a}o a $t$
        \begin{align*}
            \frac{\partial g(x, t)}{\partial t} &= \left( \frac{-1}{2} \right) \frac{-2 x + 2 t}{\left[ 1 - 2 x t + t^2 \right]^{3/2}} \\
            &= \frac{x - t}{1 - 2 x t + t^2} g(x, t) \\
            (x - t) g(x, t) &= (1 - 2 x t + t^2) \frac{\partial g(x, t)}{\partial t} \\
            (x - t) \sum_{n = 0}^\infty P_n(x) t^n &= (1 - 2 x t + t^2) \sum_{n = 0}^\infty n P_n(x) t^{n - 1} \\
            \sum_{n = 0}^\infty x P_n(x) t^n - \sum_{n = 0}^\infty P_n(x) t^{n + 1} &= \sum_{n = 0}^\infty n P_n(x) t^{n - 1} - \sum_{n = 0}^\infty 2 n x P_n(x) t^n + \sum_{n = 0}^\infty n P_n(x) t^{n + 1} \\
            \sum_{n = 0}^\infty (2n + 1) x P_n(x) t^n - \sum_{n = 1}^\infty P_{n - 1}(x) t^n &= \sum_{n = 0}^\infty (n + 1) P_{n + 1}(x) t^n + \sum_{n = 1}^\infty (n - 1) P_{n - 1}(x) t^n \\
            x P_0(x) + \sum_{n = 1}^\infty (2n + 1) x P_n(x) t^n &= P_1(x) + \sum_{n = 1}^\infty (n + 1) P_{n + 1}(x) t^n + \sum_{n = 1}^\infty n P_{n - 1}(x) t^n.
        \end{align*}
        Por fim, temos que
        \begin{align*}
            \begin{cases}
                x P_0(x) = P_1(x), \\
                (2n + 1) x P_n(x) = (n + 1) P_{n + 1}(x) + n P_{n - 1}(x), & n = 1, 2, \ldots.
            \end{cases}
        \end{align*}
    \end{solution}
\end{questions}
\bibliographystyle{plain}
\bibliography{bibliography}
\end{document}
