% Filename: lista7.tex
% This code is part of 'Solutions for MS550, M\'{e}todos de Matem\'{a}tica Aplicada I, and F520, M\'{e}todos Matem\'{a}ticos da F\'{i}sica I'
% 
% Description: This file corresponds to the solutions of homework sheet 7.
% 
% Created: 30.05.12 09:39:27 AM
% Last Change: 30.05.12 04:41:53 PM
% 
% Authors:
% - Raniere Silva (2012): initial version
% 
% Copyright (c) 2012 Raniere Silva <r.gaia.cs@gmail.com>
% 
% Permission is granted to copy, distribute and/or modify this document under the terms of the GNU Free Documentation License, Version 1.3 or any later version published by the Free Software Foundation; with no Invariant Sections, no Front-Cover Texts, and no Back-Cover Texts.
% This document is distributed in the hope that it will be useful, but WITHOUT ANY WARRANTY; without even the implied warranty of MERCHANTABILITY or FITNESS FOR A PARTICULAR PURPOSE.
% More details at <http://www.gnu.org/licenses/>
%
\documentclass[a4paper,12pt, leqno, answers]{exam}
% Customiza\c{c}\~{a}o da classe exam
\newcommand{\mycheader}{Lista 7 - Fun\c{c}\~{a}o de Legendre}
\header{MS550, F520}{\mycheader}{\thepage/\numpages}
\headrule
\footer{Dispon\'{i}vel em \\% Filename: repository.tex
% 
% This code is part of 'Solutions for MS550, M\'{e}todos de Matem\'{a}tica Aplicada I, and F520, M\'{e}todos Matem\'{a}ticos da F\'{i}sica I'
% 
% Description: This file keeps the url of the repository.
% 
% Created: 07.03.12 04:00:00 PM
% Last Change: 30.05.12 04:40:25 PM
% 
% Authors:
% - Raniere Silva (2012): initial version
% 
% Copyright (c) 2012 Raniere Silva <r.gaia.cs@gmail.com>
% 
% This work is licensed under the Creative Commons Attribution-ShareAlike 3.0 Unported License. To view a copy of this license, visit http://creativecommons.org/licenses/by-sa/3.0/ or send a letter to Creative Commons, 444 Castro Street, Suite 900, Mountain View, California, 94041, USA.
%
% This work is distributed in the hope that it will be useful, but WITHOUT ANY WARRANTY; without even the implied warranty of MERCHANTABILITY or FITNESS FOR A PARTICULAR PURPOSE.
%
\url{https://github.com/r-gaia-cs/solucoes_listas_metodos}
}{}{Reportar erros para \\\href{mailto:r.gaia.cs@gmail.com}{r.gaia.cs@gmail.com}
}
\footrule 
\pagestyle{headandfoot}
\renewcommand{\solutiontitle}{\noindent\textbf{Solu\c{c}\~{a}o:}\enspace}
\SolutionEmphasis{\slshape}
\unframedsolutions
\pointname{}

% Para impress\~{a}o
\usepackage[top=3cm, bottom=3cm, left=2cm, right=2cm]{geometry}

% Para ereaders (Kindle, Nook, Kobo, ...) and tablets (iPad, GalaxyTab, ...)
% \usepackage[papersize={180mm,240mm},margin=2mm]{geometry}
% \sloppy

% Pacotes
\usepackage[utf8]{inputenc}
\usepackage[brazil]{babel}
\usepackage{amsmath}
\usepackage{amsfonts}
\usepackage{amssymb}
\usepackage{hyperref}
\usepackage{graphicx}

% Customiza\c{c}\~{a}o do pacote amsmath
\allowdisplaybreaks[4]

% Novos ambientes
% \newenvironment{fwsolution}{\begin{EnvFullwidth}\begin{TheSolution}}{\end{TheSolution}\end{EnvFullwidth}}

% Novos comandos
\newcommand{\devp}[2]{\frac{\partial #1}{\partial #2}}
\newcommand{\grad}{\mbox{grad }}
\newcommand{\diver}{\mbox{div }}
\newcommand{\rot}{\mbox{rot }}

\begin{document}
%cover
\thispagestyle{empty}
% Filename: cover.tex
% This code is part of 'Solutions for MS550, M\'{e}todos de Matem\'{a}tica Aplicada I, and F520, M\'{e}todos Matem\'{a}ticos da F\'{i}sica I'
% 
% Description: This file corresponds to the cover.
% 
% Created: 30.05.12 04:40:25 PM
% Last Change: 31.05.12 10:11:55 PM
% 
% Authors:
% - Raniere Silva (2012): initial version
% 
% Copyright (c) 2012 Raniere Silva <r.gaia.cs@gmail.com>
% 
% Permission is granted to copy, distribute and/or modify this document under the terms of the GNU Free Documentation License, Version 1.3 or any later version published by the Free Software Foundation; with no Invariant Sections, no Front-Cover Texts, and no Back-Cover Texts.
% This document is distributed in the hope that it will be useful, but WITHOUT ANY WARRANTY; without even the implied warranty of MERCHANTABILITY or FITNESS FOR A PARTICULAR PURPOSE.
% More details at <http://www.gnu.org/licenses/>
%
\begin{center}
    \LARGE{Solu\c{c}\~{o}es para MS550, M\'{e}todos de Matem\'{a}tica Aplicada I, e F520, M\'{e}todos Matem\'{a}ticos da F\'{i}sica I}
    
    \Large{\mycheader}
\end{center}
\vspace{.5\textheight}

\begin{tabular}{|p{.9\textwidth}|}
\hline
\'{E} garantida a permiss\~{a}o para copiar, distribuir e/ou modificar este documento sob os termos da Licen\c{c}a de Documenta\c{c}\~{a}o Livre GNU (GNU Free Documentation License), Vers\~{a}o 1.2 ou qualquer vers\~{a}o posterior publicada pela Free Software Foundation; sem Se\c{c}\~{o}es Invariantes, Textos de Capa Frontal, e sem Textos de Quarta Capa.

Este documento \'{e} distribuido na esperança que possa ser \'{u}til, mas SEM NENHUMA GARANTIA; sem uma garantia implicita de ADEQUA\c{C}\~{A}O a qualquer MERCADO ou APLICA\c{C}\~{A}O EM PARTICULAR.

Mais detalhes em \url{http://www.gnu.org/licenses/}.
\\ \hline
\end{tabular}

\newpage
\setcounter{page}{1}
\begin{questions}
    \question Mostre, a partir da rela\c{c}\~{a}o de recorr\^{e}ncia que segue derivando a fun\c{c}\~{a}o geratriz, que
    \begin{parts}
        \part $(n + 1) P_n(x) = P'_{n + 1}(x) - x P'_n(x)$,
        \begin{solution}
            
        \end{solution}

        \part $(1 - x^2) P'_n(x) = P_{n - 1}(x) - n x P_n(x)$.
        \begin{solution}
            
        \end{solution}
    \end{parts}

    \question Mostre que
    \begin{align*}
        P_n(\cos \theta) = (-1)^n \frac{r^{n + 1}}{n!} \frac{\partial^n}{\partial z^n} \left( \frac{1}{r} \right).
    \end{align*}
    \begin{solution}
        
    \end{solution}

    \question Mostre, usando explicitamente coordenadas esf\'{e}ricas $(r, \theta, \phi)$, que
    \begin{align*}
        \frac{\partial}{\partial z} \left[ \frac{P_n(\cos \theta)}{r^{n + 1}} \right] &= - (n + 1) \frac{P_{n + 1}(\cos \theta)}{r^{n + 2}}.
    \end{align*}
    \begin{solution}
        
    \end{solution}

    \question Mostre que
    \begin{parts}
        \part $\int_{-1}^1 P_n(x) \,\mathrm{d}x = 0, n \geq 1$,
        \begin{solution}
            
        \end{solution}

        \part $\int_{-1}^1 x^{2n + 1} P_{2m}(x) \,\mathrm{d}x = 0, m \neq n$,
        \begin{solution}
            
        \end{solution}

        \part $\int_{-1}^1 x^m P_n(x) \,\mathrm{d}x = 0, m < n$,
        \begin{solution}
            
        \end{solution}

        \part $\int_{-1}^1 x^n P_n(x) \,\mathrm{d}x = \left[ 2^{n + 1} (n!)^2 \right] / (2n + 1)!$.
        \begin{solution}
            
        \end{solution}
    \end{parts}

    \question Mostre que
    \begin{parts}
        \part $P'_n(1) = n (n + 1) / 2$,
        \begin{solution}
            
        \end{solution}

        \part $P'_n(-1) = (-1)^{n + 1} n (n + 1) / 2$.
        \begin{solution}
            
        \end{solution}
    \end{parts}

    \question Construa (a menos de uma normaliza\c{c}\~{a}o) os polinômios de Legendre $P_n(x)$ utilizando o processo de ortogonaliza\c{c}\~{a}o de Gram-Schmidt aplicado \`{a} base $\left\{ 1, x, x^2, x^3, \ldots \right\}$ do espa\c{c}o dos polinômios no intervalo $(-1, 1)$ equipado com o produto escalar $(f, g) = \int_{-1}^1 f(x) g(x) \,\mathrm{d} x$.
    \begin{solution}
        
    \end{solution}

    \question Mostre que as fun\c{c}\~{o}es de Legentre de segunda esp\'{e}cie $Q_n(x)$ satisfazem as seguintes rela\c{c}\~{o}es de recorr\^{e}ncia:
    \begin{parts}
        \part $(2n + 1) x Q_n(x) = (n + 1) Q_{n + 1}(x) + n Q_{n - 1}(x)$,
        \begin{solution}
            
        \end{solution}

        \part $(2n + 1) Q_n(x) = Q'_{n + 1}(x) - Q'_{n - 1}(x)$.
        \begin{solution}
            
        \end{solution}
    \end{parts}

    \question Mostre que $Q_n(-x) = (-1)^{n + 1} Q_n(x)$.
    \begin{solution}
        
    \end{solution}

    \question Mostre que
    \begin{align*}
        n \left[ P_n(x) Q_{n - 1}(x) - Q_n(x) Q_{n - 1}(x) \right] = P_1(x) Q_0(x) - P_0(x) Q_1(x) = 1.
    \end{align*}
    \begin{solution}
        
    \end{solution}

    \question Mostre que os polinômios de Legendre associados $P_n^m(x)$ satisfazem as seguintes rela\c{c}\~{o}es de recorr\^{e}ncia:
    \begin{parts}
        \part $P_n^{m + 1}(x) - 2 m x \left( 1 - x^2 \right)^{-1/2} P_n^m(x) + \left[ n (n _ 1) - m (m - 1) \right] P_n^{m - 1}(x) = 0$,
        \begin{solution}
            
        \end{solution}

        \part $(2 n + 1) x P _n^m(x) = (n + m) P_{n - 1}^m(x) + (n - m + 1) P_{n + 1}^m (x) = 0$,
        \begin{solution}
            
        \end{solution}

        \part $(2n + 1) (1 - x^2)^{1/2} P_n^m(x) = P_{n + 1}^{m + 1}(x) - P_{n - 1}^{m + 1}(x)$,
        \begin{solution}
            
        \end{solution}

        \part $(1 - x^2)^{1/2} {P_n^m}'(x) = 2^{-1} P_n^{m + 1}(x) - 2^{-1} (n + m) (n - m + 1) P_n^{m - 1}(x)$.
        \begin{solution}
            
        \end{solution}
    \end{parts}

    \question Definia $P_n^{-m}(x)$ ($m \geq 0$) usando a f\'{o}rmula de Rodrigues, ou seja,
    \begin{align*}
        P_n^{-m}(x) &= \frac{(1 - x^2)^{-m/2}}{2^n n!} \frac{\mathrm{d}^{n - m}}{\mathrm{d}x^{n - m}}\left( x^2 - 1 \right)^n.
    \end{align*}
    Mostre que
    \begin{align*}
        P_n^{-m}(x) &= (-1)^m \frac{(n - m)!}{(n + m)!} P_n^m(x).
    \end{align*}
    \begin{solution}
        
    \end{solution}

    \question Mostre que
    \begin{align*}
        P_n^m(0) &= \begin{cases}
            (-1)^{(n - m)/2} (n + m - 1)!! \left[ (n - m)!! \right]^{-1}, & \text{$n + m$ \'{e} par,} \\
            0, & \text{$n + m$ \'{e} \'{i}mpar.}
        \end{cases}
    \end{align*}
    \begin{solution}
        
    \end{solution}
\end{questions}
\end{document}
