% Filename: lista8.tex
% 
% This code is part of 'Solutions for MS550, M\'{e}todos de Matem\'{a}tica Aplicada I, and F520, M\'{e}todos Matem\'{a}ticos da F\'{i}sica I'
% 
% Description: This file corresponds to the solutions of homework sheet 8.
% 
% Created: 21.05.12 08:46:45 PM
% Last Change: 16.06.12 11:18:03 AM
% 
% Authors:
% - Raniere Silva (2012): initial version
% 
% Copyright (c) 2012 Raniere Silva <r.gaia.cs@gmail.com>
% 
% This work is licensed under the Creative Commons Attribution-ShareAlike 3.0 Unported License. To view a copy of this license, visit http://creativecommons.org/licenses/by-sa/3.0/ or send a letter to Creative Commons, 444 Castro Street, Suite 900, Mountain View, California, 94041, USA.
%
% This work is distributed in the hope that it will be useful, but WITHOUT ANY WARRANTY; without even the implied warranty of MERCHANTABILITY or FITNESS FOR A PARTICULAR PURPOSE.
%
\documentclass[a4paper,12pt, leqno, answers]{exam}
% Customiza\c{c}\~{a}o da classe exam
\newcommand{\mycheader}{Lista 8 - Problema de Sturm-Liouville}
\header{MS550, F520}{\mycheader}{\thepage/\numpages}
\headrule
\footer{Dispon\'{i}vel em \\% Filename: repository.tex
% 
% This code is part of 'Solutions for MS550, M\'{e}todos de Matem\'{a}tica Aplicada I, and F520, M\'{e}todos Matem\'{a}ticos da F\'{i}sica I'
% 
% Description: This file keeps the url of the repository.
% 
% Created: 07.03.12 04:00:00 PM
% Last Change: 30.05.12 04:40:25 PM
% 
% Authors:
% - Raniere Silva (2012): initial version
% 
% Copyright (c) 2012 Raniere Silva <r.gaia.cs@gmail.com>
% 
% This work is licensed under the Creative Commons Attribution-ShareAlike 3.0 Unported License. To view a copy of this license, visit http://creativecommons.org/licenses/by-sa/3.0/ or send a letter to Creative Commons, 444 Castro Street, Suite 900, Mountain View, California, 94041, USA.
%
% This work is distributed in the hope that it will be useful, but WITHOUT ANY WARRANTY; without even the implied warranty of MERCHANTABILITY or FITNESS FOR A PARTICULAR PURPOSE.
%
\url{https://github.com/r-gaia-cs/solucoes_listas_metodos}
}{}{Reportar erros para \\\href{mailto:r.gaia.cs@gmail.com}{r.gaia.cs@gmail.com}
}
\footrule 
\pagestyle{headandfoot}
\renewcommand{\solutiontitle}{\noindent\textbf{Solu\c{c}\~{a}o:}\enspace}
\SolutionEmphasis{\itshape}
\unframedsolutions
\pointname{}

% Filename: paper_size.tex
%
% This code is part of 'Solutions for MS550, M\'{e}todos de Matem\'{a}tica Aplicada I, and F520, M\'{e}todos Matem\'{a}ticos da F\'{i}sica I'
% 
% Description: This file corresponds to the paper size output.
% 
% Created: 07.03.12 04:00:00 PM
% Last Change: 30.05.12 04:40:25 PM
% 
% Authors:
% - Raniere Silva (2012): initial version
% 
% Copyright (c) 2012 Raniere Silva <r.gaia.cs@gmail.com>
% 
% This work is licensed under the Creative Commons Attribution-ShareAlike 3.0 Unported License. To view a copy of this license, visit http://creativecommons.org/licenses/by-sa/3.0/ or send a letter to Creative Commons, 444 Castro Street, Suite 900, Mountain View, California, 94041, USA.
%
% This work is distributed in the hope that it will be useful, but WITHOUT ANY WARRANTY; without even the implied warranty of MERCHANTABILITY or FITNESS FOR A PARTICULAR PURPOSE.
%
% Para impress\~{a}o
\usepackage[top=3cm, bottom=3cm, left=2cm, right=2cm]{geometry}

% Para ereaders (Kindle, Nook, Kobo, ...) and tablets (iPad, GalaxyTab, ...)
% \usepackage[papersize={180mm,240mm},margin=2mm]{geometry}
% \sloppy


% Pacotes
\usepackage[utf8]{inputenc}
\usepackage[T1]{fontenc}
\usepackage[brazil]{babel}
\usepackage{amsmath}
\usepackage{amsfonts}
\usepackage{amssymb}
\usepackage{hyperref}
\usepackage{graphicx}

% Customiza\c{c}\~{a}o do pacote amsmath
\allowdisplaybreaks[4]

% Novos ambientes
% \newenvironment{fwsolution}{\begin{EnvFullwidth}\begin{TheSolution}}{\end{TheSolution}\end{EnvFullwidth}}

% Novos comandos
\newcommand{\devp}[2]{\frac{\partial #1}{\partial #2}}
\newcommand{\grad}{\mbox{grad }}
\newcommand{\diver}{\mbox{div }}
\newcommand{\rot}{\mbox{rot }}

\begin{document}
%cover
\thispagestyle{empty}
% Filename: cover.tex
% This code is part of 'Solutions for MS550, M\'{e}todos de Matem\'{a}tica Aplicada I, and F520, M\'{e}todos Matem\'{a}ticos da F\'{i}sica I'
% 
% Description: This file corresponds to the cover.
% 
% Created: 30.05.12 04:40:25 PM
% Last Change: 31.05.12 10:11:55 PM
% 
% Authors:
% - Raniere Silva (2012): initial version
% 
% Copyright (c) 2012 Raniere Silva <r.gaia.cs@gmail.com>
% 
% Permission is granted to copy, distribute and/or modify this document under the terms of the GNU Free Documentation License, Version 1.3 or any later version published by the Free Software Foundation; with no Invariant Sections, no Front-Cover Texts, and no Back-Cover Texts.
% This document is distributed in the hope that it will be useful, but WITHOUT ANY WARRANTY; without even the implied warranty of MERCHANTABILITY or FITNESS FOR A PARTICULAR PURPOSE.
% More details at <http://www.gnu.org/licenses/>
%
\begin{center}
    \LARGE{Solu\c{c}\~{o}es para MS550, M\'{e}todos de Matem\'{a}tica Aplicada I, e F520, M\'{e}todos Matem\'{a}ticos da F\'{i}sica I}
    
    \Large{\mycheader}
\end{center}
\vspace{.5\textheight}

\begin{tabular}{|p{.9\textwidth}|}
\hline
\'{E} garantida a permiss\~{a}o para copiar, distribuir e/ou modificar este documento sob os termos da Licen\c{c}a de Documenta\c{c}\~{a}o Livre GNU (GNU Free Documentation License), Vers\~{a}o 1.2 ou qualquer vers\~{a}o posterior publicada pela Free Software Foundation; sem Se\c{c}\~{o}es Invariantes, Textos de Capa Frontal, e sem Textos de Quarta Capa.

Este documento \'{e} distribuido na esperança que possa ser \'{u}til, mas SEM NENHUMA GARANTIA; sem uma garantia implicita de ADEQUA\c{C}\~{A}O a qualquer MERCADO ou APLICA\c{C}\~{A}O EM PARTICULAR.

Mais detalhes em \url{http://www.gnu.org/licenses/}.
\\ \hline
\end{tabular}

\newpage
\setcounter{page}{1}
\begin{questions}
    \question Encontre os autovalores e autofun\c{c}\~{o}es dos seguintes problemas de Sturm-Liouville:
    \begin{parts}
        \part $\begin{cases}
            y'' + \lambda y = 0, & 0 < x < 1, \\
            y(0) + y'(0) = 0, \\
            y(1) = 0.
        \end{cases}$
        \begin{solution}
            Para $\lambda = 0$ temos que $y'' = 0$ que implica em $y' = c$ que por sua vez implica em $y = c x + d$. Quando $x = 1$ temos
            \begin{align*}
                0 &= y(1) = c + d \Rightarrow d = -c.
            \end{align*}
            Quando $x = 0$ temos
            \begin{align*}
                0 &= y(0) + y'(0) = -c + c = 0.
            \end{align*}
            Portanto,
            \begin{align*}
                y(x) = c x + d
            \end{align*}
            \'{e} solu\c{c}\~{a}o.


            Para $\lambda > 0$, fazemos $\lambda = k^2$ com $k > 0$, e assim temos $y'' + k^2 y = 0$. Logo $y = \exp(rx)$ \'{e} solu\c{c}\~{a}o e temos
            \begin{align*}
                r^2 + k^2 = 0 \Rightarrow r = \pm i k.
            \end{align*}
            Ent\~{a}o
            \begin{align*}
                y(x) &= A \sin(k x) + B \cos(k x), \\
                y'(x) &= A k \cos(k x) - B k \sin(k x), \\
                y(x) + y'(x) &= (A - B k) \sin(k x) + (A k + B) \cos(k x).
            \end{align*}
            Quando $x = 0$ temos
            \begin{align*}
                0 &= y(0) + y'(0) = A k + B \Rightarrow A = - B k^{-1}, k \neq 0.
            \end{align*}
            Quando $x = 1$ temos
            \begin{align*}
                0 &= y(1) = A \sin(k) + B \cos(k) = -B k^{-1} \sin(k) + B \cos(k) \\
                &= B k^{-1} \left( - \sin(k) + k \cos(k) \right) \Rightarrow k = \tan(k) \text{ ou } B = 0.
            \end{align*}
            Portanto,
            \begin{align*}
                y(x) &= - (\tan(k))^{-1} \sin(\tan(k) x) + \cos(\tan(k) x), k = 1, 2, 3, \ldots.
            \end{align*}

            Para $\lambda < 0$, fazemos $\lambda = - k^2$ com $k > 0$ e assim temos $y'' - k^2 y = 0$. Logo, $y = \exp(rx)$ \'{e} solu\c{c}\~{a}o e temos
            \begin{align*}
                r^2 - k^2 = 0 \Rightarrow r = \pm k.
            \end{align*}
            Ent\~{a}o
            \begin{align*}
                y(x) &= C \cosh(k x) + D \sinh(k x).
            \end{align*}
            Quando $x = 1$ temos
            \begin{align*}
                0 &= y(1) = C \cosh(k) + D \sinh(k) \Rightarrow C = .
            \end{align*}
            Quando $x = 0$ temos
            \begin{align*}
                0 & = y(0) + y'(0) = C k \sinh(0) + D k \cosh(0) + C \cosh(0) + D \sinh(0) \\
                &= D k + C \Rightarrow C = - D.
            \end{align*}
            Portanto, existe apenas a solu\c{c}\~{a}o trivial.
        \end{solution}

        \part $\begin{cases}
            x^2 y'' + x y' + \lambda y = 0, & 0 < x < 1, \\
            y(1) = 0, \\
            y(e) = 0.
        \end{cases}$
        \begin{solution}
            Para $\lambda = 0$, temos que
            \begin{align*}
                x^2 y'' + x y' + \lambda y = x^2 y'' + x y' = 0.
            \end{align*}
            Logo, $y = x^r$ \'{e} solu\c{c}\~{a}o,
            \begin{align*}
                0 &= r (r - 1) + r = r^2 - r + r = r^2 \Rightarrow r = 0
            \end{align*}
            e portanto
            \begin{align*}
                y(x) = A x^0 + B \ln(x) x^0 = A + B \ln(x).
            \end{align*}
            Quando $x = 1$ temos
            \begin{align*}
                0 &= y(1) = A + B \ln(1) \Rightarrow A = 0.
            \end{align*}
            Quando $x = \exp(1)$ temos
            \begin{align*}
                0 & y(\exp(1)) = B \ln(\exp(1)) \Rightarrow B = 0.
            \end{align*}
            Portanto, para $\lambda = 0$ temos apenas a solu\c{c}\~{a}o trivial.

            Para $\lambda > 0$, fazendo $\lambda = k^2$ com $k > 0$, temos que
            \begin{align*}
                x^2 y'' + x y' + \lambda y = x^2 y'' + x y' + k^2 y = 0.
            \end{align*}
            Logo, $y = x^r$ \'{e} solu\c{c}\~{a}o,
            \begin{align*}
                0 &= r (r - 1) + r + k^2 = r^2 + k^2 \Rightarrow r = \pm k i
            \end{align*}
            e portanto
            \begin{align*}
                y(x) = A x^{-k i} + B x^{k i}.
            \end{align*}
            Quando $x = 1$ temos
            \begin{align*}
                0 &= y(1) = A + B \Rightarrow A = -B.
            \end{align*}
            Quando $x = \exp(1)$ temos
            \begin{align*}
                0 &= y(\exp(1)) = A (\exp(-k i) - \exp(k i)) = - A (\exp(k i) - \exp(-k i)) \\
                &= -2 A \sin(k) \Rightarrow k = n \pi, n = 1, 2, \ldots \text{ ou } A = 0.
            \end{align*}
            Portanto, para $\lambda > 0$ temos al\'{e}m da solu\c{c}\~{a}o trivial
            \begin{align*}
                y_n(x) &= x^{- n \pi i} - x^{n \pi i} \\
                &= \exp(\ln(x^{- n \pi i})) - \exp(\ln(x^{n \pi i})) \\
                &= \cos(n \pi \ln(x)) + \sin(n \pi \ln(x)) \\
                &= \sin(n \pi \ln(x)).
            \end{align*}

            Para $\lambda < 0$, fazendo $\lambda = - k^2$ com $k > 0$, temos que
            \begin{align*}
                x^2 y'' + x y' + \lambda y = x^2 y'' + x y' - k^2 y = 0.
            \end{align*}
            Logo, $y = x^r$ \'{e} solu\c{c}\~{a}o,
            \begin{align*}
                0 &= r (r - 1) + r - k^2 = r^2 - k^2 \Rightarrow r = \pm k
            \end{align*}
            e portanto
            \begin{align*}
                y(x) = A x^{k} + B x^{-k}.
            \end{align*}
            Quando $x = 1$ temos
            \begin{align*}
                0 &= y(1) = A + B \Rightarrow A = - B.
            \end{align*}
            Quando $x = \exp(1)$ temos
            \begin{align*}
                0 &= y(\exp(1)) = A \exp(k) + B \exp(-k) \Rightarrow \text{n\~{a}o tem solu\c{c}\~{a}o.}
            \end{align*}
            Portanto, para $\lambda < 0$ n\~{a}o temos solu\c{c}\~{a}o.

            Assim sendo, temos, al\'{e}m da solu\c{c}\~{a}o tivial, $y_n = \sin(n \pi \ln(x))$ quando $\lambda > 0$.
        \end{solution}

        \part[P3 de 2006] $\begin{cases}
            y'' + y' + (1 + \lambda) y = 0, & 0 < x < 1, \\
            y(0) = 0, \\
            y(1) = 0.
        \end{cases}$
        \begin{solution}
            Para $y = \exp(r x)$ temos que
            \begin{align*}
                r^2 - r + \left( 1 + \lambda \right) &= 0
            \end{align*}
            e portanto $r = (-1/2) \pm \sqrt{-3/4 - \lambda}$. Logo as solu\c{c}\~{o}es n\~{a}o triviais ocorrem quando $-3/4 - \lambda < 0$.

            Para $-3/4 - \lambda = - k^2$, $k > 0$, temos
            \begin{align*}
                r &= -1/2 \pm \sqrt{-k^2} \\
                &= -1/2 \pm i k.
            \end{align*}
            Logo,
            \begin{align*}
                y(x) &= A_1 \exp\left( (-1/2 + ik) x \right) + A_2 \exp\left( (-1/2 - i k) x \right) \\
                &= \exp(-x/2) \left( A_1 \exp(i k x) + A_2 \exp(-i k x) \right) \\
                &= \exp(-x/2) \left( B_1 \cos(k x) + B_2 \sin(k x) \right).
            \end{align*}
            Para $x = 0$,
            \begin{align*}
                0 = y(0) = \exp(0) B_1 \Rightarrow B_1 = 0,
            \end{align*}
            e para $x = 1$,
            \begin{align*}
                0 = y(1) = \exp(1) B_2 \sin(k) \Rightarrow \sin(k) = 0 \Rightarrow k = n \pi, n = 1, 2, \ldots.
            \end{align*}
            Portanto,
            \begin{align*}
                -3/4 - \lambda = -(n \pi)^2 \Rightarrow \lambda_n = n^2 \pi^2 - 3/4, n = 1, 2, \ldots.
            \end{align*}
            Se $B_1 = 0$, ent\~{a}o $k = n \pi$ e portanto
            \begin{align*}
                y_n(x) = \exp(-x/2) \sin(n \pi x), n = 1, 2, \ldots.
            \end{align*}
        \end{solution}

        \part $\begin{cases}
            y'' + \lambda y = 0, & -1 < x < 1, \\
            y(-1) = y(1), \\
            y'(-1) = y'(1).
        \end{cases}$
        \begin{solution}
            Para $\lambda = 0$ temos que
            \begin{align*}
                y'' + \lambda y = y'' = 0.
            \end{align*}
            Logo,
            \begin{align*}
               y(x) &= A x + B, \\
               y'(x) &= A.
            \end{align*}
            Quando $x = -1$ temos
            \begin{align*}
                y(-1) &= -A + B, \\
                y'(-1) &= A.
            \end{align*}
            Quando $x = 1$ temos
            \begin{align*}
                y(1) &= A + B, \\
                y'(1) &= A.
            \end{align*}
            Portanto, para $\lambda = 0$ temos apenas a solu\c{c}\~{a}o tivial.

            Para $\lambda > 0$, fazendo $\lambda = k^2$ com $k > 0$,  temos que
            \begin{align*}
                y'' + \lambda y = y'' + k^2 y = 0.
            \end{align*}
            Logo,
            \begin{align*}
                y(x) &= A \cos(k x) + B \sin(k x), \\
                y'(x) &= - A k \sin(k x) + B k \cos(k x).
            \end{align*}
            Quando $x = -1$ temos
            \begin{align*}
                y(-1) &= A \cos(-k) + B \sin(-k) = A \cos(k) - B \sin(k), \\
                y'(-1) &= -A k \sin(-k) + B k \cos(-k) = A k \sin(k) + B k \cos(k).
            \end{align*}
            Quando $x = 1$ temos
            \begin{align*}
                y(1) &= A \cos(k) + B \sin(k), \\
                y'(1) &= - A k \sin(k) + B k \cos(k).
            \end{align*}
            Portanto, para $\lambda > 0$ temos, al\'{e}m da solu\c{c}\~{a}o trivial, $y_n(x) = \cos(n \pi x)$ e $y_n(x) = \sin(n \pi x)$ onde $\lambda n = n^2 \pi^2$ e $n = 1, 2, \ldots$.

            Para $\lambda < 0$, fazendo $\lambda = - k^2$ com $k > 0$, temos que
            \begin{align*}
                y'' + \lambda y = y'' - k^2 y = 0.
            \end{align*}
            Logo,
            \begin{align*}
                y(x) &= A \cosh(k x) + B \sinh(k x), \\
                y'(x) &= - A k \sinh(k x) + B k \cosh(k x).
            \end{align*}
            Quando $x = -1$ temos
            \begin{align*}
                y(-1) &= A \cosh(-k) + B \sinh(-k), \\
                y'(-1) &= -A k \sinh(-k) + B k \cosh(-k).
            \end{align*}
            Quando $x = 1$ temos
            \begin{align*}
                y(1) &= A \cosh(k) + B \sinh(k), \\
                y'(1) &= - A k \sinh(k) + B k \cosh(k).
            \end{align*}
            Portanto, para $\lambda < 0$ n\~{a}o temos solu\c{c}\~{a}o.

            Assim sendo, temos, al\'{e}m da solu\c{c}\~{a}o tivial, $y_n(x) = \cos(n \pi x)$ e $y_n(x) = \sin(n \pi x)$ onde $\lambda n = n^2 \pi^2$ e $n = 1, 2, \ldots$ quando $\lambda > 0$.
        \end{solution}

        \part $\begin{cases}
            y'' - 3y' + 3(1 + \lambda)y = 0, & 0 < x < \pi, \\
            y(0) = 0, \\
            y(\pi) = 0.
        \end{cases}$
        \begin{solution}
            Sabendo que $y = \exp(r x)$ \'{e} solu\c{c}\~{a}o temos que
            \begin{align*}
                0 &= r^2 - 3 r + 3 (1 + \lambda) \Rightarrow r = \left( 3 \pm \sqrt{-3 - 12 \lambda} \right) 2^{-1}.
            \end{align*}

            Para $-3 - 12 \lambda = 0$, temos que
            \begin{align*}
                y(x) = A \exp(3 x 2^{-1}) + B x \exp(3 x 2^{-1}).
            \end{align*}
            Quando $x = 0$ temos
            \begin{align*}
                0 &= y(0) = A.
            \end{align*}
            Quando $x = \pi$ temos
            \begin{align*}
                0 &= y(\pi) = B \pi \exp(3 \pi 2^{-1}) \Rightarrow B = 0.
            \end{align*}
            Portanto, para $-3 - 12 \lambda = 0$ temos apenas a solu\c{c}\~{a}o trivial.

            Para $-3 - 12 \lambda > 0$, temos que
            \begin{align*}
                y(x) &= C \exp\left( (3 + k) x 2^{-1} \right) + D \exp\left( (3 - k) x 2^{-1} \right).
            \end{align*}
            Quando $x = 0$ temos
            \begin{align*}
                0 &= y(0) = C + D \Rightarrow C = - D.
            \end{align*}
            Quando $x = \pi$ temos
            \begin{align*}
                0 &= y(\pi) = C \left[ \exp\left( (3 + k) \pi 2^{-1} \right) - \exp\left( (3 - k) \pi 2^{-1} \right) \right] \\
                &= C \exp(3/2) \left[ \exp(k \pi 2^{-1}) - \exp(- k \pi 2^{-1}) \right] \Rightarrow C = 0.
            \end{align*}
            Portanto, para $-3 - 12 \lambda > 0$ temos apenas a solu\c{c}\~{a}o trivial.


            Para $-3 - 12 \lambda < 0$ temos
            \begin{align*}
                y(x) = E \exp\left( (3 + k i) x 2^{-1} \right) + F \exp\left( (3 - k i) x 2^{-1} \right).
            \end{align*}
            Quando $x = 0$ temos
            \begin{align*}
                0 &= y(0) = E + F \Rightarrow E = - F.
            \end{align*}
            Quando $x = \pi)$ temos
            \begin{align*}
                0 &= y(\pi) = E \exp(3 \pi 2^{-1}) \left[ \exp(k i \pi 2^{-1}) - \exp(- k i \pi 2^{-1}) \right] \\
                &= E \exp(3 \pi 2^{-1}) \left[ \cos(k \pi 2^{-1}) + i \sin(k \pi 2^{-1}) - \cos(k \pi 2^{-1}) + i \sin(k \pi 2^{-1}) \right] \\
                &= \sin(k \pi 2^{-1}) \Rightarrow k = 2n.
            \end{align*}
            Portanto, para $-3 - 12 \lambda < 0$ temos, al\'{e}m da solu\c{c}\~{a}o trivial, $y(x) = \exp(3 x 2^{-1}) \sin(n x)$ onde $k = 2n = \sqrt{- 3 - 12 \lambda}$.

            Assim sendo, temos, al\'{e}m da solu\c{c}\~{a}o trivial, $y(x) = \exp(3 x 2^{-1}) \sin(n x)$, onde $k = 2n = \sqrt{- 3 - 12 \lambda}$, quando $-3 - 12 \lambda < 0$.
        \end{solution}

        \part[P3 de 2006] $\begin{cases}
            \frac{\mathrm{d}}{\mathrm{d}x} \left[ (2 + x)^2 \frac{\mathrm{d}y}{\mathrm{d}x} \right] = -\lambda y, & -1 < x < 1, \\
            y(-1) = 0, \\
            y(1) = 0.
        \end{cases}$
        \begin{solution}
            Temos que
            \begin{align*}
                \frac{\mathrm{d}}{\mathrm{d}x}\left[ (2 + x)^2 \frac{\mathrm{d}y}{\mathrm{d}x} \right] &= (2 + x)^2 \frac{\mathrm{d}^2y}{\mathrm{d}x^2} + 2 (2 + x) \frac{\mathrm{d}y}{\mathrm{d}x} = -\lambda y
            \end{align*}
            \'{e} uma equa\c{c}\~{a}o de Guler. Logo, $y = (2 + x)^r$ e portanto
            \begin{align*}
                r (r - 1) + 2 r + \lambda = 0
            \end{align*}
            que implica em $r = \left( -1 \pm \sqrt{1 - 4 \lambda} \right) 2^{-1}$.

            Para $1 - 4 \lambda > 0$ temos que $r_1 = \left( -1 + \sqrt{1 - 4 \lambda} \right) 2^{-1}$ e $r_2 = \left( -1 - \sqrt{1 - 4 \lambda} \right) 2^{-1}$ e portanto
            \begin{align*}
                y(x) &= A_1 (2 + x)^{r_1} + A_2 (2 + x)^{r_2}.
            \end{align*}
            Quando $x = -1$,
            \begin{align*}
                0 = y(-1) = A_1 + A_2 \Rightarrow A_1 = - A_2.
            \end{align*}
            Quando $x = 1$,
            \begin{align*}
                0 = y(1) = A_1 3^{r_1} + A_2 3^{r_2} = A_1 (3^{r_1} - 3^{r_2}) \Rightarrow A_1 = 0 \Rightarrow = 0.
            \end{align*}
            Assim concluimos que para $1 - 4 \lambda > 0$ temos apenas a solu\c{c}\~{a}o trivial.

            Para $1 - 4 \lambda = 0$ temos que $r_1 = r_2 = -1/2$ e portanto
            \begin{align*}
                y(x) = A_3 (2 + x)^{-1/2} + A_4 (2 + x)^{-1/2} \ln(2 + x).
            \end{align*}
            Quando $x = -1$,
            \begin{align*}
                0 = y(-1) = A_3 \Rightarrow A_3 = 0.
            \end{align*}
            Quando $x = 1$,
            \begin{align*}
                0 = y(1) = A_4 3^{-1/2} \ln(3) \Rightarrow A_4 = 0.
            \end{align*}
            Assim concluimos que para $1 - 4 \lambda = 0$ temos apenas a solu\c{c}\~{a}o trivial.

            Para $1 - 4 \lambda < 0$ fazemos $1 - 4 \lambda = - k^2$, $k > 0$. Temos ent\~{a}o que $r = \left( -1 \pm i k \right) 2^{-1}$ e portanto
            \begin{align*}
                y(x) &= B_1 (x + 2)^{-1/2 + i k 2^{-1}} + B_2 (x + 2)^{-1/2 - i k 2^{-1}} \\
                &= (x + 2)^{-1/2} \left[ B_1 \exp(i k 2^{-1} \ln(x + 2)) + B_2 \exp(-i k 2^{-1} \ln(x + 2)) \right] \\
                &= (x + 2)^{-1/2} \left[ C_1 \cos(k 2^{-1} \ln(x + 2)) + C_2 \sin(k 2^{-1} \ln(x + 2)) \right].
            \end{align*}
            Quando $x = -1$,
            \begin{align*}
                0 = y(-1) = C_1 \Rightarrow C_1 = 0.
            \end{align*}
            Quando $x = 1$,
            \begin{align*}
                0 = y(1) = 3^{-1/2} C_2 \sin(k 2^{-1} \ln(3)) \Rightarrow k 2^{-1} \ln(3) = n \pi, n = 1, 2, \ldots.
            \end{align*}
            Logo,
            \begin{align*}
                1 - 4 \lambda = -\left( 2 n \pi / \ln(3) \right)^2
            \end{align*}
            que implica em
            \begin{align*}
                \lambda_n = 1/4 + \left( n \pi / \ln(3) \right)^2, n = 1, 2, \ldots.
            \end{align*}
            Assim concluimos que para $1 - 4 \lambda < 0$ temos como solu\c{c}\~{a}o
            \begin{align*}
                y_n(x) = (2 + x)^{-1/2} \sin\left( n \pi (\ln(3))^{-1} \ln(2 + x) \right).
            \end{align*}
        \end{solution}

        \part $\begin{cases}
            x^2 y'' + x y' + \lambda y = 0, & 0 < x < 1, \\
            y(1) = 0, \\
            \lim_{x \to 0^+} | y(x) | < \infty.
        \end{cases}$
        \begin{solution}
            
        \end{solution}

        \part $\begin{cases}
            y'' + \lambda y = 0, & 0 < x < \infty, \\
            y(0) = 0, \\
            \lim_{x \to \infty} | y(x) | < \infty.
        \end{cases}$
        \begin{solution}
            
        \end{solution}

        \part $\begin{cases}
            x y '' + y' + \lambda x y = 0, & 0 < x < 1, \\
            y(1) = 0, \\
            \lim_{x \to 0^-} | y(x) | < \infty.
        \end{cases}$
        \begin{solution}
            
        \end{solution}
        
        \part $\begin{cases}
            \frac{\mathrm{d}}{\mathrm{d}x} \left[ (x - a) (b - x) \frac{\mathrm{d}y}{\mathrm{d}x} \right] + \lambda y = 0, & a < x < b, \\
            \lim_{x \to a} | y(x) | < \infty, \\
            \lim_{x \to b} | y(x) | < \infty.
        \end{cases}$
        \begin{solution}
            
        \end{solution}

        \part $\begin{cases}
            y'' - y = 0, & 0 < x < 1, \\
            y(0) = 0, \\
            y(1) = 1.
        \end{cases}$
        \begin{solution}
            
        \end{solution}

        \part $\begin{cases}
            y'' + 4 y' + 7y = 0, & 0 < x < 1, \\
            y(0) = 0, \\
            y'(1) = 1.
        \end{cases}$
        \begin{solution}
            
        \end{solution}

        \part $\begin{cases}
            y'' - 6 y' + 25 y = 0, & 0 < x < \pi/4, \\
            y'(0) = 1, \\
            y(\pi/4) = 0.
        \end{cases}$
        \begin{solution}
            
        \end{solution}

        \part $\begin{cases}
            y'' + 2 y' + y = x, & 0 < x < 2, \\
            y(0) = 0, \\
            y(2) = 3.
        \end{cases}$
        \begin{solution}
            
        \end{solution}

        \part $\begin{cases}
            \left( x^2 y' \right)' + \lambda x^{-2}y = 0, & 1 < x < 2, \\
            y(1) = 0,\\
            y(2) = 0.
        \end{cases}$
        \begin{solution}
            
        \end{solution}

        \part $\begin{cases}
            x^2 y'' + xy' + \left( \lambda x^2 - 1/4 \right)y = 0, & \pi/2 < x < 3 \pi 2^{-1}, \\
            y(\pi/2) = 0, \\
            y(3 \pi 2^{-1}) = 0.
        \end{cases}$
        \begin{solution}
            
        \end{solution}

        \part $\begin{cases}
            y^{(4)} - \lambda y = 0, & 0 < x < 1, \\
            y(0) = y''(0) = y(1) = y''(1) = 0.
        \end{cases}$
        \begin{solution}
            
        \end{solution}

        \part $\begin{cases}
            y^{(4)} - \lambda y = 0, & 0 < x < 1, \\
            y(0) = y'(0) = y''(1) = y'''(1) = 0.
        \end{cases}$
        \begin{solution}
            
        \end{solution}

        \part $\begin{cases}
            (1 - x^2) y'' - 2 x y' + \lambda y = 0, & 0 < x < 1, \\
            y(0) - 0, \\
            \lim_{x \to 1} | y(x) | < \infty.
        \end{cases}$
        \begin{solution}
            
        \end{solution}
    \end{parts}

    \question Encontre os autovalores e autofun\c{c}\~{o}es do problema
    \begin{align*}
        \begin{cases}
            x^2 y'' + 3 x y' - y = - \lambda y, & 1 < x < 2, \\
            y(1) = 0, \\
            y(2) = 0.
        \end{cases}
    \end{align*}
    \begin{solution}
        Temos que
        \begin{align*}
            x^2 y'' + 3 x y' + (\lambda - 1) y &= 0
        \end{align*}
        \'{e} uma equa\c{c}\~{a}o de Euler de modo que $y = x^r$ e portanto
        \begin{align*}
            r (r - 1) + 3 r + (\lambda - 1) = 0.
        \end{align*}
        Logo, $r = -1 \pm \sqrt{2 - \lambda}$.

        Para $2 - \lambda > 0$ temos que $r_1 = - 1 + \sqrt{2 - \lambda}$ e $r_2 = -1 - \sqrt{2 - \lambda}$ e portanto
        \begin{align*}
            y(x) = A_1 x^{r_1} + A_2 x^{r_2}.
        \end{align*}
        Quando $x = 1$,
        \begin{align*}
            0 = y(1) = A_1 + A_2.
        \end{align*}
        Quando $x = 2$,
        \begin{align*}
            0 = y(2) = A_1 2^{r_1} + A_2 2^{r_2}.
        \end{align*}
        Pelas duas equa\c{c}\~{a}os anteriores temos que $A_1 = A_2 = 0$ e assim concluimos que para $2 - \lambda > 0$ temos apenas a solu\c{c}\~{a}o trivial.

        Para $2 - \lambda = 0$ temos que $r_1 = r_2 = -1$ e portanto
        \begin{align*}
            y(x) = B_1 x^{-1} + B_2 x^{-1} \ln(x).
        \end{align*}
        Quando $x = 1$,
        \begin{align*}
            0 = y(1) = B_1.
        \end{align*}
        Quando $x = 2$,
        \begin{align*}
            0 = y(2) = B_1 2^{-1} + B_2 2^{-1} \ln(2).
        \end{align*}
        Pelas duas equa\c{c}\~{o}es anteriores temos que $B_1 = B_2 = 0$ e assim concluimos que para $2 - \lambda = 0$ temos apenas a solu\c{c}\~{a}o tivial.

        Para $2 - \lambda < 0$ fazemos $2 - \lambda = - k^2$, $k > 0$. Temos ent\~{a}o que $r_1 = -1 + i k$ e $r_2 = -1 - i k$ e portanto
        \begin{align*}
            y(x) &= C_1 x^{-1 + i k} + C_2 x^{-1 - i k} \\
            &= x^{-1} \left( C_1 x^{i k} + C_2 x^{-i k} \right) \\
            &= x^{-1} \left( C_1 \exp(i k \ln(x)) + C_2 \exp(-i k \ln(x)) \right) \\
            &= x^{-1} \left( C_1' \cos(k \ln(x)) + C_2' \sin(k \ln(X)) \right).
        \end{align*}
        Quando $x = 1$,
        \begin{align*}
            0 = y(1) = 1 \left( C_1' + 0 \right) \Rightarrow C_1' = 0.
        \end{align*}
        Quando $x = 2$,
        \begin{align*}
            0 = y(2) = 2^{-1} C_2' \sin(k \ln(2)) \Rightarrow k \ln(2) = n \pi, n = 1, 2, \ldots.
        \end{align*}
        Assim, os autovalores s\~{a}o $\lambda_n = 2 + k_n^2 = 2 + (n \pi / \ln(2))^2$, $n = 1, 2, \ldots$, e as autofun\c{c}\~{o}es s\~{a}o
        \begin{align*}
            y_n(x) = x^{-1} \sin \left( n \pi (\ln(2))^{-1} \ln(x) \right), n = 1, 2, \ldots.
        \end{align*}
    \end{solution}

    \question[T4 de 2010] Determine quais das equa\c{c}\~{o}es abaixo possuem solu\c{c}\~{a}o oscilat\'{o}ria (isto \'{e}, com um n\'{u}mero infinito de zeros) no intervalo $(0, \infty)$. Justifique detalhadamente suas respostas.
    \begin{parts}
        \part $y'' + (\cos x + 2) y = 0$,
        \begin{solution}
            
        \end{solution}

        \part $y'' - \exp(-x) y = 0$,
        \begin{solution}
            
        \end{solution}

        \part $y'' + x^{-1} y = 0$.
        \begin{solution}
            
        \end{solution}
    \end{parts}

    \question[T4 de 2010] Considere a equa\c{c}\~{a}o diferencial $x^2 u'' + \lambda u = 0$ no intervalo $[1, \exp(1)]$.
    \begin{parts}
        \part Determine os autovalores e autofun\c{c}\~{o}es correspondentes para o problema de Sturm-Liouville associados \`{a} EDO acima sujeita \`{a}s condi\c{c}\~{o}es de controno $u(1) = u(\exp(1)) = 0$.
        \begin{solution}
            
        \end{solution}

        \part Escreva a condi\c{c}\~{a}o de ortogonalidade entre essas autofun\c{c}\~{o}es e a verifique explicitamente (fazendo um c\'{a}lculo independente da teoria de Sturm-Liouville).
        \begin{solution}
            
        \end{solution}
    \end{parts}

    \question Considere a equa\c{c}\~{a}o diferencial $y'' - y' + \lambda y = 0$ no intervalo $[0, 1]$.
    \begin{parts}
        \part Determine os autovalores e autofun\c{c}\~{o}es correspondentes para o problema de Sturm-Liouville associados \`{a} EDO acima sujeita \`{a}s condi\c{c}\~{o}es de controno $y(0) = y(1) = 0$.
        \begin{solution}
            
        \end{solution}

        \part Escreva a condi\c{c}\~{a}o de ortogonalidade entre essas autofun\c{c}\~{o}es.
        \begin{solution}
            
        \end{solution}
    \end{parts}

    \question[P2 de 2011, E de 2011] Encontre os autovalores e autofun\c{c}\~{o}es do seguinte problema de Sturm-Liouville:
    \begin{align*}
        \begin{cases}
            x(x y')' + \lambda y = 0, & 1 < x < \exp(2\pi) \\
            y'(1) = y'(\exp(2\pi)) = 0.
        \end{cases}
    \end{align*}
    Escreva a rela\c{c}\~{a}o de orotogonalidade satisfeita por essas autofun\c{c}\~{o}es.
    \begin{solution}
        Temos que $y = x^r$ de modo que
        \begin{align*}
            r (r - 1) + r + \lambda = 0
        \end{align*}
        e portanto $ r = \pm \sqrt{- \lambda}$.

        Para $\lambda < 0$ fazemos $\lambda = - k^2$, $k > 0$. Temos ent\~{a}o que $r = \pm k$ e portanto
        \begin{align*}
            y(x) &= A x^k + B x^{-k}, \\
            y'(x) &= A k x^{k - 1} - B k x^{k - 1}.
        \end{align*}
        Quando $x = 1$,
        \begin{align*}
            y'(1) = 0 \Rightarrow A + B = 0.
        \end{align*}
        Quando $x = \exp(2\pi)$,
        \begin{align*}
            y'(\exp(2\pi)) = 0 \Rightarrow A = 0.
        \end{align*}
        Pelas equa\c{c}\~{o}es acima temos que $A = B = 0$ e assim concluimos que para $\lambda < 0$ temos apenas a solu\c{c}\~{a}o trivial.

        Para $\lambda = 0$ temos
        \begin{align*}
            y(x) &= A_1 + B_1 \ln(x), \\
            y'(x) & = B_1 x^{-1}.
        \end{align*}
        Quando $x = 1$,
        \begin{align*}
            y'(1) = 0 \Rightarrow B_1 = 0.
        \end{align*}
        Quando $x = \exp(2\pi)$,
        \begin{align*}
            y'(\exp(2\pi)) = 0.
        \end{align*}
        Pelas equa\c{c}\~{o}es acima temos que $y = A_1$ para $\lambda = 0$.

        Para $\lambda > 0$ fazemos $\lambda = k^2$, $k > 0$. Temos ent\~{a}o que $r = \pm i k$ e portanto
        \begin{align*}
            y(x) &= A_2 x^{i k} + B_2 x^{-i k} \\
            &= A_2 \exp(i k \ln(x)) + B_2 \exp(-i k \ln(x)) \\
            &= C_1 \cos(k \ln(x)) + C_2 \sin(k \ln(x)), \\
            y'(x) &= -k x^{-1} C_1 \sin(k \ln(x)) + k x^{-1} C_2 \cos(k \ln(x)).
        \end{align*}
        Quando $x = 1$,
        \begin{align*}
            y'(1) = 0 \Rightarrow C_2 = 0.
        \end{align*}
        Quando $x = \exp(2\pi)$,
        \begin{align*}
            y'(\exp(2\pi)) = 0 \Rightarrow \sin(k \ln(\exp(2\pi))) = \sin(k 2 \pi) = 0 \Rightarrow 2 k \pi = n \pi, n = 1, 2, \ldots.
        \end{align*}
        Pelas equa\c{c}\~{o}es acima temos que $y_n = \cos(n 2^{-1} \ln(x))$, $n = 1, 2, \ldots$, e $\lambda_n = n^2 / 4$.

        Associando os casos em que $\lambda = 0$ e $\lambda > 0$ temos que
        \begin{align*}
            \begin{cases}
                y_n = \cos\left( n 2^{-1} \ln(x) \right), \\
                \lambda_n = n^2 / 4, n = 0, 1, 2, \ldots.
            \end{cases}
        \end{align*}

        Quanto a rela\c{c}\~{a}o de ortogonalidade temos
        \begin{align*}
            \int_1^{\exp(2\pi)} \cos(n 2^{-1} \ln(x)) \cos(m 2^{-1} \ln(x)) x^{-1} \,\mathrm{d}x = N_n \delta_{mn}, m, n = 0, 1, 2, \ldots.
        \end{align*}
    \end{solution}
\end{questions}
\end{document}
